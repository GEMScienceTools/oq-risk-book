% OpenQuake Book Glossary 
% To cite a glossary element in a document:
%	\gls{seismicsourcedata}
%	\Gls{seismicsourcedata} - First initial is uppercase
%	\GLS{seismicsourcedata} - All initials are uppercase
%	\glspl{seismicsourcedata} - Plural
% To process the glossary:
% 	makeglossaries oqb

%
% ------- A
\newglossaryentry{areasource}{
	name = area source,
	description={A source type usually adopted to model distributed 
	seismicity. In an area source the seismicity occurrence rate 
    is assumed uniform over the source area; this produces an hazard 
    pattern with a plateau of constant hazard inside the polygon 
    delimiting the area source and values of hazard that tend to 
    decrease as we move away from the border of the source}
}
\newglossaryentry{asset}{
    name = asset,
    description={An asset is an element with a certain value, which can include buildings or population. For example, an asset can include an individual building at a given location, or a number of buildings that are grouped, co-located at a single location and classified with the same \gls{taxonomy}}.  
}
%
% ------- B
\newglossaryentry{branch}{
	name = branch,
	plural= branches,
	description={
	The simplest element in a logic tree; it belongs to a 
	\gls{branchset} where it represents one possible option among a finite 
	number of alternatives. A branch is associated with a weight 
	value \citep{scherbaum2011} if the \gls{branchset} represents the 
	epistemic uncertainty on a parameter or a model when the \gls{branchset} 
	is used to specify alternative models (e.g. district \glspl{acr:mfd})
	}
}
\newglossaryentry{branchinglevel}{
	name = branching level,
	description={It indicates the position where a \gls{branchset} or a 
	\gls{branch} is located in a logic tree structure. For example, 
	the first branching level of the 
	\gls{seismicsourcelogictree} always contains one or several 
	\glspl{initialseismicsourceinputmodel}
	}
}
\newglossaryentry{branchset}{
	name = branch set,
	description={The structure describing the epistemic uncertainty on 
	a specific parameter or model included in a logic tree structure. 
	It ensembles a number of \glspl{branch}, each one representing a 
	discrete alternative}
}
%
% ------- C
\newacronym{cpsha}{cPSHA}{Classical PSHA}
\newglossaryentry{configurationfile}{
	name =  configuration file,
	description = {
	Usually the file containing the information necessary to run a calculation
	in OpenQuake
	}
}
\newglossaryentry{charfaultsource}{
	name = characteristic fault source,
	description={
	A fault source typology where ruptures always cover the entire fault surface
	}
}
\newglossaryentry{complexfaultsource}{
	name = complex fault source,
	description={
	A source typology usually adopted to model subduction interface faults
	}
}
%
% ------- D

\newglossaryentry{deductible}{
	name = deductible,
	description = {A parameter used in the calculation of insured losses that establishes the economic value that needs to be deducted from the ground-up losses.}
}

\newglossaryentry{seismichazarddisaggregation}{
	name =  seismic hazard disaggregation,
	description = {
	A methodology to investigate the contributions to a specific
	level of hazard in terms of fundamental variables commonly used
	to characterize seismic sources and ground motion models (e.g. 
	magnitude, source-site distance, \gls{epsilon}}
}
\newglossaryentry{dip}{
	name = dip,
	description={
    The dip is the steepest angle of descent of the fault plane
    relative to a horizontal plane; it is measured in degrees [0,90] 
	}
}
\newglossaryentry{disaggregationmatrix}{
	name =  disaggregation matrix,
	description = {
	A multi-dimensional matrix used to systematically store the contributions
	to a level of hazard to be disaggregated and that is specified by the 
	user.
	See also \gls{seismichazarddisaggregation}}
}
%
% ------- E
\newacronym{acr:erf}{ERF}{Earthquake\- Rup\-ture\- Forecast}
\newacronym{acr:epsha}{ePSHA}{Event-based PSHA}
%
\newglossaryentry{earthquakeruptureforecast}{
	name = earthquake rupture forecast,
	description={
	A list of all possible ruptures generated by all the sources included 
	in a seismic source model. Each element in the list contains: the rupture 
	geometry and the rupture probability of occurrence in a given time span. 
	%
	See also the definition available on the 
	\href{http://www.opensha.org/glossary-earthquakeRuptureForecast}
	{OpenSHA website}}
}
\newglossaryentry{earthquakeruptureforecastcalculator}{
	name = earthquake rupture forecast calculator,
	description={
	Calculator producing a \gls{seismicsourcemodel} from a 
	\gls{seismicsourcelogictree} 
	}
}
%
\newglossaryentry{epsilon}{
	name = epsilon,
	description={
	normalized residual of the ground motion}
}
\newglossaryentry{exposure model}{
	name = exposure model,
	description={
	A set of \glspl{asset} grouped according to their geographical location, 
	\gls{taxonomy} and value}
}
%
% ------- F
\newglossaryentry{faulttrace}{
	name = fault trace,
	description={A curve representing the intersection between the surface 
    containing the fault surface (or its prolongation) 
	and the topographic surface.
    \begin{figure}[!ht]
    \centering
    \includegraphics[width=10cm]{./figures/hazard/single_rupture.pdf}
    \end{figure}
    }
}
\newglossaryentry{fragility function}{
	name = fragility function,
	description = {the probability of exceeding a set of limit states, 
	given an intensity measure level. These functions can be discrete or
	continuous}
}
\newglossaryentry{fragility model}{
	name = fragility model,
	description = {A set of \glspl{fragility function} used to model the 
	fragility of all the \glspl{asset} in the \gls{exposure model}.}
}
\newglossaryentry{frequencymagnitudedistribution}{
	name = magnitude-frequency distribution,
	description = {See \gls{mfd}
	}
}
%
% ------- G
\newacronym{acr:gem}{GEM}{Global Earthquake Model}
\newacronym{acr:gmf}{GMF}{Ground Motion Field}
\newacronym{acr:gmpe}{GMPE}{Ground Motion Prediction Equation}
\newacronym{acr:gmlt}{GMLT}{Ground Motion Logic Tree (see 
    \gls{groundmotionlogictree}} 
\newglossaryentry{gridsource}{
	name = grid source,
	description={
	It's a source typology usually adopted to model distributed 
	seismicity. It's routinely produced by a seismicity smoothing 
	algorithm (one of the most famous algorithm is the one proposed 
	by \citet{frankel1995})}
}
\newglossaryentry{groundmotionfield}{
	name = ground-motion field,
	description={An object describing the geographic distribution around 
	a rupture of a ground motion intensity measure}
}
\newglossaryentry{groundmotionfieldcalc}{
	name = ground-motion field calculator,
	description={An \gls{acr:oqe} calculator that given a rupture computes the 
	geographic distribution of a ground motion intensity parameter. Currently
	OQ can generate ground motion fields using a \gls{acr:gmpe}}
}
\newglossaryentry{groundmotionlogictree}{
	name = ground-motion logic tree,
	description={A method used to systematically describe the epistemic 
	uncertainties related to the ground motion models used in the 
	computation of hazard using a specific \gls{pshainputmodel}}
}
\newglossaryentry{groundmotionmodel}{
	name = ground-motion model,
	description={An object that given a rupture with specific properties
	computes the expected ground motion at the given site. In simplest case 
	a ground motion model corresponds to a \gls{groundmotionpredictioneq}. 
	In case of complex PSHA input models, the produced ground motion models 
	contains a set of \glspl{acr:gmpe}, one for each tectonic region considered.
	}
}
\newglossaryentry{groundmotionparameter}{
	name = ground-motion parameter,
	description={A scalar or vector quantity describing a relevant property
	    of the shaking such as intensity (e.g. PGA or Spectral Acceleration) 
	    or duration, equivalent number of cycles 
	    \citep[see for example][]{hancock2005})
	}
}
\newglossaryentry{groundmotionpredictioneq}{
	name = ground-motion prediction equation,
	description={
		An equation that - given some fundamental parameters characterizing 
		the source, the propagation path and the site (in the simplest 
		case magnitude, distance and V$_\text{S,30}$) - computes the 
		value $GM$ of a (scalar) ground motion intensity parameter.
	}
}
\newglossaryentry{groundmotionsystem}{
	name = ground-motion system,
	description={An object containing a list of \gls{groundmotionlogictree}}
}
%
% ------- I 
\newglossaryentry{initialseismicsourceinputmodel}{
	name = initial seismic source input model,
	description={It's the ensable of information needed to fully describe 
        the seismic sources composing a seismic source input model. The 
        initial seismic source input model is included in the first 
	    branching level of a seismic source logic tree}
}

\newglossaryentry{insured losses}{
	name = insured losses,
	description = {Fraction of the ground-up losses that can be covered by the insurance industry, according to a certain policy.}
}

\newglossaryentry{investigationtime}{
	name = investigation time,
	description={The time interval considered to calculate hazard; usually 
	it corresponds to 50 years}
}
%
% ------- L
\newglossaryentry{limit}{
	name = limit,
	description = {A parameter used in the calculation of insured losses that establishes the maximum economic amount that can be covered by the insurance industry, according to a certain insurance policy.}
}
\newglossaryentry{logictree}{
	name = logic tree,
	description={Data structure used to systematically describe uncertainties
	on parameters and models used in a PSHA study}
}
\newglossaryentry{logictreeprocessor}{
	name = logic tree processor,
	description={An OQ calculator that takes the PSHA Input Model and creates 
	many realisations of a \gls{seismicsourcemodel} and of a 
	\gls{groundmotionmodel}}
} 
\newacronym{acr:ltmcs}{LTMCS}{Logic Tree Monte Carlo Sampler}
%
%
\newglossaryentry{msr}{
    name = magnitude-scaling relationship,
    description={
        An empirical relationship linking the magnitude with a parameter
        describing the size of the corresponding rupture (e.g. the area 
        of the rupture or the rupture length)
    }
}
\newacronym{acr:mfd}{MFD}{Magnitude-Frequency Distribution}
\newglossaryentry{mfd}{
    name = magnitude-frequency distribution,
    description={
   	    A distribution describing the frequency of earthquakes with 
        a specific magnitude. It can be continuous or discrete. 
        One frequency-magnitude distribution frequently adopted in 
        \gls{acr:psha} is the double truncated Gutenberg-Richter 
        distribution
    }
}
%
% ------- O
\newacronym{acr:hazlib}{oq-hazardlib}{OpenQuake hazard library}
\newglossaryentry{opensha}{
	name = OpenSHA,
	description = {OpenSHA is an open-source, advanced Java-based 
	platform for conducting Seismic Hazard Analysis - 
	(see \href{http://opensha.org}{OpenSHA website})}
}
\newacronym{acr:oqe}{oq-en\-gine}{OpenQuake-engine}
%
% ------- P
\newacronym{acr:pga}{PGA}{Peak Ground Acceleration}
\newacronym{acr:pgv}{PGV}{Peak Ground Velocity}
\newacronym[description={\glslink{psha}{Probabilistic 
    Seismic Hazard Analysis}}]{acr:psha}{PSHA}{Probabilistic Seismic 
    Hazard Analysis}
\newglossaryentry{pointsource}{
	name=point source, 
	description={The elemental source typology used in OpenQuake-engine to 
	model distributed seismicity}	
}
\newglossaryentry{pshainputmodel}{
	name=PSHA input model, 
	description={An object containing the information necessary to describe 
	the seismic source and the ground motion models - plus the related 
	epistemic uncertainties}	
}
\newglossaryentry{psha}{
	name = probabilistic seismic hazard analysis, 
	description={A methodology to compute seismic hazard by taking into 
	account the potential contributions coming from all the sources 
	of engineering importance for a specified site}	
}
%
% ------- R
\newacronym{acr:rrup}{$\text{r}_{\text{rup}}$}{closest distance between 
	the site and rupture}
\newglossaryentry{rupture}{
	name=earthquake rupture, 
	description={A 3D surface - representing a portion or the entire 
    fault surface - over which a slip event (i.e. an earthquake) 
    occurs.
	}	
}
\newglossaryentry{ruptureaspectratio}{
	name=rupture aspect ratio, 
	description={It's the ratio between the lenght and the width of an
		earthquake rupture
	}	
}
\newglossaryentry{rake}{
	name=rake, 
	description={The 
	}	
}
%
% ------- S
\newacronym{acr:ssha}{SSHA}{Scenario Based Seismic Hazard Analysis}
\newglossaryentry{scenariohazard}{
	name = scenario based seismic hazard analysis,
	plural= scenario based seismic hazard analyses,
	description = {An analyis of seismic hazard based on the selection of
        one or a few ruptures and the computation of the expected ground 
        motion at a set of sites using a \gls{gmpe} accounting ground motion
        variability.
	}
}
\newglossaryentry{seismicityhistory}{
	name = seismicity history,
	plural= seismicity histories,
	description = {An object containing a set ruptures  
	representative of the possible seismicity generated by the 
	sources in a \gls{seismicsourcemodel} during the investigation 
	time $t$
	}
}
\newglossaryentry{seismicityrate}{
	name = seismicity rate,
	description = {Number of events per unit of time (if not better 
	specified, the definition of a seismicity rate generally presumes 
	a time independent 
	}
}
\newglossaryentry{seismicsourcedata}{
	name = seismic source data,
	description={An object containing the information necessary 
	to completely describe a \gls{acr:psha} seismic source i.e. seismic 
	source type, position, geometry and seismicity occurrence 
	model}
}
\newglossaryentry{seismicsourcelogictree}{
	name = seismic source logic tree,
	description={Logic tree structure defined to describe in 
	structured and systematic way the epistemic uncertainties 
	characterizing the seismic source model. The first 
	branching level in the logic tree by definition contains one or
	several alternative \gls{initialseismicsourceinputmodel}}
}
\newacronym{acr:ssim}{SSIM}{Seismic Source Input Model}
\newglossaryentry{seismicsourceinputmodel}{
	name = seismic source input model,
	description={An object containing a list of \gls{seismicsourcedata}.
    In the OpenQuake-engine a seismic source model doesn't contain 
    epistemic uncertainty}
}
\newglossaryentry{seismicsource}{
	name = seismic source,
    description={An object that can generate}
}
\newacronym{acr:ssm}{SSM}{Seismic Source Model}
\newglossaryentry{seismicsourcemodel}{
	name = seismic source model,
	description={An object containing a list of \glspl{seismicsource}
    objects}
}
\newacronym{acr:scec}{SCEC}{Southern California Earthquake Center}
\newglossaryentry{seismicsourcesystem}{
	name = seismic source system,
	description={An object containing a list of 
        \glspl{initialseismicsourceinputmodel}
	and the \gls{seismicsourcelogictree}}
}
\newglossaryentry{simplefaultsource}{
	name = simple fault source,
	description={
	A source typology usually adopted to model shallow structures with an
	uncomplicated geometry
	}
}
\newacronym{acr:ses}{SES}{Stochastic Event Set}
\newglossaryentry{stochasticeventset}{
	name = stochastic event set,
	description={An object containing one or many \glspl{seismicityhistory} 
	}
}
\newglossaryentry{strike}{
	name = strike,
	description={
	The strike direction correspond to the angle between the north and
	the direction you take so that when you walk along the \gls{faulttrace}
	the fault dips on your right. 
	}
}
\newacronym{acr:sa}{S$_a$}{Spectral Acceleration}
%
% ------- T
\newglossaryentry{taxonomy}{
	name = taxonomy,
	description = {Scheme used to classify the \glspl{asset}. For buildings,
	a classification scheme has been proposed by \gls{acr:gem} which considers
	a number of attributes including lateral load resisting system and its 
	material, height, year of construction. The taxonomy is currently used to 
	link the \glspl{asset} in the \gls{exposure model} to the relevant 
	\gls{vulnerability function} or \gls{fragility function}}
}
\newglossaryentry{tectonicregion}{
	name = tectonic region,
	description = {A area on the topographic surface that can be considered 
	homogeneous in terms of tectonic properties such as the prevalent 
	seismogenic properties and/or the seismic wave propagation properties
	}
}
\newglossaryentry{temporaloccurrencemodel}{
	name = temporal occurrence model,
	description = {Usually a probabilistic model giving the probability of
	occurrence of an event in a specified \gls{investigationtime}
	}
}
%
% ------- U
\newacronym{acr:usgs}{USGS}{United States Geological Survey}
%
% ------- V 
\newglossaryentry{vulnerability function}{
	name = vulnerability function,
	description = {A function that describes the probability distribution
	of loss ratio, conditioned on an intensity measure level. Currently only 
	discrete vulnerability functions are supported}
}
\newglossaryentry{vulnerability model}{
	name = vulnerability mod\-el,
	description = {A set of \glspl{vulnerability function} used to model the 
	physical vulnerability of all the \glspl{asset} in the \gls{exposure model}}
}
\newglossaryentry{acr:vs30}{
	name = V$_{S,30}$,
	description = {Average shear wave velocity of the materials in the uppermost 30m of the soil column}
}
