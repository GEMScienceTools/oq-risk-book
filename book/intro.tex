% -----------------------------------------------------------------------------
\section{OpenQuake-engine introduction}
\gls{acr:oqe} is the seismic hazard and risk calculation software developed by
the Global Earthquake Model. By following current standards in software developments, like test-driven development and continuous
integration, OpenQuake aims a becoming an open, and community-driven tool for
seismic hazard and risk analysis.

The source code of the OpenQuake-engine is available on a public web-based repository
at the following address \href{http://github.com/gem/oq-engine}{http://github.com/gem/oq-engine}.
% -----------------------------------------------------------------------------
\section{Running the OpenQuake-engine}
\index{Running OpenQuake!introduction}
\label{sec:intro}
The execution of an analysis using the OpenQuake-engine is launched from 
the command line of a terminal. A schematic list of the options that 
can be used for the execution of the \gls{acr:oqe} can be obtained 
with the following command:
\begin{Verbatim}[frame=single, commandchars=\\\{\}, fontsize=\small]
user@ubuntu:~$ oq-engine --help
\end{Verbatim}
The result is the following:
\VerbatimInput[frame=single, commandchars=\\\{\}, fontsize=\scriptsize]{./oqum/hazard/verbatim/help.txt}