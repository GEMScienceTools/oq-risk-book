% ------------------------------------------------------------------------------
% The OpenQuake-engine Risk Quality Assurance Report
%
% Authors:
%  Anirudh Rao         - GEM Model Facility, Pavia, Italy
%
% License:
% Document distributed under the CC BY-NC-SA 4.0 License
% Creative Commons Attribution-NonCommercial-ShareAlike 4.0 International
% http://creativecommons.org/licenses/by-nc-sa/4.0/
%
% Copyright:
% © GEM Foundation, Pavia, Italy. August 2015.
%
%-------------------------------------------------------------------------------
%  PACKAGES AND OTHER DOCUMENT CONFIGURATIONS
%-------------------------------------------------------------------------------

\documentclass[11pt,fleqn]{book} % ------------------ Left-justified equations -
\input{configuration/configuration.tex} % -------------------- Load packages -
%-------------------------------------------------------------------------------
The goal of this book is to provide a comprehensive and transparent description
of the features of the OpenQuake-engine (v1.2). This book is designed to be 
readable by someone with basic understanding of Probabilistic Seismic Hazard 
and Risk Analysis, but no previous knowledge of the OpenQuake-engine is assumed.
%
The OpenQuake-engine is an effort promoted and actively developed by the
Global Earthquake Model, a public-private partnership initiated by the Global
Science Forum of the Organisation for Economic Co-operation and Development 
(OECD)\footnote{A short description of the process promoted by OECD is available 
here http://www.oecd.org/science/sci-tech/theglobalearthquakemodelgem.htm}.

The OpenQuake-engine is the result of an effort carried out jointly by the
Information Technology and Scientific teams working at the GEM Secretariat. 
It is freely distributed under an Affero GPL license 
(more information available at this link 
\href{http://www.gnu.org/licenses/agpl-3.0.html}{http://www.gnu.org/licenses/agpl-3.0.html})
%------------------------------------------------------------------------------- % ------------------------ Load template -

\begin{document}
\input{book/glossary.tex} % ---------------------------------- Load glossary -

%-------------------------------------------------------------------------------
%  COVER PAGE
%-------------------------------------------------------------------------------

\includepdf[pages=-]{figures/oq-risk-qa-cover.pdf}

%-------------------------------------------------------------------------------
%  TITLE PAGE
%-------------------------------------------------------------------------------

\begingroup
\thispagestyle{empty}
%\AddToShipoutPicture*{\put(6,5){\includegraphics[scale=1]{background}}}
\par\normalfont\fontsize{15}{15}\sffamily\selectfont
“OpenQuake: Calculate, share, explore”
\centering
\vspace*{9cm}
\par\normalfont\fontsize{35}{35}\sffamily\selectfont
Testing procedures adopted in the development of the risk 
component of the OpenQuake-engine\par % Book title
\endgroup

%-------------------------------------------------------------------------------
%  COPYRIGHT PAGE
%-------------------------------------------------------------------------------

\newpage
~\vfill
\thispagestyle{empty}

\noindent
   \textbf{Authors} \\
   Anirudh Rao$^1$, Michele Simionato$^1$ \hfill \\
   \hfill \\
   \small
   \begin{tabular}{p{4cm}p{4cm}p{4cm}}
   $^1$ GEM Model Facility \hfill \newline
   via Ferrata, 1 \hfill \newline
   20133 Pavia \hfill \newline
   Italy \hfill \newline
   \end{tabular} \hfill \newline
   %
   Email address (for all the authors):\hfill\\
   $<$name.surname$>$@globalquakemodel.org\hfill\\
   \normalsize

\noindent \copyright\ \textsc{2015 GEM Foundation}\\ % Copyright notice
\noindent \textsc{globalquakemodel.org/openquake}\\ % URL
\noindent \hfill\\
\noindent
   {\textbf{Citation}} \hfill \\
   Please cite this document as: \hfill \\
   Rao, Anirudh and Simionato, Michele (2015). 
   Testing procedures adopted in the development of the risk 
   component of the OpenQuake-engine. 
   \textit{Global Earthquake Model (GEM) Technical Report 2015-08. 
   doi: 10.13117/GEM.OPENQUAKE.TR2015.08, 90 pages.} \hfill \\
\noindent \hfill\\
\noindent
   {\bf{Disclaimer}} \hfill \\
   This report is distributed in the hope that it will be 
   useful, but without any warranty: without even the implied warranty of 
   merchantability or fitness for a particular purpose. While every precaution 
   has been taken in the preparation of this document, in no event shall the 
   authors of the Manual and the GEM Foundation be liable to any party for 
   direct, indirect, special, incidental, or consequential damages, including 
   lost profits, arising out of the use of information contained in this 
   document or from the use of programs and source code that may accompany it, 
   even if the authors and GEM Foundation have been advised of the possibility 
   of such damage. The report provided hereunder is on as ``as is'' basis, and the 
   authors and GEM Foundation have no obligations to provide maintenance, 
   support, updates, enhancements, or modifications. \hfill \\
% \noindent
%    The current version of the report has been revised only by members of the 
%    GEM model facility and it must be considered a draft copy.
% \vspace{0.4cm} \hfill \\
\noindent \hfill\\
\noindent
   {\bf{License}} \hfill \\
   This report is distributed under the Creative Commons License 
   Attribution-NonCommercial-ShareAlike 4.0 International 
   (\href{http://creativecommons.org/licenses/by-nc-sa/4.0/}
   {CC BY-NC-SA 4.0}). 
   You can download this Manual and share it with 
   others as long as you provide proper credit, but you cannot change 
   it in any way or use it commercially.\hfill \\

\noindent \textit{First printing, August 2015} % Printing/edition date

%-------------------------------------------------------------------------------
%  TABLE OF CONTENTS
%-------------------------------------------------------------------------------

\chapterimage{figures/chapter_head_1.pdf} % Table of contents heading image
\pagestyle{empty} % No headers
\tableofcontents % Print the table of contents itself
\cleardoublepage % Forces the first chapter to start on the right
\pagestyle{fancy} % Print headers again

%-------------------------------------------------------------------------------
%  FOREWORD
%-------------------------------------------------------------------------------
% \chapterimage{figures/chapter_head_2.pdf} % Chapter heading image
% \chapter*{Preface}
% \addcontentsline{toc}{chapter}{Preface}
% \input{qareport/preamble.tex}

%-------------------------------------------------------------------------------
%  THE QA REPORT
%-------------------------------------------------------------------------------

% ------------------------------------------------------- Part I: Introduction -
\part{Introduction}

\chapterimage{figures/chapter_head_2.pdf} % Chapter heading image
\chapter{Software Testing}
   \label{chap:intro}
   The current document describes the testing procedures adopted in the development of the hazard component of the \gls{acr:oqe}, the open source hazard and risk software developed by the Global Earthquake Model initiative.

Nowadays seismic hazard analysis serves different needs coming from a variety of users and applications.

These may encompass engineering design, assessment of earthquake risk to portfolios of assets within the insurance and reinsurance sectors, engineering seismological research, and effective mitigation via public policy in the form of urban zoning and building design code formulation.

Decisions based on seismic risk results may have impacts on
population, properties and capitals, possibly with important repercussions on our day-to-day life. For these reasons, it is recommendable that the generation of hazard models and their calculation is based on well-recognized, state-of-the-art and tested techniques, requirements that must be reconciled with the need to regularly incorporate recent advances given the progress carried out within the scientific community.

The features described below contribute to fulfill these requirements: %
\begin{itemize}
\item Software should have a modular and flexible structure capable of incorporating new features and - as a consequence - offer users the most recent and advanced techniques.
%
In very general terms, modularity is the level to which a component of a system can be moved, replaced or reused.
%
In software design, modularity means the separation of the software
into smaller independent components that can be implemented, maintained and tested easily and efficiently.
%
\item Software should have and extensive test coverage which captures possible errors and avoids regressions (i.e. unexpected behaviors introduced by new features).
%
Software testing \parencite{myers2012} is an important, complex and vast discipline which helps in developing methods and processes aimed at certifying the extent to which a computer code behaves according to the original design intent and user specifications.
\end{itemize}


\begin{figure}[!ht]
\centering
\includegraphics[width=9cm]{figures/oq_engine_structure.pdf}
\caption{A schematic describing the main components of the OpenQuake-engine software.}
\label{fig:oqe-structure}
\end{figure}


The \gls{acr:oqe} includes different levels of modularity. The first is the one separating the engine itself into a number of libraries (see Figure \ref{fig:oqe-structure}), each one containing well identified knowledge, objects and methods (e.g. the OQ-hazardlib  includes objects and methods needed to compute probabilistic seismic hazard and the OQ-risklib contains methods to compute scenario and probabilistic seismic risk).

The second one pertains to the data model adopted in the development of each library as a result of the abstraction process.

According to \textcite{berkes2012} scientific software must be:
\begin{itemize}
\item Error proof
\item Flexible and able to accommodate different methods
\item Reproducible and re-usable
\end{itemize}

\section{Testing and Quality Assurance}
\label{sec:intro-testing-qa}
Despite the distinction between software testing (in some cases also called Quality Control) and \gls{acr:sqa} being somewhat vague and partly open to personal judgment, it's clear that \gls{acr:sqa} is a more comprehensive and overarching process than software testing.
%
\gls{acr:sqa} aims at the definition of the best processes that should be used to provide guarantees that user expectations will be met.
%
Software testing focuses instead on detecting software faults by inspecting and testing the product at different stages of development.

% -----------------------------------------------------------------------------
\subsection{Software testing}
Software testing can be implemented at different stages of the development process, with varying strategies to approach the problem.
%
The \gls{acr:oqe} and the associated libraries are developed following an agile paradigm. This development strategy is organized in a way that the creation of the real code is completed in parallel and fully integrated with the software testing process.

The software engineering community provides a wide range of testing levels
and typologies. In the current document we consider just a portion of them
with the specific intent to illustrate the standards used in the development of the \gls{acr:oqe} and particularly of its risk calculation component.

% -----------------------------------------------------------------------------
\subsection{Quality assurance}
From the IEEE ``Standard for Software Quality Assurance Processes'':
\emph{Software quality assurance is a set of activities that define and assess the adequacy of software processes to provide evidence that establishes confidence that the software processes are appropriate for and produce software products of suitable quality for their intended purposes. A key attribute of SQA is the objectivity of the SQA function with respect to the project. The SQA function may also be organizationally independent of the project; that is, free from technical, managerial, and financial pressures from the project.} In this document we are not covering topics related to \gls{acr:sqa} since this would go beyond its scope.

\section{Organization of Report}
This document is organized into four chapters.

The current chapter provides a very brief and general introduction to software testing with a focus on the testing of scientific software.

The second chapter describes the module, or unit testing procedures adopted in the development of the \gls{acr:oqe} and we discuss some examples. The continuous integration mechanism used for development is also discussed.

The third chapter describes the general framework for the acceptance tests for the OpenQuake risk calculators. A brief overview of the theoretical background for the different calculators is also provided in this chapter.

The fourth chapter describes the different test cases, input models, and results for the acceptance tests implemented for the OpenQuake scenario risk, classical risk, and event-based risk calculators.

% In the fifth chapter, we compare the loss curves computed using the event-based calculator with the corresponding loss curves computed using the classical-PSHA based calculator.

% In the sixth chapter, we illustrate tests comparing the results computed with the \gls{acr:oqe} against the ones computed using different probabilistic seismic risk analysis software.

% Chapter seven describes the OpenQuake risk demos and the average 

% The final chapter describes the set of 

% -------------------------------------------------------- Part II: Unit Tests -
\thispagestyle{empty}
\part{Unit Tests}

\chapterimage{figures/chapter_head_2.pdf} % Chapter heading image
\chapter{Unit Testing in the OpenQuake-engine}
   \label{chap:unit-tests}
   This chapter provides an introduction to the module (unit) testing procedures \citep{myers2012} and describes the estensive series of tests implemented in the \gls{acr:oqe}.

\section{Overview of Unit-Testing}
\label{sec:ut-overview}
At the first level of the code testing process is the practice of ``unit-testing''. This process is a central tenet of test-driven software development and is widely established as a means of ``best-practice''. According to \citet{myers2012}, \emph{``Unit testing is a process of testing the individual subprograms, subroutines, or procedures in a program. That is, rather than initially testing the program as a whole, testing is first focused on the smaller building blocks of the program.''}.

Before looking closely at the OpenQuake-engine approach to unit-testing it is important to establish what are the precise objectives of the unit-testing process and the benefits (and limitations) that it brings.

\subsection{Correctness of implementation}
This objective is obviously the primary goal of unit-testing, to ensure that each function of the code is operating in the manner expected by the developer. ``Correctness'', in this case, requires that the function produces both the correct output, but also if there are cases in which function may fail then the means of failure should be predictable. The following is a relatively simple example of how a unit-test relates to a function:

Consider a simple function to multiply two numbers and take the logarithm of the result. A relevant analogy may be that of a magnitude scaling relation calculation, in which both a rupture length and rupture width are required, and the logarithm of the area may be needed by the function itself. In this circumstance a negative value in either of the two inputs would result in a calculation error. This could be coded in the following manner:

\begin{lstlisting}[frame=single]
def get_log_area(length, width):
    if (length < 0) or (width < 0):
        raise valueError("Both inputs must be positive")
    else:
        return log10(length * width)
\end{lstlisting}

From the description above it is evident that the user requirements inform the manner in which the function should behave (i.e. negative
values cannot be tolerated). To ensure that the function is operating correctly, we wish to write a set of tests that will confirm the behaviour is correct:

\begin{enumerate}
\item If both $a$ and $b$ are equal to 10.0, then the function     should return 2.0
\item If $a = -1$ and $b = 10$ the function should raise an error     reporting the stated message ``Both inputs must be positive''.
\item If $a = 10$ and $b = -1$ the function should raise an error     reporting the stated message ``Both inputs must be positive''.
\item If $a = -1$ and $b = -1$ the function should raise an error     reporting the stated message ``Both inputs must be positive''.
\end{enumerate}

A unit-test for this function is an additional function that will check that both cases are satisfied, and will report an error if not.

A comprehensive unit-test suite for a software may fulfil two objectives: \textbf{line coverage} and \textbf{parameter coverage}. The former should ensure that, in as far as possible, every line (or statement) in the code is executed at some point in the testing process. The latter should ensure that the behaviour of the function is predictable when supplied with ``unusual'' parameters. In the above example, both objectives are satisfied by the tests. The first test will result in a positive valued ``area'', thus executing the second branch of the logical path, the second test will result in a negative area and will execute the first logical branch. %

Therefore all lines of the code are covered and the line coverage is complete. We also see that in this simple example there are four possible cases: i) a is positive and b is positive, ii) a is positive and b is negative, ii) a is negative and b is positive, and iv) both a and b are negative. Only the first case is valid, therefore the first test ensures that they provide the correct answer (usually verified by independent means), whilst the remaining tests should ensure that the function raises the correct error. Thus the full parameter space of the input is ensured.

The above case is, of course, trivial; however, as shall be seen in due course, this same process can be applied in more complex contexts.

Furthermore, the same unit-testing approach can be applied not only to individual components within the hazard and risk calculations, but also to full calculations, essentially verifying that the hazard curves and loss curves produced by the full OpenQuake probabilistic hazard and risk calculators are in agreement with those produced independently (sometimes by hand calculations).

\subsection{Identify problems prior to software release}
This advantage is largely self-explanatory, but for many software projects this can reduce the possibility of requiring \emph{a posteriori} fixes to the code (patches). By compiling a comprehensive suite of unit-tests, and following a software development and release process that should automatically run the tests at the point of packaging, this should ensure that new features added to the software cannot inadvertently break other components.

Another critical component of software testing is catching user-input errors at the source and preventing calculations with erroneous inputs from running in the first place. Thus, the input models associated with a calculation are carefully analysed prior to the computation for errors and incompatible parameters, and an error is issued to the user.

\subsection{Facilitate improvements in performance}
In the creation of software intended to perform demanding scientific calculations, like those commonly associated with probabilistic seismic hazard and risk analysis, the issue of computational performance and efficiency is a major one. There is a continuing need to improve the speed and reduce the work required to undertake the haard and risk calculations. To implement improvements it is necessary to ensure that optimisations do not modify the outputs of the calculation, only the speed at which they are performed. Thus, unit-testing is absolutely fundamental to this process as optimisation cannot be undertaken readily without a means to ensure the calculation outputs have not changed.

\section{Continuous Integration}
\label{sec:ut-ci}
\gls{acr:oqe} is developed and packaged within a ``continuous integration'' system (\href{https://ci.openquake.org/}{https://ci.openquake.org/}),
based on the open-source software ``Jenkins'' (\href{http://jenkins-ci.org/}{http://jenkins-ci.org/}). Continuous integration is used in large software projects to run a full test suite of the complete software, either at fixed time intervals or, as in the current case, when any new code is committed to the repository. The continuous integration system does the following:

\begin{enumerate}
\item Run the full set of unit-tests for all code in all of the linked repositories. This will include the main (or ``master'') branch of the software repository, i.e. the one that will be used for packaging of the software, as well as some development branches.
\item Run a test of the software installation. This test will install the software on a dedicated platform and check that the installation of the software is successful. This test also ensures that if changes occur in the dependency packages, and these changes affect or compromise the installation and operation of the software, these problems are recognised immediately.
\item The software will also run standard Python tests for quality of code, compilation of documentation etc.
\item Several long-running tests may also be run. These implement larger scale seismic hazard and risk calculations designed to test the overall performance of the engine.
\end{enumerate}

If at any point the tests should fail, the OpenQuake development team will be notified automatically. This ensures that software that is failing any of the tests will remain on the main branch of the repository for the minimum amount of time possible. Furthermore, if the continuous integration tests fail, the new code will not be integrated into the nightly package of the software.

\section{Unit-Tests in the OpenQuake Risk Library}
\label{sec:ut-risklib}
\subsection{Component level tests}
The unit-testing at the component level breaks the functions into simple
calculations whose results can be verified by hand. These tests, similar in nature to that illustrated previously, provide the majority of the line and parameter coverage needed to ensure a robust code.

To illustrate the comprehensive nature of the coverage we consider the example of the lognormal distribution function used for representing continuous vulnerability functions:


%
The test suite for this one function is illustrative of several key components of the unit-testing. First is the use of an independent tool to provide the expected values of the calculation under simple conditions. Second is the use of ``extreme cases'' such as polar locations, or across the International Dateline. These ensure that the function can be global in application.

The nature of the interdependencies between the functions also means that one a functions own unit-test is verified, the function can then form the basis for testing other conditions. So for example, the geodetic distance tools also contain a method to calculate the minimum distance between a collection of points and a single point. Rather than requiring new expected distances for the different conditions, the geodetic distance function can then be used to construct tests for functions that utilise it. This makes the testing process more efficient, and reduces the need to write large numbers of tests in order to ensure correct behaviour of the function.

\section{Summary}
\label{sec:ut-summary}
In this chapter we have outlined both the process and the key benefits of developing comprehensive unit-tests for OpenQuake-engine, as well as outlining the operation of the continuous integration system, which should ensure that code with the potential to break the tests cannot be packaged and released. The unit-tests themselves have not been discussed in detail as nearly one thousand tests are executed during the unit-test process. However, to view the comprehensive set of tests, the reader is encouraged to refer to the full test-suite, which is open and available on the OpenQuake code repository (\href{https://github.com/gem/oq-risklib/tree/master/openquake/risklib/tests}
{https://github.com/gem/oq-risklib/tree/master/openquake/risklib/tests}).
   \cleardoublepage

% ------------------------------------------------- Part III: Acceptance Tests -
\thispagestyle{empty}
\part{Acceptance Tests}

\chapterimage{figures/chapter_head_2.pdf} % Chapter heading image
\chapter{Framework for Acceptance Testing}
   \label{chap:acc-framework}
   \section{Verification Framework}
The main purpose of the acceptance tests is to ensure that the risk calculators work according to the design specifications and to verify that the calculators produce correct results for a variety of input cases. Correctness of the test case results is verified by comparing with hand calculations for the simple test cases or with alternate implementaions in Julia for the complex cases.

\section{Theoretical Background}
\subsection{Basic concepts}
   \label{subsec:theory-basic-concepts}
   % Rupture
% GMF
% SES
% VF
% FF
% HC
% Monte Carlo
% GMPE

An earthquake \textit{rupture model} describes the magnitude, geometry, and source typology of an earthquake occurrence.

Given an earthquake rupture, the simulation of ground shaking values on a set of locations $\bm{x}=(x_{1}, x_{2}, ..., x_{N})$
forms a \textit{ground motion field} ($GMF$). The ground motion field is simulated by sampling the probability distribution defined by the ground motion model.

Seismicity in a region is described by a seismic source model ($SSM$), which is a collection of \textit{independent seismic sources}. Independence of seismic sources implies that the occurrence of an earthquake rupture in a source does not affect the probability of earthquake occurrence in the other sources.

The main parameters describing a seismic source are the geometry constraining the earthquake rupture locations, and the \textit{magnitude-frequency distribution}, defining the average annual occurrence rate over a magnitude range. A seismic source model ($SSM$) can be therefore defined as a set of $I$ seismic sources ($Src$):
\begin{equation}
SSM = \left\{Src_{1}, Src_{2}, ..., Src_{I}\right\}
\end{equation}

Each source generates \textit{independent earthquake ruptures}. Independence of earthquake ruptures implies that the occurrence of an earthquake rupture in a source does not affect the probability of occurrence of the other potential earthquake ruptures in the same source. A generic $i$-th source defines therefore a set of $J$ earthquake ruptures:
\begin{equation}
Src_{i} = \left\{Rup_{i1}, Rup_{i2}, ..., Rup_{iJ}\right\}
\end{equation}





Probabilistic seismic hazard analysis allows calculating the probabilities of exceeding,
at least once in a given time span, and at a given site, a set of ground motion
parameter levels considering all possible earthquake ruptures defined in a
seismic source model. Such a list of probability values is usually referred to
as \textit{hazard curve}.


\subsection{Scenario risk}
   \label{subsec:theory-scenario-risk}
   \input{qareport/theory-scenario-risk.tex}
\subsection{Scenario damage}
   \label{subsec:theory-scenario-damage}
   \input{qareport/theory-scenario-damage.tex}
\subsection{Classical PSHA-based risk}
   \label{subsec:theory-classical-risk}
   \input{qareport/theory-classical-risk.tex}
\subsection{Classical PSHA-based damage}
   \label{subsec:theory-classical-damage}
   \input{qareport/theory-classical-damage.tex}
\subsection{Event-based risk}
   \label{subsec:theory-eventbased}
   \input{qareport/theory-eventbased.tex}
   \cleardoublepage
   \cleardoublepage
\chapter{Test Cases and Results}
   \label{chap:acc-tests}
   \section{Scenario Risk Calculator}
   \label{sec:acc-sr}
   The tests for the scenario risk calculator assume the correct computation of the ground motion fields at the locations of the assets in the exposure model. Thus, the risk tests implicitly rely on the acceptance tests for the scenario hazard calculator.

The rupture model used for the tests comprises a magnitude $M_W 6.7$ rupture on a vertical strike-slip fault.\\

\noindent Details of the rupture are given below:\\

\noindent
Fault type: Strike slip\\
Fault dip: $90^{\circ}$\\
Fault plane depths: 0--20 km\\
Fault coordinates:\\
South end: $38.0000^{\circ} N$, $122.0000^{\circ} W$\\
North end: $38.2248^{\circ} N$, $122.0000^{\circ} W$\\
Rupture magnitude: 6.7\\
Rupture hypocenter: $38.1124^{\circ} N$, $122.0000^{\circ} W$\\
Hypocenter depth: 10 km\\

The complete collection of input models and job configuration files used in these test cases can be accessed here:
\href{https://github.com/gem/oq-risklib/tree/master/openquake/qa_tests_data/scenario_risk}
{https://github.com/gem/oq-risklib/tree/master/openquake/qa\_tests\_data/scenario\_risk}

\subsection{Single asset tests}
\label{subsec:acc-sr-single}
The single asset test cases are designed to test the basic elements of the scenario risk calculator, such as:

\begin{itemize}
\item basic loss field computation
\item calculation of mean and standard deviation of scenario loss
\end{itemize}

\input{qareport/tables/tab-asset.tex}

The location and taxonomy of the single asset in the exposure model used for the single-asset test cases for the scenario risk calculator are given in Table \ref{tab:asset}.

% ---------------------------------------------------------------------------
\subsubsection{Case 1a}
\input{qareport/cases/sr-1a}
\input{qareport/results/sr-1a}
Table \ref{tab:result-sr-1a} shows the comparison of the OpenQuake result with the expected result.

% ---------------------------------------------------------------------------
\subsubsection{Case 1b}
\input{qareport/cases/sr-1b}
\input{qareport/results/sr-1b}
Table \ref{tab:result-sr-1b} shows the comparison of the OpenQuake result with the expected result.

% ---------------------------------------------------------------------------
\subsubsection{Case 1c}
\input{qareport/cases/sr-1c}
\input{qareport/results/sr-1c}
Table \ref{tab:result-sr-1c} shows the comparison of the OpenQuake result with the expected result.

% ---------------------------------------------------------------------------
\subsubsection{Case 1d}
\input{qareport/cases/sr-1d}
\input{qareport/results/sr-1d}
Table \ref{tab:result-sr-1d} shows the comparison of the OpenQuake result with the expected result.

% ---------------------------------------------------------------------------
\subsubsection{Case 1e}
\input{qareport/cases/sr-1e}
\input{qareport/results/sr-1e}
Table \ref{tab:result-sr-1e} shows the comparison of the OpenQuake result with the expected result.

% ---------------------------------------------------------------------------
\subsubsection{Case 1f}
\input{qareport/cases/sr-1f}
\input{qareport/results/sr-1f}
Table \ref{tab:result-sr-1f} shows the comparison of the OpenQuake result with the expected result.

% ---------------------------------------------------------------------------
\subsubsection{Case 1g}
\input{qareport/cases/sr-1g}
\input{qareport/results/sr-1g}
Table \ref{tab:result-sr-1g} shows the comparison of the OpenQuake result with the expected result.

% ---------------------------------------------------------------------------
\subsubsection{Case 1h}
\input{qareport/cases/sr-1h}
\input{qareport/results/sr-1h}
Table \ref{tab:result-sr-1h} shows the comparison of the OpenQuake result with the expected result.

% ---------------------------------------------------------------------------
\subsubsection{Case 2a}
\input{qareport/cases/sr-2a}
Table \ref{tab:result-sr-2a} shows the comparison of the OpenQuake result with the expected result.

% ---------------------------------------------------------------------------
\subsubsection{Case 2b}
\input{qareport/cases/sr-2b}
Table \ref{tab:result-sr-2b} shows the comparison of the OpenQuake result with the expected result.

% ---------------------------------------------------------------------------
\subsubsection{Case 2c}
\input{qareport/cases/sr-2c}
Table \ref{tab:result-sr-2c} shows the comparison of the OpenQuake result with the expected result.

% ---------------------------------------------------------------------------
\subsubsection{Case 2d}
\input{qareport/cases/sr-2d}
\input{qareport/results/sr-2a}
\input{qareport/results/sr-2b}
\input{qareport/results/sr-2c}
\input{qareport/results/sr-2d}
Table \ref{tab:result-sr-2d} shows the comparison of the OpenQuake result with the expected result.

% ---------------------------------------------------------------------------
\subsubsection{Case 3a}
\input{qareport/cases/sr-3a}
\input{qareport/results/sr-3a}
Table \ref{tab:result-sr-3a} shows the comparison of the OpenQuake result with the expected result.

% ---------------------------------------------------------------------------
\subsubsection{Case 3b}
\input{qareport/cases/sr-3b}
\input{qareport/results/sr-3b}
Table \ref{tab:result-sr-3b} shows the comparison of the OpenQuake result with the expected result.
% ---------------------------------------------------------------------------
\subsubsection{Case 3c}
\input{qareport/cases/sr-3c}
\input{qareport/results/sr-3c}
Table \ref{tab:result-sr-3c} shows the comparison of the OpenQuake result with the expected result.

% ---------------------------------------------------------------------------
\subsubsection{Case 3d}
\input{qareport/cases/sr-3d}
\input{qareport/results/sr-3d}
Table \ref{tab:result-sr-3d} shows the comparison of the OpenQuake result with the expected result.


% -----------------------------------------------------------------------------
\subsection{Multiple asset tests}
\label{subsec:acc-sr-multiple}
The multiple asset test cases are designed to test the loss aggregation functions of the scenario risk calculator, such as:

\begin{itemize}
\item portfolio loss computation for a given ground motion field
\item calculation of mean and standard deviation of portfolio scenario loss
\end{itemize}

\input{qareport/tables/tab-assets.tex}

The list of assets in the exposure model used for the multiple-asset test cases for the scenario risk calculator is given in Table \ref{tab:assets}.

Ten thousand ground motion fields are generated for the given rupture, taking into consideration both the inter-event and intra-event variability in the ground motion. The ground motion prediction equation used is Boore and Atkinson (2008), and the Jayaram and Baker (2009) model for spatial correlation of ground motion values is applied.

\input{qareport/tables/tab-scenario-gmfs-multiple}

Table \ref{tab:scenario-gmfs-multiple} lists three of the ten thousand ground motion fields generated.

% ---------------------------------------------------------------------------
\subsubsection{Case 5a}
The purpose of this case is to test the computation of the mean and standard deviation of the loss for multiple assets, given variability in both the ground motion values and in the vulnerability function. The computation of the mean and standard deviation of the portfolio loss is also tested in this case. The asset vulnerability correlation factor is zero for this test case.


\input{qareport/results/tab-result-scenario-risk-5a}

Table \ref{tab:result-scenario-risk-5a} shows the comparison of the OpenQuake result with the expected result.

% ---------------------------------------------------------------------------
\subsubsection{Case 5b}


\input{qareport/results/tab-result-scenario-risk-5b}

Table \ref{tab:result-scenario-risk-5b} shows the comparison of the OpenQuake result with the expected result.

% ---------------------------------------------------------------------------
\subsubsection{Case 5c}


\input{qareport/results/tab-result-scenario-risk-5c}

Table \ref{tab:result-scenario-risk-5c} shows the comparison of the OpenQuake result with the expected result.

% ---------------------------------------------------------------------------
\subsubsection{Case 5d}


\input{qareport/results/tab-result-scenario-risk-5d}

Table \ref{tab:result-scenario-risk-5d} shows the comparison of the OpenQuake result with the expected result.

% ---------------------------------------------------------------------------

% -----------------------------------------------------------------------------
\subsection{Insurance tests}
\label{subsec:acc-sr-insurance}
\subsubsection{Case 7a}


\input{qareport/results/tab-result-scenario-risk-7a}

Table \ref{tab:result-scenario-risk-7a} shows the comparison of the OpenQuake result with the expected result.

% ---------------------------------------------------------------------------

% -----------------------------------------------------------------------------
\subsection{Calculation with logic-trees}
\label{subsec:acc-sr-lt}
\subsubsection{Case 6a}
\input{qareport/cases/sr-6a}
\input{qareport/results/sr-6a}
Table \ref{tab:result-sr-6a} shows the comparison of the OpenQuake result with the expected result.

\section{Scenario Damage Calculator}
   \label{sec:acc-sd}
   The tests for the scenario damage calculator assume the correct computation of the ground motion fields at the locations of the assets in the exposure model. Thus, the risk quality assurance tests implicitly rely on the acceptance tests for the scenario hazard calculator.

The rupture model used for the tests comprises a magnitude $M6.7$ rupture on a vertical strike-slip fault.\\

\noindent Details of the rupture are given below:\\

\noindent
Fault type: Strike slip\\
Fault dip: $90^{\circ}$\\
Fault plane depths: 0--20 km\\
Fault coordinates:\\
South end: $38.0000^{\circ} N$, $122.0000^{\circ} W$\\
North end: $38.2248^{\circ} N$, $122.0000^{\circ} W$\\
Rupture magnitude: 6.7\\
Rupture hypocenter: $38.1124^{\circ} N$, $122.0000^{\circ} W$\\
Hypocenter depth: 10 km\\

The complete collection of input models and job configuration files used in these test cases can be accessed here:
\href{https://github.com/gem/oq-risklib/tree/master/openquake/qa_tests_data/scenario_damage}
{https://github.com/gem/oq-risklib/tree/master/openquake/qa\_tests\_data/scenario\_damage}

\subsection{Single asset tests}
\label{subsec:acc-sd-single}
The single asset test cases are designed to test the basic elements of the scenario damage calculator, such as:

\begin{itemize}
\item loss field computation for a given set of ground motion fields
\item calculation of mean and standard deviation of scenario loss
\end{itemize}

The location and taxonomy of the single asset in the exposure model used for the single-asset test cases for the scenario risk calculator are given in Table \ref{tab:asset}.

% ---------------------------------------------------------------------------
\subsubsection{Case 1a}
Test case 1a uses a set of 10 identical ground motion values.

The purpose of this case is to test the accurate computation of the probability of occurence of each damage state, given zero variability in the ground motion values. This case exercises the interpolation of discrete fragility functions. Hand calculations are used to verify the correctness of the results obtained.

% \input{qareport/tables/tab-scenario-gmfs-identical-10}

% Table \ref{tab:scenario-gmfs-identical-10} lists five of the ten ground motion values used in this test case.

\input{qareport/tables/tab-ff-disc-zndl}

Table \ref{tab:ff-disc-zndl} shows the mean loss ratios and corresponding coefficients of variation in the vulnerability function used in this test case.

Since all ground motion values are identical, calculation of the damage distribution is straightforward in this case. The ground motion value at the location of the single asset is $PGA = 0.5 g$. The discrete fragility function for this case provides damage state probabilities of exceedance at intensity measure levels $0.4 g$ and $0.6 g$, but none at $0.5 g$. The exceedance probabilities at $0.4 g$ and $0.6 g$ corresponding to the discrete damage states $[ds_1, ds_2, ds_3, ds_4]$ are $[0.152, 0.014, 0.008, 0.006]$ and $[0.846, 0.129, 0.085, 0.067]$ respectively.

The exceedance probabilities at $0.5 g$ are obtained by interpolating between these two sets of values. Linear interpolation gives exceedance probabilities of $[0.499, 0.0715, 0.08225, 0.0365]$ for $PGA = 0.5 g$. The probabilities of damage state occurrence are given by the pairwise differences of the exceedance probabilities as $[0.4990 - 0.0715, 0.0715 - 0.08225, 0.08225 - 0.0365, 0.0365]$

Since there is no variability in the ground motion, the mean loss ratio is also $0.15$, and the standard deviation of the loss ratio is $0.0$.

These numbers are multiplied by the asset value of $10,000$ to give the mean and standard deviation of loss for the scenario as $1,500$ and $0$ respectively.

\input{qareport/results/tab-result-scenario-damage-1a}

Table \ref{tab:result-scenario-damage-1a} shows the comparison of the OpenQuake result with the expected result.

% ---------------------------------------------------------------------------
\subsubsection{Case 1b}
Test case 1b uses the same set of identical ground motion values as case 1a, and described in Table \ref{tab:scenario-gmfs-identical-10}. However, in contrast to case 1a, variability in the loss ratio \emph{is} considered in the vulnerability function for this case.

\input{qareport/tables/tab-ff-cont-zmin}

Table \ref{tab:ff-cont-zmin} shows the mean loss ratios and corresponding coefficients of variation in the vulnerability function used in this test case.

Similar to case 1a described above, linear interpolation gives a mean loss ratio of $0.15$ for $PGA = 0.5 g$. The vulnerability function for this case provides coefficients of variation for the loss ratio at intensity measure levels $0.4 g$ and $0.6 g$, but none at $0.5 g$. The CoVs of the loss ratio at $0.4 g$ and $0.6 g$ are $0.24$ and $0.32$ respectively. The coefficient of variation of the loss ratio for $PGA = 0.5 g$ is thus obtained by linear interpolation as $0.28$.

The loss ratio at $PGA = 0.5 g$ follows a lognormal distribution with a mean of $0.15$ and a standard deviation of $0.28 \times 0.15 = 0.042$.

Since there is no variability in the ground motion, the mean loss ratio for the scenario is also $0.15$, and the standard deviation of the loss ratio is $0.042$.

These numbers are multiplied by the asset value of $10,000$ to give the mean and standard deviation of loss for the scenario as $1,500$ and $420$ respectively.

\input{qareport/results/tab-result-scenario-damage-1b}

Table \ref{tab:result-scenario-damage-1b} shows the comparison of the OpenQuake result with the expected result.

% ---------------------------------------------------------------------------
\subsubsection{Case 1c}
Variability in the ground motion is considered in all cases starting from case 1c. Ten thousand ground motion fields are generated for the given rupture, taking into consideration both the inter-event and intra-event variability in the ground motion. The ground motion prediction equation used is Boore and Atkinson (2008).

The purpose of this case is to test the computation of the mean and standard deviation of the loss, given variability in both the ground motion values and no variability in the vulnerability function.

\input{qareport/tables/tab-scenario-gmfs-single}

Table \ref{tab:scenario-gmfs-single} lists five of the ten thousand ground motion values generated by OpenQuake.

Since the mean loss ratios in the vulnerability function are not a linear function of the intensity measure levels, an analytical solution for the mean and standard deviation of loss for the scenario cannot be found. Thus, in order to check the OpenQuake results, an alternate implementation of the calculator algorithm in the programming language Julia is used.

The mean and standard deviation of the logarithm of the ground motion calculated at the location of the asset as obtained by using the Boore and Atkinson (2008) equation are $-0.648$ and $0.564$ respectively. Assuming a lognormal distribution for the variability in the ground motion, one million ground motion values are generated using Julia with these logarithmic mean and standard deviation values.

The variability in the loss ratios is zero for the vulnerability function used in this test case. Thus, the loss ratio for each simulated ground motion value is obtained through interpolation on the mean loss ratios provided by the vulnerability function. The mean and standard deviation of loss ratio for the scenario are estimated simply as the mean and standard deviation of the million simulated loss ratios.

\input{qareport/results/tab-result-scenario-damage-1c}

Table \ref{tab:result-scenario-damage-1c} shows the comparison of the OpenQuake result with the expected result.

% ---------------------------------------------------------------------------
\subsubsection{Case 1d}
The purpose of this case is to test the computation of the mean and standard deviation of the loss, given variability in both the ground motion values and in the vulnerability function.

Similar to case 1c, one million ground motion values are generated using Julia with the logarithmic mean and standard deviation of the ground motion value at the location of the asset as obtained by using the Boore and Atkinson (2008).

The mean loss ratio and standard deviation of loss ratio for each simulated ground motion value are obtained through interpolation on the mean loss ratios and corresponding coefficients of variation provided by the vulnerability function. Using the interpolated mean and standard deviation of loss ratios, one loss ratio is sampled for each ground motion value, assuming a lognormal distibution.

The mean and standard deviation of loss ratio for the scenario are estimated simply as the mean and standard deviation of the million simulated loss ratios.

\input{qareport/results/tab-result-scenario-damage-1d}

Table \ref{tab:result-scenario-damage-1d} shows the comparison of the OpenQuake result with the expected result.

% ---------------------------------------------------------------------------
\subsubsection{Case 2a}


\input{qareport/results/tab-result-scenario-damage-2a}

Table \ref{tab:result-scenario-damage-2a} shows the comparison of the OpenQuake result with the expected result.

% ---------------------------------------------------------------------------
\subsubsection{Case 3a}


\input{qareport/results/tab-result-scenario-damage-3a}

Table \ref{tab:result-scenario-damage-3a} shows the comparison of the OpenQuake result with the expected result.

% -----------------------------------------------------------------------------
\subsection{Multiple asset tests}
\label{subsec:acc-sd-multiple}
The multiple asset test cases are designed to test the loss aggregation functions of the scenario damage calculator, such as:

\begin{itemize}
\item damage distribution per taxonomy
\item damage distribution for the portfolio
\end{itemize}

% ---------------------------------------------------------------------------
\subsubsection{Case 2a}
\input{qareport/cases/sd-2a}



% -----------------------------------------------------------------------------
\subsection{Calculation with logic-trees}
\label{subsec:acc-sd-lt}
% ---------------------------------------------------------------------------
\subsubsection{Case 3a}
\input{qareport/results/sd-3a}

Table \ref{tab:result-sd-3a} shows the comparison of the OpenQuake results with the expected results.

\section{Classical Risk Calculator}
   \label{sec:acc-cr}
   The tests for the classical PSHA-based risk calculator assume the correct computation of the hazard curves at the locations of the assets in the exposure model. Thus, the risk tests implicitly rely on the acceptance tests for the classical PSHA-based hazard calculator.

The source model used for the tests comprises a single vertical strike-slip fault with a Gutenberg-Richter b-value equal to 0.9 and a slip rate of 2 mm/yr. The MFD is a Gutenberg-Richter distribution truncated between magnitudes 5.0 and 6.5, while the Ground Motion Prediction Equation (GMPE) used is \citet{boore2008}.\\

\noindent Details of the fault geometry are given below:\\

\noindent
Fault type: Strike slip\\
Fault dip: $90^{\circ}$\\
Fault plane depths: 0--12 km\\
Fault coordinates:\\
South end: $38.0000^{\circ} N$, $122.0000^{\circ} W$\\
North end: $38.2248^{\circ} N$, $122.0000^{\circ} W$\\

Figure~\ref{fig:fault-sites}, shows the fault described above and the site geometry for the test cases described in the following sections. The single asset tests use only site $1$ shown in the figure, whereas the multiple asset tests use all seven of the sites. The geometry of the fault and locations of the sites match those selected by \citet{thomas2010} in their effort to verify PSHA computer programs.

\begin{figure}[htbp]
\centering
\includegraphics[width=12cm]{qareport/figures/fig-fault-sites}
\caption{Fault and site geometry for the classical risk tests. From \citet{thomas2010}}
\label{fig:fault-sites}
\end{figure}

The complete collection of input models and job configuration files used in these test cases can be accessed here:
\href{https://github.com/gem/oq-risklib/tree/master/openquake/qa_tests_data/classical_risk}
{https://github.com/gem/oq-risklib/tree/master/openquake/qa\_tests\_data/classical\_risk}

\subsection{Single asset tests}
\label{subsec:acc-cr-single}
The single asset test cases are designed to test the basic elements of the classical-PSHA based risk calculator, such as:

\begin{itemize}
\item asset loss ratio exceedance curve computation
\item asset loss exceedance curve computation
\end{itemize}

The location and taxonomy of the single asset in the exposure model used for the single-asset test cases for the classical risk calculator are given in Table~\ref{tab:asset}.

% ---------------------------------------------------------------------------
\subsubsection{Case 1a}
\input{qareport/cases/cr-1a}
\input{qareport/results/cr-1a}
Table~\ref{tab:result-cr-1a} shows the comparison of the OpenQuake result for average annual loss with the expected result.

% % ---------------------------------------------------------------------------
\subsubsection{Case 1b}
\input{qareport/cases/cr-1b}
\input{qareport/results/cr-1b}
Table~\ref{tab:result-cr-1b} shows the comparison of the OpenQuake result for average annual loss with the expected result.

% % % ---------------------------------------------------------------------------
\subsubsection{Case 1c}
\input{qareport/cases/cr-1c}
\input{qareport/results/cr-1c}
Table~\ref{tab:result-cr-1c} shows the comparison of the OpenQuake result for average annual loss with the expected result.

% % % ---------------------------------------------------------------------------
\subsubsection{Case 1d}
\input{qareport/cases/cr-1d}
\input{qareport/results/cr-1d}
Table~\ref{tab:result-cr-1d} shows the comparison of the OpenQuake result for average annual loss with the expected result.

% % % ---------------------------------------------------------------------------
\subsubsection{Case 1e}
\input{qareport/cases/cr-1e}
\input{qareport/results/cr-1e}
Table~\ref{tab:result-cr-1e} shows the comparison of the OpenQuake result for average annual loss with the expected result.

% % % ---------------------------------------------------------------------------
\subsubsection{Case 2a}
\input{qareport/cases/cr-2a}
\input{qareport/results/cr-2a}
Table \ref{tab:result-cr-1b} shows the comparison of the OpenQuake result for average annual nonstructural loss with the expected result.

% % ---------------------------------------------------------------------------
\subsubsection{Case 2b}
\input{qareport/cases/cr-2b}
\input{qareport/results/cr-2b}
Table \ref{tab:result-cr-2b} shows the comparison of the OpenQuake result for average annual contents loss with the expected result.

% % ---------------------------------------------------------------------------
\subsubsection{Case 2c}
\input{qareport/cases/cr-2c}
\input{qareport/results/cr-2c}
Table \ref{tab:result-cr-2c} shows the comparison of the OpenQuake result for average annual downtime loss with the expected result.

% % ---------------------------------------------------------------------------
\subsubsection{Case 2d}
\input{qareport/cases/cr-2d}
\input{qareport/results/cr-2d}
Table \ref{tab:result-cr-2d} shows the comparison of the OpenQuake result for average annual fatalities with the expected result.

% % ---------------------------------------------------------------------------
\subsubsection{Case 3a}
\input{qareport/cases/cr-3a}
\input{qareport/results/cr-3a}
Table \ref{tab:result-cr-3a} shows the comparison of the OpenQuake result for expected loss over 75 years with the expected result.

% % ---------------------------------------------------------------------------
\subsubsection{Case 4a}
\input{qareport/cases/cr-4a}
\input{qareport/results/cr-4a}
Table \ref{tab:result-cr-4a} shows the comparison of the OpenQuake result for average annual loss with the expected result.

% % ---------------------------------------------------------------------------
\subsubsection{Case 4b}
\input{qareport/cases/cr-4b}
\input{qareport/results/cr-4b}
Table \ref{tab:result-cr-4b} shows the comparison of the OpenQuake result for average annual loss with the expected result.

% % ---------------------------------------------------------------------------
\subsubsection{Case 4c}
\input{qareport/cases/cr-4c}
\input{qareport/results/cr-4c}
Table \ref{tab:result-cr-4c} shows the comparison of the OpenQuake result for average annual loss with the expected result.

% % ---------------------------------------------------------------------------
\subsubsection{Case 4d}
\input{qareport/cases/cr-4d}
\input{qareport/results/cr-4d}
Table \ref{tab:result-cr-4d} shows the comparison of the OpenQuake result for average annual loss with the expected result.


% -----------------------------------------------------------------------------
\subsection{Multiple asset tests}
\label{subsec:acc-cr-multiple}
The multiple asset test cases are designed to test the correct working of the classical risk calculator for a portfolio of assets of different taxonomies. The vulnerability functions for the different taxonomies are based on different intensity measure types.

The list of assets in the exposure model used for the multiple-asset test cases for the classical risk calculator is given in Table \ref{tab:assets-tax3}.

\subsubsection{Case 5a}
\input{qareport/cases/cr-5a}
\input{qareport/results/cr-5a}
Table~\ref{tab:result-cr-5a} shows the comparison of the OpenQuake results for average annual losses for the seven assets with the expected results.

% % -----------------------------------------------------------------------------
% \subsection{Insurance tests}
% \label{subsec:acc-cr-insurance}
% \subsubsection{Case 8a}


\input{qareport/results/tab-result-classical-risk-8a}

Table \ref{tab:result-classical-risk-8a} shows the comparison of the OpenQuake result with the expected result.

% ---------------------------------------------------------------------------

% % -----------------------------------------------------------------------------
\subsection{Calculation with logic-trees}
\label{subsec:acc-cr-lt}
\subsubsection{Case 7a}
\input{qareport/cases/cr-7a}
\input{qareport/results/cr-7a}

Table \ref{tab:result-cr-7a} shows the comparison of the OpenQuake result for average annual losses for the two branches with the expected results.

% ---------------------------------------------------------------------------


% \section{Classical Damage Calculator}
%    \label{sec:acc-cd}
%    The tests for the classical PSHA-based damage calculator assume the correct computation of the hazard curves at the locations of the assets in the exposure model. Thus, the damage tests implicitly rely on the acceptance tests for the classical PSHA-based hazard calculator.

The source model used for the tests comprises a single vertical strike-slip fault with a Gutenberg-Richter b-value equal to 0.9 and a slip rate of 2 mm/yr. The MFD is a Gutenberg-Richter distribution truncated between magnitudes 5.0 and 6.5, while the Ground Motion Prediction Equation (GMPE) used is Sadigh et al. (1997), with sigma set to zero.\\

\noindent Details of the fault geometry are given below:\\

\noindent
Fault type: Strike slip\\
Fault dip: $90^{\circ}$\\
Fault plane depths: 0--12 km\\
Fault coordinates:\\
South end: $38.0000^{\circ} N$, $122.0000^{\circ} W$\\
North end: $38.2248^{\circ} N$, $122.0000^{\circ} W$\\

The complete collection of input models and job configuration files used in these test cases can be accessed here:
\href{https://github.com/gem/oq-risklib/tree/master/openquake/qa_tests_data/classical_damage}
{https://github.com/gem/oq-risklib/tree/master/openquake/qa\_tests\_data/classical\_damage}

\subsection{Single asset tests}
\label{subsec:cd-single}
The single asset test cases are designed to test the basic elements of the classical-PSHA based damage calculator, such as:

\begin{itemize}
\item damage state occurrence probabilities for assets
\end{itemize}

The location and taxonomy of the single asset in the exposure model used for the single-asset test cases for the classical damage calculator are given in Table~\ref{tab:asset}.


% ---------------------------------------------------------------------------
\subsubsection{Case 1a}
\input{qareport/cases/cd-1a}
\input{qareport/results/cd-1a}

Table \ref{tab:result-cd-1a} shows the comparison of the OpenQuake result with the expected result.




% -----------------------------------------------------------------------------
\subsection{Multiple asset tests}
\label{subsec:cd-multiple}
The multiple asset test cases are designed to test the correct working of the classical damage calculator for a portfolio of assets of different taxonomies using different fragility functions.

\subsubsection{Case 3a}
\input{qareport/cases/cd-3a}
\input{qareport/results/cd-3a}
Table \ref{tab:result-cd-3a} shows the comparison of the OpenQuake results with the expected results for assets $a1$, $a2$, and $a3$.

% -----------------------------------------------------------------------------
\subsection{Calculation with logic-trees}
\label{subsec:cd-lt}
\subsubsection{Case 7a}


\input{qareport/results/cd-7a}

Table \ref{tab:result-cd-7a} shows the comparison of the OpenQuake result with the expected result.
% \section{Event-Based Risk Calculator}
%    \label{sec:acc-ebr}
%    The OpenQuake event based risk calculator builds on the outputs of the event based hazard calculator, employing Monte Carlo sampling techniques to compute event loss tables, asset loss exceedance curves, average annual asset losses, portfolio loss exceedance curves, and average annual portfolio loss. The tests for the event-based risk calculator assume the correct computation of the ground motion fields at the locations of the assets in the exposure model. Thus, the risk tests implicitly rely on the acceptance tests for the event-based hazard calculator.

\subsection{Single asset tests}
\label{subsec:acc-ebr-single}
The single asset test cases are designed to test the basic elements of the event-based risk calculator, such as:

\begin{itemize}
\item asset event loss table computation
\item asset loss exceedance curve computation
\end{itemize}

The location and taxonomy of the single asset in the exposure model used for the single-asset test cases for the event-based risk calculator are given in Table \ref{tab:asset}.

% ---------------------------------------------------------------------------
\subsubsection{Case 1a}
\input{qareport/cases/ebr-1a}
\input{qareport/results/ebr-1a}
Table \ref{tab:result-ebr-1a} shows the comparison of the OpenQuake result for average annual loss with the expected result.

% ---------------------------------------------------------------------------
\subsubsection{Case 1b}
\input{qareport/cases/ebr-1b}
\input{qareport/results/ebr-1b}
Table \ref{tab:result-ebr-1b} shows the comparison of the OpenQuake result for average annual loss with the expected result.

% ---------------------------------------------------------------------------
\subsubsection{Case 1c}
\input{qareport/cases/ebr-1c}
\input{qareport/results/ebr-1c}
Table \ref{tab:result-ebr-1c} shows the comparison of the OpenQuake result for average annual loss with the expected result.
% ---------------------------------------------------------------------------
\subsubsection{Case 1d}
\input{qareport/cases/ebr-1d}
\input{qareport/results/ebr-1d}
Table \ref{tab:result-ebr-1d} shows the comparison of the OpenQuake result for average annual loss with the expected result.

% ---------------------------------------------------------------------------
\subsubsection{Case 1e}
\input{qareport/cases/ebr-1e}
\input{qareport/results/ebr-1e}
Table \ref{tab:result-ebr-1e} shows the comparison of the OpenQuake result for average annual loss with the expected result.

% ---------------------------------------------------------------------------
\subsubsection{Case 1f}
\input{qareport/cases/ebr-1f}
\input{qareport/results/ebr-1f}
Table \ref{tab:result-ebr-1f} shows the comparison of the OpenQuake result for average annual loss with the expected result.

% ---------------------------------------------------------------------------
\subsubsection{Case 1g}
\input{qareport/cases/ebr-1g}
\input{qareport/results/ebr-1g}
Table \ref{tab:result-ebr-1g} shows the comparison of the OpenQuake result for average annual loss with the expected result.

% ---------------------------------------------------------------------------
\subsubsection{Case 2a}
\input{qareport/cases/ebr-2a}
\input{qareport/results/ebr-2a}
Table \ref{tab:result-ebr-2a} shows the comparison of the OpenQuake result for average annual loss with the expected result.

% ---------------------------------------------------------------------------
\subsubsection{Case 2b}
\input{qareport/cases/ebr-2b}
\input{qareport/results/ebr-2b}
Table \ref{tab:result-ebr-2b} shows the comparison of the OpenQuake result for average annual loss with the expected result.

% ---------------------------------------------------------------------------
\subsubsection{Case 2c}
\input{qareport/cases/ebr-2c}
\input{qareport/results/ebr-2c}
Table \ref{tab:result-ebr-2c} shows the comparison of the OpenQuake result for average annual loss with the expected result.

% ---------------------------------------------------------------------------
\subsubsection{Case 2d}
\input{qareport/cases/ebr-2d}
\input{qareport/results/ebr-2d}
Table \ref{tab:result-ebr-2d} shows the comparison of the OpenQuake result for average annual loss with the expected result.

% ---------------------------------------------------------------------------
\subsubsection{Case 3a}
\input{qareport/cases/ebr-3a}
\input{qareport/results/ebr-3a}
Table \ref{tab:result-ebr-3a} shows the comparison of the OpenQuake result for average annual loss with the expected result.


% -----------------------------------------------------------------------------
\subsection{Multiple asset tests}
\label{subsec:acc-ebr-multiple}
The multiple asset test cases are designed to test the loss aggregation functions of the event-based risk calculator, such as:

\begin{itemize}
\item portfolio loss computation for a given ground motion field
\item calculation of portfolio loss exceedance curves
\end{itemize}

The major differences from the single asset calculations that need to be considered in the multiple asset tests include the possibility of having spatially-correlated ground motion fields and correlated vulnerability models for different assets of the same taxonomy.

% ---------------------------------------------------------------------------
\subsubsection{Case 6a}
\input{qareport/cases/ebr-6a}
\input{qareport/results/ebr-6a}
Table \ref{tab:result-ebr-6a} shows the comparison of the OpenQuake result for average asset losses and average portfolio loss with the expected result.

% ---------------------------------------------------------------------------
\subsubsection{Case 6b}
\input{qareport/cases/ebr-6b}
\input{qareport/results/ebr-6b}
Table \ref{tab:result-ebr-6b} shows the comparison of the OpenQuake result for average portfolio loss with the expected result.

% ---------------------------------------------------------------------------
\subsubsection{Case 6c}
\input{qareport/cases/ebr-6c}
\input{qareport/results/ebr-6c}
Table \ref{tab:result-ebr-6c} shows the comparison of the OpenQuake result for average portfolio loss with the expected result.

% ---------------------------------------------------------------------------

% -----------------------------------------------------------------------------
% \subsection{Calculation with logic-trees}
% \label{subsec:acc-ebr-lt}
% \input{qareport/acceptance/acc-ebr-lt}

% -----------------------------------------------------------------------------
% \subsection{Insurance tests}
% \label{subsec:acc-ebr-insurance}
% The insurance test cases are designed to test the insurance related elements of the event-based risk calculator, such as:

\begin{itemize}
\item asset insured loss table computation
\item asset insured loss exceedance curve computation
\item portfolio insured loss table computation
\item portfolio insured loss exceedance curve computation
\end{itemize}

% ---------------------------------------------------------------------------
\subsubsection{Case 7a}
\input{qareport/cases/ebr-7a}
\input{qareport/results/ebr-7a}
Table \ref{tab:result-ebr-7a} shows the comparison of the OpenQuake result for average annual loss with the expected result.

% ---------------------------------------------------------------------------
\subsubsection{Case 7b}
\input{qareport/cases/ebr-7b}
\input{qareport/results/ebr-7b}
Table \ref{tab:result-ebr-7b} shows the comparison of the OpenQuake result for average annual loss with the expected result.

% ---------------------------------------------------------------------------
\subsubsection{Case 7c}
\input{qareport/cases/ebr-7c}
\input{qareport/results/ebr-7c}
Table \ref{tab:result-ebr-7c} shows the comparison of the OpenQuake result for average annual loss with the expected result.

%    \cleardoublepage
   \cleardoublepage

% % -------------------------------------------------- Part IV: Benchmark Tests -
% \thispagestyle{empty}
% \part{Benchmark Tests}

% \chapterimage{figures/chapter_head_2.pdf} % Chapter heading image
% \chapter{Comparing the Classical and Event-Based Risk Calculators}
%    \label{chap:bm-cr-ebr}
%    \input{qareport/benchmarking/bm-cr-ebr.tex}
%    \cleardoublepage

% \chapterimage{figures/chapter_head_2.pdf} % Chapter heading image
% \chapter{Comparison with Other Softwares}
%    \label{chap:bm-comparison}
%    \input{qareport/benchmarking/bm-comparison.tex}
%    \cleardoublepage

% % -------------------------------------------------- Part V: Performance Tests -
% \thispagestyle{empty}
% \part{Performance Tests}

% \chapterimage{figures/chapter_head_2.pdf} % Chapter heading image
% \chapter{Demos}
%    \label{chap:perf-demos}
%    \section{Scenario Risk Calculator}
   \label{sec:demos-sr}
   \input{qareport/demos/demos-sr.tex}
\section{Classical Risk Calculator}
   \label{sec:demos-cr}
   \input{qareport/demos/demos-cr.tex}
\section{Event-Based Risk Calculator}
   \label{sec:demos-ebr}
   \input{qareport/demos/demos-ebr.tex}
   \cleardoublepage
%    \cleardoublepage
% \chapter{Stress Tests}
%    \label{chap:perf-stress}
%    \section{Scenario Risk Calculator}
   \label{sec:stress-sr}
   \input{qareport/stress-tests/stress-sr.tex}
\section{Classical Risk Calculator}
   \label{sec:stress-cr}
   \input{qareport/stress-tests/stress-cr.tex}
\section{Event-Based Risk Calculator}
   \label{sec:stress-ebr}
   \input{qareport/stress-tests/stress-ebr.tex}
   \cleardoublepage
%    \cleardoublepage

% -------------------------------------------------------- Part VI: Appendices -
% \thispagestyle{empty}
% \part*{Appendices}
% \appendix

% \chapterimage{figures/chapter_head_2.pdf} % Chapter heading image
% \chapter{Chapter Name}
%    \input{qareport/template.tex}
%    \cleardoublepage

%-------------------------------------------------------------------------------
%  BIBLIOGRAPHY
%-------------------------------------------------------------------------------

\chapter*{Bibliography}
\addcontentsline{toc}{chapter}{\textcolor{darkgray}{Bibliography}}
\printbibliography[heading=bibempty]

%-------------------------------------------------------------------------------
%  INDEX & GLOSSARY
%-------------------------------------------------------------------------------

\cleardoublepage
\phantomsection
\setlength{\columnsep}{0.75cm}
\addcontentsline{toc}{chapter}{\textcolor{darkgray}{Index}}
\printindex
\printglossary
\hfill \\ \thispagestyle{empty} \clearpage % ---------------- Final empty page -

\end{document}
