The tests for the classical PSHA-based damage calculator assume the correct computation of the hazard curves at the locations of the assets in the exposure model. Thus, the damage tests implicitly rely on the acceptance tests for the classical PSHA-based hazard calculator.

The source model used for the tests comprises a single vertical strike-slip fault with a Gutenberg-Richter b-value equal to 0.9 and a slip rate of 2 mm/yr. The MFD is a Gutenberg-Richter distribution truncated between magnitudes 5.0 and 6.5, while the Ground Motion Prediction Equation (GMPE) used is Sadigh et al. (1997), with sigma set to zero.\\

\noindent Details of the fault geometry are given below:\\

\noindent
Fault type: Strike slip\\
Fault dip: $90^{\circ}$\\
Fault plane depths: 0--12 km\\
Fault coordinates:\\
South end: $38.0000^{\circ} N$, $122.0000^{\circ} W$\\
North end: $38.2248^{\circ} N$, $122.0000^{\circ} W$\\

The complete collection of input models and job configuration files used in these test cases can be accessed here:
\href{https://github.com/gem/oq-risklib/tree/master/openquake/qa_tests_data/classical_damage}
{https://github.com/gem/oq-risklib/tree/master/openquake/qa\_tests\_data/classical\_damage}

\subsection{Single asset tests}
\label{subsec:classical-damage-single}
The single asset test cases are designed to test the basic elements of the classical-PSHA based damage calculator, such as:

\begin{itemize}
\item damage state occurrence probabilities for assets
\end{itemize}

The location and taxonomy of the single asset in the exposure model used for the single-asset test cases for the classical damage calculator are given in Table \ref{tab:assets}.


% ---------------------------------------------------------------------------
\subsubsection{Case 1a}
Test case 1a 

\begin{table}[htbp]

\centering
\begin{tabular}{ l l l r r r }

\hline
\rowcolor{anti-flashwhite}
\bf{Asset} & \bf{Damage State} & \bf{Result} & \bf{Expected} & \bf{OpenQuake} & \bf{Difference}\\
\hline
a1 & No damage & Probability of occurrence &  &  & \% \\
   & Slight    & Probability of occurrence &  &  & \% \\
   & Moderate  & Probability of occurrence &  &  & \% \\
   & Extreme   & Probability of occurrence &  &  & \% \\
   & Complete  & Probability of occurrence &  &  & \% \\
\hline
\end{tabular}

\caption{Results for classical damage test case 1a}
\label{tab:result-classical-damage-1a}
\end{table}

Table \ref{tab:result-classical-damage-1a} shows the comparison of the OpenQuake result with the expected result.

% ---------------------------------------------------------------------------
\subsubsection{Case 1b}
Test case 1b 

\begin{table}[htbp]

\centering
\begin{tabular}{ l l l r r r }

\hline
\rowcolor{anti-flashwhite}
\bf{Asset} & \bf{Damage State} & \bf{Result} & \bf{Expected} & \bf{OpenQuake} & \bf{Difference}\\
\hline
a1 & No damage & Probability of occurrence & $9.83 \times 10^{1}$ & $9.83 \times 10^{1}$ & 0\% \\
   & Slight    & Probability of occurrence & $1.04 \times 10^{3}$ & $1.04 \times 10^{3}$ & 0\% \\
   & Moderate  & Probability of occurrence & $2.89 \times 10^{3}$ & $2.89 \times 10^{3}$ & 0\% \\
   & Extreme   & Probability of occurrence & $3.29 \times 10^{3}$ & $3.29 \times 10^{3}$ & 0\% \\
   & Complete  & Probability of occurrence & $1.01 \times 10^{2}$ & $1.01 \times 10^{2}$ & 0\% \\
\hline
\end{tabular}

\caption{Results for classical damage test case 1b}
\label{tab:result-classical-damage-1b}
\end{table}

Table \ref{tab:result-classical-damage-1b} shows the comparison of the OpenQuake result with the expected result.

% ---------------------------------------------------------------------------
\subsubsection{Case 1c}


\begin{table}[htbp]

\centering
\begin{tabular}{ l l l r r r }

\hline
\rowcolor{anti-flashwhite}
\bf{Asset} & \bf{Damage State} & \bf{Result} & \bf{Expected} & \bf{OpenQuake} & \bf{Difference}\\
\hline
a1 & No damage & Probability of occurrence &  &  & \% \\
   & Slight    & Probability of occurrence &  &  & \% \\
   & Moderate  & Probability of occurrence &  &  & \% \\
   & Extreme   & Probability of occurrence &  &  & \% \\
   & Complete  & Probability of occurrence &  &  & \% \\
\hline
\end{tabular}

\caption{Results for classical damage test case 1c}
\label{tab:result-classical-damage-1c}
\end{table}

Table \ref{tab:result-classical-damage-1c} shows the comparison of the OpenQuake result with the expected result.

% ---------------------------------------------------------------------------
\subsubsection{Case 2a}


\begin{table}[htbp]

\centering
\begin{tabular}{ l l l r r r }

\hline
\rowcolor{anti-flashwhite}
\bf{Asset} & \bf{Damage State} & \bf{Result} & \bf{Expected} & \bf{OpenQuake} & \bf{Difference}\\
\hline
a1 & No damage & Probability of occurrence & $9.71 \times 10^{-1}$ & $9.71 \times 10^{-1}$ & 0\% \\
   & Slight    & Probability of occurrence & $4.53 \times 10^{-3}$ & $4.53 \times 10^{-3}$ & 0\% \\
   & Moderate  & Probability of occurrence & $8.48 \times 10^{-3}$ & $8.48 \times 10^{-3}$ & 0\% \\
   & Extreme   & Probability of occurrence & $5.29 \times 10^{-3}$ & $5.29 \times 10^{-3}$ & 0\% \\
   & Complete  & Probability of occurrence & $1.10 \times 10^{-2}$ & $1.10 \times 10^{-2}$ & 0\% \\
\hline
\end{tabular}

\caption{Results for classical damage test case 2a}
\label{tab:result-classical-damage-2a}
\end{table}

Table \ref{tab:result-classical-damage-2a} shows the comparison of the OpenQuake result with the expected result.

% ---------------------------------------------------------------------------
\subsubsection{Case 2b}


\begin{table}[htbp]

\centering
\begin{tabular}{ l l l r r r }

\hline
\rowcolor{anti-flashwhite}
\bf{Asset} & \bf{Damage State} & \bf{Result} & \bf{Expected} & \bf{OpenQuake} & \bf{Difference}\\
\hline
a1 & No damage & Probability of occurrence & $9.71 \times 10^{1}$ & $9.71 \times 10^{1}$ & 0\% \\
   & Slight    & Probability of occurrence & $4.52 \times 10^{3}$ & $4.52 \times 10^{3}$ & 0\% \\
   & Moderate  & Probability of occurrence & $8.48 \times 10^{3}$ & $8.48 \times 10^{3}$ & 0\% \\
   & Extreme   & Probability of occurrence & $5.29 \times 10^{3}$ & $5.29 \times 10^{3}$ & 0\% \\
   & Complete  & Probability of occurrence & $1.10 \times 10^{2}$ & $1.10 \times 10^{2}$ & 0\% \\
\hline
\end{tabular}

\caption{Results for classical damage test case 2b}
\label{tab:result-classical-damage-2b}
\end{table}

Table \ref{tab:result-classical-damage-2b} shows the comparison of the OpenQuake result with the expected result.

% ---------------------------------------------------------------------------
\subsubsection{Case 3a}


\begin{table}[htbp]

\centering
\begin{tabular}{ l l l r r r }

\hline
\rowcolor{anti-flashwhite}
\bf{Asset} & \bf{Damage State} & \bf{Result} & \bf{Expected} & \bf{OpenQuake} & \bf{Difference}\\
\hline
a1 & No damage & Probability of occurrence & $9.72 \times 10^{-1}$ & $9.72 \times 10^{-1}$ & 0\% \\
   & Slight    & Probability of occurrence & $8.02 \times 10^{-3}$ & $8.02 \times 10^{-3}$ & 0\% \\
   & Moderate  & Probability of occurrence & $1.06 \times 10^{-2}$ & $1.06 \times 10^{-2}$ & 0\% \\
   & Extreme   & Probability of occurrence & $5.73 \times 10^{-3}$ & $5.73 \times 10^{-3}$ & 0\% \\
   & Complete  & Probability of occurrence & $3.15 \times 10^{-3}$ & $3.15 \times 10^{-3}$ & 0\% \\
\hline
\end{tabular}

\caption{Results for classical damage test case 3a}
\label{tab:result-classical-damage-3a}
\end{table}

Table \ref{tab:result-classical-damage-3a} shows the comparison of the OpenQuake result with the expected result.

% ---------------------------------------------------------------------------
\subsubsection{Case 4a}


\begin{table}[htbp]

\centering
\begin{tabular}{ l l l r r r }

\hline
\rowcolor{anti-flashwhite}
\bf{Asset} & \bf{Damage State} & \bf{Result} & \bf{Expected} & \bf{OpenQuake} & \bf{Difference}\\
\hline
a1 & No damage & Probability of occurrence & $9.71 \times 10^{-1}$ & $9.71 \times 10^{-1}$ & 0\% \\
   & Slight    & Probability of occurrence & $4.43 \times 10^{-3}$ & $4.43 \times 10^{-3}$ & 0\% \\
   & Moderate  & Probability of occurrence & $8.41 \times 10^{-3}$ & $8.41 \times 10^{-3}$ & 0\% \\
   & Extreme   & Probability of occurrence & $5.29 \times 10^{-3}$ & $5.29 \times 10^{-3}$ & 0\% \\
   & Complete  & Probability of occurrence & $1.10 \times 10^{-2}$ & $1.10 \times 10^{-2}$ & 0\% \\
\hline
\end{tabular}

\caption{Results for classical damage test case 4a}
\label{tab:result-classical-damage-4a}
\end{table}

Table \ref{tab:result-classical-damage-4a} shows the comparison of the OpenQuake result with the expected result.

% ---------------------------------------------------------------------------
\subsubsection{Case 4b}


\begin{table}[htbp]

\centering
\begin{tabular}{ l l l r r r }

\hline
\rowcolor{anti-flashwhite}
\bf{Asset} & \bf{Damage State} & \bf{Result} & \bf{Expected} & \bf{OpenQuake} & \bf{Difference}\\
\hline
a1 & No damage & Probability of occurrence & $1.09 \times 10^{1}$ & $1.09 \times 10^{1}$ & \% \\
   & Slight    & Probability of occurrence & $4.43 \times 10^{2}$ & $4.43 \times 10^{2}$ & \% \\
   & Moderate  & Probability of occurrence & $1.38 \times 10^{1}$ & $1.38 \times 10^{1}$ & \% \\
   & Extreme   & Probability of occurrence & $1.44 \times 10^{1}$ & $1.44 \times 10^{1}$ & \% \\
   & Complete  & Probability of occurrence & $5.65 \times 10^{1}$ & $5.65 \times 10^{1}$ & \% \\
\hline
\end{tabular}

\caption{Results for classical damage test case 4b}
\label{tab:result-classical-damage-4b}
\end{table}

Table \ref{tab:result-classical-damage-4b} shows the comparison of the OpenQuake result with the expected result.

% ---------------------------------------------------------------------------
\subsubsection{Case 4c}


\begin{table}[htbp]

\centering
\begin{tabular}{ l l l r r r }

\hline
\rowcolor{anti-flashwhite}
\bf{Asset} & \bf{Damage State} & \bf{Result} & \bf{Expected} & \bf{OpenQuake} & \bf{Difference}\\
\hline
a1 & No damage & Probability of occurrence & $1.09 \times 10^{1}$ & $1.09 \times 10^{1}$ & \% \\
   & Slight    & Probability of occurrence & $4.43 \times 10^{2}$ & $4.43 \times 10^{2}$ & \% \\
   & Moderate  & Probability of occurrence & $1.38 \times 10^{1}$ & $1.38 \times 10^{1}$ & \% \\
   & Extreme   & Probability of occurrence & $1.44 \times 10^{1}$ & $1.44 \times 10^{1}$ & \% \\
   & Complete  & Probability of occurrence & $5.65 \times 10^{1}$ & $5.65 \times 10^{1}$ & \% \\
\hline
\end{tabular}

\caption{Results for classical damage test case 4c}
\label{tab:result-classical-damage-4c}
\end{table}

Table \ref{tab:result-classical-damage-4c} shows the comparison of the OpenQuake result with the expected result.

% ---------------------------------------------------------------------------
\subsubsection{Case 5a}


\begin{table}[htbp]

\centering
\begin{tabular}{ l l l r r r }

\hline
\rowcolor{anti-flashwhite}
\bf{Asset} & \bf{Damage State} & \bf{Result} & \bf{Expected} & \bf{OpenQuake} & \bf{Difference}\\
\hline
a1 & No damage & Expected number of buildings &  &  & \% \\
   & Slight    & Expected number of buildings &  &  & \% \\
   & Moderate  & Expected number of buildings &  &  & \% \\
   & Extreme   & Expected number of buildings &  &  & \% \\
   & Complete  & Expected number of buildings &  &  & \% \\
\hline
\end{tabular}

\caption{Results for classical damage test case 5a}
\label{tab:result-classical-damage-5a}
\end{table}

Table \ref{tab:result-classical-damage-5a} shows the comparison of the OpenQuake result with the expected result.


% -----------------------------------------------------------------------------
\subsection{Multiple asset tests}
\label{subsec:classical-damage-multiple}
The multiple asset test cases are designed to test the correct working of the classical risk calculator for a portfolio of assets of different taxonomies. The vulnerability functions for the different taxonomies are based on different intensity measure types.

The list of assets in the exposure model used for the multiple-asset test cases for the classical risk calculator is given in Table \ref{tab:assets}.

\subsubsection{Case 6a}


\begin{table}[htbp]

\centering
\begin{tabular}{ l l l r r r }

\hline
\rowcolor{anti-flashwhite}
\bf{Asset} & \bf{Damage State} & \bf{Result} & \bf{Expected} & \bf{OpenQuake} & \bf{Difference}\\
\hline
a1 & No damage & Probability of occurrence &  &  & \% \\
   & Slight    & Probability of occurrence &  &  & \% \\
   & Moderate  & Probability of occurrence &  &  & \% \\
   & Extreme   & Probability of occurrence &  &  & \% \\
   & Complete  & Probability of occurrence &  &  & \% \\
\hline
a2 & No damage & Probability of occurrence &  &  & \% \\
   & Slight    & Probability of occurrence &  &  & \% \\
   & Moderate  & Probability of occurrence &  &  & \% \\
   & Extreme   & Probability of occurrence &  &  & \% \\
   & Complete  & Probability of occurrence &  &  & \% \\
\hline
a3 & No damage & Probability of occurrence &  &  & \% \\
   & Slight    & Probability of occurrence &  &  & \% \\
   & Moderate  & Probability of occurrence &  &  & \% \\
   & Extreme   & Probability of occurrence &  &  & \% \\
   & Complete  & Probability of occurrence &  &  & \% \\
\hline
a4 & No damage & Probability of occurrence &  &  & \% \\
   & Slight    & Probability of occurrence &  &  & \% \\
   & Moderate  & Probability of occurrence &  &  & \% \\
   & Extreme   & Probability of occurrence &  &  & \% \\
   & Complete  & Probability of occurrence &  &  & \% \\
\hline
a5 & No damage & Probability of occurrence &  &  & \% \\
   & Slight    & Probability of occurrence &  &  & \% \\
   & Moderate  & Probability of occurrence &  &  & \% \\
   & Extreme   & Probability of occurrence &  &  & \% \\
   & Complete  & Probability of occurrence &  &  & \% \\
\hline
a6 & No damage & Probability of occurrence &  &  & \% \\
   & Slight    & Probability of occurrence &  &  & \% \\
   & Moderate  & Probability of occurrence &  &  & \% \\
   & Extreme   & Probability of occurrence &  &  & \% \\
   & Complete  & Probability of occurrence &  &  & \% \\
\hline
a7 & No damage & Probability of occurrence &  &  & \% \\
   & Slight    & Probability of occurrence &  &  & \% \\
   & Moderate  & Probability of occurrence &  &  & \% \\
   & Extreme   & Probability of occurrence &  &  & \% \\
   & Complete  & Probability of occurrence &  &  & \% \\
\hline
\end{tabular}

\caption{Results for classical damage test case 7a}
\label{tab:result-classical-damage-7a}
\end{table}

Table \ref{tab:result-classical-damage-6a} shows the comparison of the OpenQuake result with the expected result.

% -----------------------------------------------------------------------------
\subsection{Calculation with logic-trees}
\label{subsec:classical-damage-logictrees}
\subsubsection{Case 7a}


\begin{table}[htbp]

\centering
\begin{tabular}{ l l l l r r r }

\hline
\rowcolor{anti-flashwhite}
\bf{Branch} & \bf{Asset} & \bf{Damage State} & \bf{Result} & \bf{Expected} & \bf{OpenQuake} & \bf{Difference}\\
\hline
b1 & a1 & No damage & Probability of occurrence &  &  & \% \\
        & Slight    & Probability of occurrence &  &  & \% \\
        & Moderate  & Probability of occurrence &  &  & \% \\
        & Extreme   & Probability of occurrence &  &  & \% \\
        & Complete  & Probability of occurrence &  &  & \% \\
\hline
b2 & a1 & No damage & Probability of occurrence &  &  & \% \\
        & Slight    & Probability of occurrence &  &  & \% \\
        & Moderate  & Probability of occurrence &  &  & \% \\
        & Extreme   & Probability of occurrence &  &  & \% \\
        & Complete  & Probability of occurrence &  &  & \% \\
\hline
\end{tabular}

\caption{Results for classical damage test case 7a}
\label{tab:result-classical-damage-7a}
\end{table}

Table \ref{tab:result-classical-damage-7a} shows the comparison of the OpenQuake result with the expected result.
