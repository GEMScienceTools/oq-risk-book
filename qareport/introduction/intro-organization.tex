This document is organized into four chapters. The current chapter provides a very brief and general introduction to software testing with a focus on the testing of scientific software. The second chapter describes the module, or unit testing procedures adopted in the development of the \gls{acr:oqe} and we discuss some examples. The continuous integration mechanism used for development is also discussed.

The third chapter describes the general framework for the acceptance tests for the OpenQuake risk calculators. A brief overview of the theoretical background for the different calculators is also provided in this chapter. The fourth chapter describes the different test cases, input models, and results for the acceptance tests implemented for the OpenQuake scenario risk, classical risk, and event-based risk calculators.

% In the fifth chapter, we compare the loss curves computed using the event-based calculator with the corresponding loss curves computed using the classical-PSHA based calculator.

% In the sixth chapter, we illustrate tests comparing the results computed with the \gls{acr:oqe} against the ones computed using different probabilistic seismic risk analysis software.

% Chapter seven describes the OpenQuake risk demos and the average

% The final chapter describes the set of