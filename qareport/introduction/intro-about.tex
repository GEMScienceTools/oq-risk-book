The current document describes the extensive testing procedures adopted in the development of the risk component of the \gls{acr:oqe}, the open source hazard and risk software developed by the Global Earthquake Model initiative. A similar document describing the testing procedures for the hazard component of the engine is also available \citep{pagani2014_oqtesting}.

The OpenQuake risk library comprises a set of calculators capable of computing human or economic losses for a collection of assets, caused by a given scenario event, or by considering the probability of all possible events that might happen within a region within a certain time span.

The OpenQuake library comprises a set of calculators for assessing the potential economic losses and human casualties for a given ``scenario'' event or over a certain period of time. The outputs from the risk calculators may be used by disaster preparedness teams for gaining understanding about the potential consequences of an earthquake on a city, by structural engineers for evaluating the potential benefits of earthquake-resistant design or retrofitting and in the design of building codes, or by insurance and reinsurance firms for assessing the earthquake risk for a portfolio of assets.

These use cases clearly illustrate the need for rigorous testing and quality assurance of the software in question, and it is imperative that the implementation of the risk calculators be based on well-recognized, state-of-the-art and tested techniques; requirements that must be reconciled with the need to regularly incorporate recent advances given the progress carried out within the scientific community.

According to \citet{berkes2012} scientific software must be:
\begin{itemize}
\item Error proof
\item Flexible and able to accommodate different methods
\item Reproducible and re-usable
\end{itemize}

The features described below contribute to fulfill these requirements: %
\begin{itemize}
\item Software should have a modular and flexible structure capable of incorporating new features and - as a consequence - offer users the most recent and advanced techniques.
%
In very general terms, modularity is the level to which a component of a system can be moved, replaced or reused.
%
In software design, modularity means the separation of the software
into smaller independent components that can be implemented, maintained and tested easily and efficiently.
%
\item Software should have and extensive test coverage which captures possible errors and avoids regressions (i.e. unexpected behaviors introduced by new features).
%
Software testing \citep{myers2012} is an important, complex and vast discipline which helps in developing methods and processes aimed at certifying the extent to which a computer code behaves according to the original design intent and user specifications.
\end{itemize}

The \gls{acr:oqe} includes different levels of modularity. The first is the one separating the engine itself into a number of libraries, each one containing well identified knowledge, objects and methods (e.g. the OQ-hazardlib includes objects and methods needed to compute probabilistic seismic hazard and the OQ-risklib contains methods to compute scenario and probabilistic seismic risk). The second one pertains to the data model adopted in the development of each library as a result of the abstraction process.