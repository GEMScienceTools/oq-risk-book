The classical PSHA-based risk calculator convolves through numerical integration, the probabilistic vulnerability functions for an asset with the seismic hazard curve at the location of the asset, to give the loss distribution for the asset within a specified time period. The calculator requires the definition of an exposure model, a vulnerability model for each loss type of interest with vulnerability functions for each taxonomy represented in the exposure model, and hazard curves calculated in the region of interest. Loss curves and loss maps can currently be calculated for five different loss types using this calculator: structural losses, nonstructural losses, contents losses, downtime losses, and occupant fatalities. The main results of this calculator are loss exceedance curves for each asset, which describe the probability of exceedance of different loss levels over the specified time period, and loss maps for the region, which describe the loss values that have a given probability of exceedance over the specified time period.

The hazard curves required for this calculator can be calculated by the OpenQuake engine for all asset locations in the exposure model using the classical PSHA approach (Cornell, 1968; McGuire, 1976). The use of logic-trees allows for the consideration of uncertainty in the choice of a GMPE model for the different tectonic region types in the region and in the choice of vulnerability functions for the different taxonomy types in the exposure model.
