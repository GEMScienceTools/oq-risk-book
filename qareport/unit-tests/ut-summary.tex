In this chapter we have outlined both the process and the key benefits of developing comprehensive unit-tests for OpenQuake-engine, as well as outlining the operation of the continuous integration system, which should ensure that code with the potential to break the tests cannot be packaged and released. The unit-tests themselves have not been discussed in detail as nearly one thousand tests are executed during the unit-test process. However, to view the comprehensive set of tests, the reader is encouraged to refer to the full test-suite, which is open and available on the OpenQuake code repository (\href{https://github.com/gem/oq-risklib/tree/master/openquake/risklib/tests}
{https://github.com/gem/oq-risklib/tree/master/openquake/risklib/tests}).

Furthermore, we have also discussed how the OpenQuake-engine development tries to facilitate correct implementation of features such as ground motion prediction equations. For relatively simple conditions, a selection of PEER tests \citep{thomas2010} are built into the testing process, making OpenQuake-engine unique amongst other hazard software in integrating the verification into the development process.

The following chapters will expand in greater detail upon the additional loss curve benchmark tests, which both follow and expand upon the PEER testing process.
