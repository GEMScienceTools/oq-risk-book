The purpose of this case is to test vulnerability functions that are specified as discrete probability mass functions rather than the parametric lognormal or beta distributions seen in the previous cases.

\begin{table}[htbp]

\centering
\begin{tabular}{ l c c c c c c c c c c c}

\hline
\rowcolor{anti-flashwhite}
\bf{LR | PGA} & \bf{0.05g} & \bf{0.20g} & \bf{0.40g} & \bf{0.60g} & \bf{1.00g} & \bf{1.40g} & \bf{1.60g} & \bf{2.00g} \\
\hline
\bf{0.000} & 0.995 & 0.950 & 0.490 & 0.300 & 0.140 & 0.030 & 0.010 & 0.004 \\
\bf{0.005} & 0.004 & 0.030 & 0.380 & 0.400 & 0.300 & 0.100 & 0.030 & 0.006 \\
\bf{0.050} & 0.001 & 0.015 & 0.080 & 0.160 & 0.240 & 0.300 & 0.100 & 0.010 \\
\bf{0.200} & 0.000 & 0.004 & 0.020 & 0.080 & 0.160 & 0.260 & 0.300 & 0.030 \\
\bf{0.450} & 0.000 & 0.001 & 0.015 & 0.030 & 0.100 & 0.180 & 0.300 & 0.180 \\
\bf{0.800} & 0.000 & 0.000 & 0.010 & 0.020 & 0.040 & 0.100 & 0.180 & 0.390 \\
\bf{1.000} & 0.000 & 0.000 & 0.005 & 0.010 & 0.020 & 0.030 & 0.080 & 0.380 \\
\hline
\end{tabular}

\caption{Vulnerability function specified using a discrete probability distribution. The values in each column specify the probability of occurrence of the corresponding loss ratio from the first column, for the ground motion intensity listed in the first row.}
\label{tab:vf-pm-tax1}
\end{table}



The vulnerability function used in this test case is shown in Table~\ref{tab:vf-pm-tax1}. This vulnerability function specifies a set of loss and the corresponding probabilities of occurrence for these loss ratios at different intensity measure levels.

The same one thousand identical ground motion values described earlier in Case~1c and shown in Table~\ref{tab:gmfs-iden-l1-1000} are used in this test case.

The vulnerability function for this case provides probabilities of occurrence for a set of loss ratios ${0.000, 0.005, 0.050, 0.200, 0.450, 0.800, 1.000}$ at intensity measure levels $PGA = 0.4 g$ and $0.6 g$, but not at $0.5 g$. The specified set of probabilities for $PGA = 0.4 g$ are ${0.950, 0.030, 0.015, 0.004, 0.001, 0.000, 0.000}$, and those at $PGA = 0.6 g$ are ${0.490, 0.380, 0.080, 0.020, 0.015, 0.010, 0.005}$. Linear interpolation is used to obtain the probabilities of occurrence for the same set of loss ratios at $PGA = 0.5 g$ as ${0.72, 0.205, 0.0475, 0.012, 0.008, 0.005, 0.0025}$.

For the discrete random variable LR, which has the probability mass function $lr_1 \mapsto p_1, \dotsc, lr_n \mapsto p_n$, the mean and standard deviation are calculated as:

\begin{equation}
\mu_{LR} = \sum_{i=1}^{n} p_i \dot lr_i
\end{equation}

\begin{equation}
\sigma_{LR} = \sqrt{\sum_{i=1}^{n} p_i \dot lr_i^2 - \mu_{LR}^2}
\end{equation}
