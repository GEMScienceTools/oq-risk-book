This case is designed to test the computation of the mean and standard deviation of individual asset losses and also for the portfolio, when the vulnerability models of different assets of the same taxonomy are treated as partially correlated, with a correlation coefficient of $0.5$. In OpenQuake, this can be specified in the job configuration file, by setting the value of the parameter `asset\_correlation' to $0.5$.

As in the previous cases for zero and full correlation, ten thousand ground motion fields are generated for the given rupture, and these ground motion fields are also used for the corresponding calculation in Julia. The list of assets in the exposure model used for this case is given in Table~\ref{tab:assets-tax1}. Table~\ref{tab:vf-ln-tax1-nzcov} shows the mean loss ratios and corresponding coefficients of variation in the vulnerability function used in this test case.

Since the sampled loss ratios conditional on a given ground motion field for different assets of the same taxonomy are assumed to be  correlated in this case, we proceed by first generating a vector of \emph{epsilons} for each taxonomy from the multivariate standard normal distribution which has the symmetric covariance matrix with $1.0$ as the diagonal elements and $\rho = 0.5$ as the off-diagonal elements.

Now, for each asset of that taxonomy, the parameters $m$ and $s$ are obtained for the ground motion value at the location of the asset through interpolation on the specified vulnerability model. Each asset of a particular taxonomy is also assigned a value of $\epsilon$ from the vector of \emph{epsilons} for that taxonomy sampled as described above. The parameters $m$ and $s$ are then converted to the parameters $\mu$ and $\sigma$ of the corresponding normal distribution, and the sampled loss ratio is obtained simply as $\exp (\mu + \epsilon * \sigma)$.

After the asset event loss tables are compiled by sampling correlated loss values as described above, the rest of the calculation concerning the derivation of loss curves and average loss proceeds as in previous cases.