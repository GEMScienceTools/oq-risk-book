Whereas the previous case was concerned with checking the correct implementation and usage of \emph{discrete} fragility functions, the purpose of this case is to verify the correct calculation of damage distribution statistics for the classical damage calculator using \emph{continuous} (lognormal CDF) fragility functions.

Table~\ref{tab:ff-cont-tax1-zmin} shows the mean and standard deviation of the ground motion intensity level for the four damage states, which are the parameters for the lognormal fragility function used in this test case.

The only difference in the calculation procedure compared with Case~1a is the criterion used for selecting the set of intensity levels at which to compute the hazard curve. OpenQuake discretizes the continuous lognormal fragility functions into a set of intensity levels and corresponding probabilities of exceedance for each damage state. The minimum and maximum intensity levels used for this discretization are those specified in the fragility model file. The number of intervals into which this range of intensity levels is discretized is specified using the configuration parameter `continuous\_fragility\_discretization'. In this case, the minimum and maximum intensity levels specifed in the continuous fragility model are $0.0 g$ and $5.0 g$ respectively. The parameter `continuous\_fragility\_discretization' is set to 29, which means that OpenQuake will discretize the interval $[0.0, 5.0]g$ into 29 equal subintervals, yielding thirty intensity levels including the minimum and maximum limits.

The damage state exceedance probabilities are obtained by evaluating the complementary cumulative distribution function (CCDF) of the lognormal distribution at each of these thirty intensity levels, for the set of four damage states. Now, the calculation proceeds in the same manner as described in Case~1a above.