This case is designed to test the computation of the mean and standard deviation of individual asset losses and also for the portfolio, when the vulnerability models of different assets of the same taxonomy are treated as uncorrelated. In OpenQuake, this can be specified in the job configuration file, by setting the value of the parameter `asset\_correlation' to zero.

Ten thousand ground motion fields are generated for the given rupture, taking into consideration both the inter-event and intra-event variability in the ground motion. The ground motion prediction equation used is Boore and Atkinson (2008), and the Jayaram and Baker (2009) model for spatial correlation of ground motion values is applied. These ground motion fields are also used for the corresponding calculation in Julia.

\begin{table}[htbp]

\centering
\begin{tabular}{ l c c l }

\hline

\hline
\end{tabular}
\rowcolor{anti-flashwhite}
\bf{GMF \#} & \bf{Site 1} & \bf{Site 2} & \bf{Site 3} & \bf{Site 4} & \bf{Site 5} & \bf{Site 6} & \bf{Site 7}\\
1 & 1.125 & 0.261 & 0.073 & 0.756 & 0.121 & 0.671 & 0.274 \\
2 & 0.386 & 0.198 & 0.058 & 0.660 & 0.396 & 0.529 & 0.297 \\
\vdots & \vdots & \vdots & \vdots & \vdots & \vdots & \vdots & \vdots \\
10,000 & 0.265 & 0.106 & 0.114 & 0.213 & 0.293 & 0.639 & 0.093 \\
\hline
\end{tabular}

\caption{10,000 simulated ground motion fields with spatial correlation}
\label{tab:gmfs-sim-l7-10000}
\end{table}
\input{qareport/tables/tab-assets-tax1}

Table~\ref{tab:gmfs-sim-l7-10000} lists five of the ten thousand ground motion fields generated by the OpenQuake scenario hazard calculator. The list of assets in the exposure model used for the test cases involving asset correlation (Cases~4b, 4c, and 4d) is given in Table~\ref{tab:assets-tax1}. Table~\ref{tab:vf-ln-tax1-nzcov} shows the mean loss ratios and corresponding coefficients of variation in the vulnerability function used in this test case.

Since the sampled loss ratios conditional on a given ground motion field for different assets of the same taxonomy are assumed to be uncorrelated in this case, a loss ratio is sampled independently for each asset from the univariate lognormal distribution for that asset for each ground motion field.