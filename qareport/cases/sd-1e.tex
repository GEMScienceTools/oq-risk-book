The purpose of this case is to test the computation of the mean and standard deviation of the damage state probabilities when the ground motion fields are not predefined as in previous cases, but generated by OpenQuake by sampling from the distribution defined by the selected ground motion prediction equation. Ten thousand ground motion fields are generated for the given rupture, taking into consideration both the inter-event and intra-event variability in the ground motion. The ground motion prediction equation used is \citet{boore2008}.

Table \ref{tab:gmfs-sim-l1-10000} lists five of the ten thousand ground motion values generated by OpenQuake. Table~\ref{tab:ff-disc-tax1-zndl} shows the set of ground motion intensity levels and corresponding probabilities of exceedance for the four damage states for the discrete fragility function used in this test case.

In order to check the OpenQuake results, an alternate implementation of the calculator algorithm in the programming language Julia is used for comparison. In order to provide a representative baseline for the comparison, one million ground motion fields are used in the Julia calculation.

The mean and standard deviation of the logarithm of the ground motion calculated at the location of the asset as obtained by using the \citet{boore2008} equation are $-0.648$ and $0.564$ respectively. Assuming a lognormal distribution for the variability in the ground motion, one million ground motion values are generated using Julia with these logarithmic mean and standard deviation values.

For each simulated ground motion value, the damage state exceedance probabilities are obtained through interpolation on the discrete fragility function specified for this case. The probabilities of damage state occurrence are calculated by the pairwise differences of the exceedance probabilities of adjacent damage states as described in Case~1a.

The mean and standard deviation of the damage state probabilities and also the probability of observing no damage are finally calculated using the above sets of probabilities collected from each of the one million ground motion simulations.