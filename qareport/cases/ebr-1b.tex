The time period for both the hazard and the risk calculations in this case are one year. This case uses a collection of 100,000 stochastic event sets, each spanning one year, to generate a set of ground motion fields representative of the seismicity of the specified region, collectively spanning a period of 100,000 years. A single stochastic event set (SES) may contain zero or more ruptures that are generated based on the frequency distribution of the sources. Each SES in this case represents one simulation of the possible events that might occur in one year.

Table~\ref{tab:vf-ln-tax1-zcov} shows the mean loss ratios and corresponding coefficients of variation in the lognormal vulnerability function used in this case. There is no uncertainty in the vulnerability function used for this case. The coefficient of variation of the loss ratio is zero at all intensity measure levels. The purpose of this case is to test the correct interpolation of the mean loss ratios of the vulnerability function at intermediate intensity measure levels.

Since there is no variability in the loss ratio, calculation of the loss ratios for each ground motion field is straightforward in this case. Since the coefficients of variation in the vulnerability function are all zero, the lognormal distribution devolves into the degenerate distribution. The hazard calculation produces 4,115 ruptures over the 100,000 cumulative time span, and 4,115 corresponding ground motion fields. The ground motion values at the location of the single asset are $[0.074, 0.154, 0.118, 0.203, \dots, 0.288] g$ (4,115 values in total).

The calculation of the loss ratios given the ground motion values proceeds in exactly the same manner as described in the Scenario Risk calculator examples. The mean loss ratio and coefficient of variation of the loss ratio are obtained by linear interpolation from the provided vulnerability model for each of the above 4,115 ground motion values. Finally, a loss ratio is sampled from the lognormal distribution defined by the interpolated mean and standard deviation parameters for each ground motion value.

These numbers are multiplied by the asset value of $10,000$ to give 4,115 sampled loss values.