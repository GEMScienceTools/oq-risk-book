The OpenQuake scenario damage calculator allows the user to employ more than one ground motion prediction equation (GMPE) for computing the ground motion fields used for the loss calculation. The mean and standard deviation of the damage state probabilities (or fractions), are calculated and output for each GMPE branch independently. No sampling is involved, and any branch weights assigned to the different GMPE branches are ignored.

A single asset is used in this test case. The two ground motion prediction equations used are \citet{boore2008}, and \citet{chiou2008}. Table~\ref{tab:ff-disc-tax1-zndl} shows the parameters of the discrete fragility function used in this test case. Two hazard curves are computed, one for each of the GMPE branches specified in the logic tree. The procedure used for calculating the damage state probabilities for the asset remains the same as described in Case~1a. The damage state probabilities are computed individually for each of the two hazard branches.