The purpose of this case is to test the computation of the mean and standard deviation of the number of buildings in each damage state, given variability in the ground motion values. The computation of the mean and standard deviation of the portfolio loss is also tested in this case.

\begin{table}

\centering
\begin{tabular}{ c c c c l }

\hline
\rowcolor{anti-flashwhite}
\bf{Site} & \bf{Taxonomy} & \bf{Latitude} & \bf{Longitude} & \bf{Comment} \\
\hline
1 & tax1 & 38.113 & -122.000 & On fault midpoint, along strike \\
2 & tax2 & 38.113 & -122.114 & 10 km west of fault, at midpoint \\
3 & tax1 & 38.113 & -122.570 & 50 km west of fault, at midpoint \\
4 & tax3 & 38.000 & -122.000 & South end of fault \\
5 & tax1 & 37.910 & -122.000 & 10 km south of fault, along strike \\
6 & tax2 & 38.225 & -122.000 & North end of fault \\
7 & tax1 & 38.113 & -121.886 & 10 km east of fault, at midpoint \\
\hline
\end{tabular}

\caption{Asset sites and taxonomies for the multiple-asset, multiple-taxonomy test cases}
\label{tab:assets-tax3}
\end{table}
The list of assets and their taxonomies are shown in Table~\ref{tab:assets-tax3}.

Five precomputed ground motion fields are used as the starting point for this case. These ground motion fields take into consideration both the inter-event and intra-event variability in the ground motion. The ground motion prediction equation used is Boore and Atkinson (2008), and the Jayaram and Baker (2009) model for spatial correlation of ground motion values is applied.

\begin{table}[htbp]

\centering
\begin{tabular}{ c c c c c c c c }

\hline
\rowcolor{anti-flashwhite}
\bf{GMF \#} & \bf{Site 3} & \bf{Site 2} & \bf{Site 5} & \bf{Site 4} & \bf{Site 1} & \bf{Site 6} & \bf{Site 7}\\
\hline
1 & 0.15g & 0.17g & 0.21g & 0.56g & 0.25g & 0.38g & 0.14g \\
2 & 0.05g & 0.21g & 0.18g & 0.69g & 0.94g & 0.72g & 0.43g \\
3 & 0.05g & 0.18g & 0.06g & 0.58g & 0.46g & 0.24g & 0.22g \\
4 & 0.15g & 0.46g & 0.72g & 0.79g & 0.81g & 0.29g & 0.51g \\
5 & 0.15g & 0.48g & 0.95g & 1.70g & 1.70g & 0.63g & 0.25g \\
\hline
\end{tabular}

\caption{Five precomputed spatially correlated ground motion fields (PGA). The sites are sorted first by longitude, then by latitude.}
\label{tab:gmfs-diff-l7-5}
\end{table}

Table~\ref{tab:gmfs-diff-l7-5} lists the ground motion fields used in this test case.


\begin{table}[htbp]

\centering
\begin{tabular}{ l l c c c c c c c c }

\hline
\rowcolor{anti-flashwhite}
\bf{Taxonomy} & \bf{PGA} & \bf{0.05g} & \bf{0.20g} & \bf{0.40g} & \bf{0.60g} & \bf{0.80g} & \bf{1.00g} & \bf{\dots} & \bf{2.00g} \\
\hline
\bf{tax1} & \bf{Mean LR} & 0.01 & 0.04 & 0.10 & 0.20 & 0.33 & 0.50 & \dots & 0.99 \\
    & \bf{CoV LR} & 0.0 & 0.0 & 0.0 & 0.0 & 0.0 & 0.0 & \dots & 0.0 \\
\hline
\bf{tax2} & \bf{Mean LR} & 0.01	& 0.02 & 0.05 & 0.11 & 0.18 & 0.26 & \dots & 0.51 \\
    & \bf{CoV LR} & 0.0 & 0.0 & 0.0 & 0.0 & 0.0 & 0.0 & \dots & 0.0 \\
\hline
\bf{tax3} & \bf{Mean LR} & 0.01 & 0.04 & 0.09 & 0.18 & 0.28 & 0.47 & \dots & 0.91 \\
    & \bf{CoV LR} & 0.0 & 0.0 & 0.0 & 0.0 & 0.0 & 0.0 & \dots & 0.0 \\
\hline
\end{tabular}

\caption{Lognormal vulnerability functions for three building typologies}
\label{tab:vf-ln-tax3-zcov}
\end{table}

Table~\ref{tab:vf-ln-tax3-zcov} shows the mean loss ratios and corresponding coefficients of variation in the vulnerability function used in this test case.

Since there is no variability in the loss ratio, calculation of the loss ratios is straightforward in this case.