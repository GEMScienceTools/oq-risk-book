The purpose of this case is to test the computation of the mean and standard deviation of the damage state probabilities for a portfolio of assets when the ground motion fields are not predefined, but generated by OpenQuake by sampling from the distribution defined by the selected ground motion prediction equation.

The exposure and fragility models used in this case are the same as those used in Case~2a. Ten thousand ground motion fields are generated for the given rupture, taking into consideration both the inter-event and intra-event variability in the ground motion. The ground motion prediction equation used is \citet{boore2008}, and the \citet{jayaram2009} model for spatial correlation of ground motion values is applied.

Table~\ref{tab:gmfs-sim-l7-10000} lists five of the ten thousand ground motion fields generated by the OpenQuake scenario hazard calculator.

In order to provide a representative baseline for the comparison, one million ground motion fields are used in the Julia implementation of the calculator.

\begin{table}[htbp]

\centering
\begin{tabular}{ l l l r r r }

\hline
\rowcolor{anti-flashwhite}
\bf{Asset} & \bf{Damage State} & \bf{Result} & \bf{Expected} & \bf{OpenQuake} & \bf{Difference}\\
\hline
a1 & none & Mean & 0.3726 & 0.3726 & 0.00\% \\
   &      & Std. & 0.2420 & 0.2405 & 0.61\% \\
   & ds1  & Mean & 0.2988 & 0.2996 & -0.26\% \\
   &      & Std. & 0.0837 & 0.0824 & 1.65\% \\
   & ds2  & Mean & 0.1428 & 0.1429 & -0.08\% \\
   &      & Std. & 0.0676 & 0.0673 & 0.53\% \\
   & ds3  & Mean & 0.0731 & 0.0731 & 0.02\% \\
   &      & Std. & 0.0505 & 0.0505 & -0.09\% \\
   & ds4  & Mean & 0.1127 & 0.1118 & 0.78\% \\
   &      & Std. & 0.1368 & 0.1345 & 1.70\% \\
\hline

\hline

\hline
\end{tabular}

\caption{Results for scenario damage test case 2b — individual assets}
\label{tab:result-sd-2b}
\end{table}
\begin{table}[htbp]

\centering
\begin{tabular}{ l l l r r r }

\hline
\rowcolor{anti-flashwhite}
\bf{Taxonomy} & \bf{Damage State} & \bf{Result} & \bf{Expected} & \bf{OpenQuake} & \bf{Difference}\\
\hline
tax1 & none & Mean & 2.4752 & 2.4752 & 0.00\% \\
     & ds1  & Mean & 0.7294 & 0.7294 & 0.00\% \\
     & ds2  & Mean & 0.3257 & 0.3257 & 0.00\% \\
     & ds3  & Mean & 0.1736 & 0.1736 & 0.00\% \\
     & ds4  & Mean & 0.2962 & 0.2962 & 0.00\% \\
\hline
tax2 & none & Mean & 1.6703 & 1.6703 & 0.00\% \\
     & ds1  & Mean & 0.1832 & 0.1832 & 0.00\% \\
     & ds2  & Mean & 0.1082 & 0.1082 & 0.00\% \\
     & ds3  & Mean & 0.0304 & 0.0304 & 0.00\% \\
     & ds4  & Mean & 0.0078 & 0.0078 & 0.00\% \\
\hline
tax3 & none & Mean & 0.6130 & 0.6130 & 0.00\% \\
     & ds1  & Mean & 0.1422 & 0.1422 & 0.00\% \\
     & ds2  & Mean & 0.1800 & 0.1800 & 0.00\% \\
     & ds3  & Mean & 0.0467 & 0.0467 & 0.00\% \\
     & ds4  & Mean & 0.0181 & 0.0181 & 0.00\% \\
\hline
\end{tabular}

\caption{Results for scenario damage test case 2a — aggregated by taxonomy}
\label{tab:result-sd-2b-tax}
\end{table}
\begin{table}[htbp]

\centering
\begin{tabular}{ l l r r r }

\hline
\rowcolor{anti-flashwhite}
\bf{Damage State} & \bf{Result} & \bf{Expected} & \bf{OpenQuake} & \bf{Difference}\\
\hline
none & Mean & 4.7585 & 4.7585 & 0.00\% \\
 ds1 & Mean & 1.0547 & 1.0547 & 0.00\% \\
 ds2 & Mean & 0.6140 & 0.6140 & 0.00\% \\
 ds3 & Mean & 0.2507 & 0.2507 & 0.00\% \\
 ds4 & Mean & 0.3221 & 0.3221 & 0.00\% \\
\hline
\end{tabular}

\caption{Results for scenario damage test case 2a — overall portfolio}
\label{tab:result-sd-2b-total}
\end{table}

Table~\ref{tab:result-sd-2a} shows the comparison of the OpenQuake result with the expected result for assets $a1$, $a2$, and $a3$. Table~\ref{tab:result-sd-2a-tax} shows the comparison of the OpenQuake result with the expected result for taxonomies $tax1$, $tax2$, and $tax3$. Finally, Table~\ref{tab:result-sd-2b-total} shows the comparison of the OpenQuake result with the expected result for the overall portfolio.