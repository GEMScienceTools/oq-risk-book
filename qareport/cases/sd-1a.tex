Test case 1a uses a set of 5 precomputed ground motion values to test the correct interpolation of the damage state exceedance probabilities of the discrete fragility function at intermediate intensity measure levels.

Table~\ref{tab:gmfs-diff-l1-5} lists the five ground motion values used in this test case.

\begin{table}[htbp]

\centering
\begin{tabular}{ c c c c c c c c c c }

\hline
\rowcolor{anti-flashwhite}
\bf{LS | PGA} & \bf{0.2g} & \bf{0.4g} & \bf{0.6g} & \bf{0.8g} & \bf{1.0g} & \bf{1.2g} & \bf{1.4g} & \bf{\dots} & \bf{5.0g} \\
\hline
\bf{ds1} & 0.000 & 0.152 & 0.846 & 0.993 & 1.000 & 1.000 & 1.000 & \dots & 1.000 \\
\bf{ds2} & 0.000 & 0.014 & 0.129 & 0.350 & 0.576 & 0.747 & 0.857 & \dots & 1.000 \\
\bf{ds3} & 0.000 & 0.008 & 0.085 & 0.196 & 0.325 & 0.450 & 0.561 & \dots & 0.993 \\
\bf{ds4} & 0.000 & 0.006 & 0.067 & 0.171 & 0.263 & 0.354 & 0.438 & \dots & 0.951 \\
\hline
\end{tabular}

\caption{Discrete fragility function with zero no damage limit}
\label{tab:ff-disc-tax1-zndl}
\end{table}

Table~\ref{tab:ff-disc-tax1-zndl} shows the set of ground motion intensity levels and corresponding probabilities of exceedance for the four damage states for the discrete fragility function used in this test case.

Since all ground motion values are identical, calculation of the damage distribution is straightforward in this case. The ground motion value at the location of the single asset is $PGA = 0.5 g$. The discrete fragility function for this case provides damage state probabilities of exceedance at intensity measure levels $0.4 g$ and $0.6 g$, but none at $0.5 g$. The exceedance probabilities at $0.4 g$ and $0.6 g$ corresponding to the discrete damage states $[ds_1, ds_2, ds_3, ds_4]$ are $[0.152, 0.014, 0.008, 0.006]$ and $[0.846, 0.129, 0.085, 0.067]$ respectively.

The exceedance probabilities at $0.5 g$ are obtained by interpolating between these two sets of values. Linear interpolation gives exceedance probabilities of $[0.499, 0.0715, 0.08225, 0.0365]$ for $PGA = 0.5 g$. The probabilities of damage state occurrence are given by the pairwise differences of the exceedance probabilities as $[0.4990 - 0.0715, 0.0715 - 0.08225, 0.08225 - 0.0365, 0.0365]$

Since there is no variability in the ground motion, the mean loss ratio is also $0.15$, and the standard deviation of the loss ratio is $0.0$.

These numbers are multiplied by the asset value of $10,000$ to give the mean and standard deviation of loss for the scenario as $1,500$ and $0$ respectively.