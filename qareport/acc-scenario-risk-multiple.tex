The multiple asset test cases are designed to test the loss aggregation functions of the scenario risk calculator, such as:

\begin{itemize}
\item portfolio loss computation for a given ground motion field
\item calculation of mean and standard deviation of portfolio scenario loss
\end{itemize}

\begin{table}

\centering
\begin{tabular}{ c l c c l }

\hline
\rowcolor{anti-flashwhite}
\bf{Site} & \bf{Taxonomy} & \bf{Latitude} & \bf{Longitude} & \bf{Comment} \\
\hline
1 & Wood & 38.113 & -122.000 & On fault midpoint, along strike \\
2 & Wood & 38.113 & -122.114 & 10 km west of fault, at midpoint \\
3 & RC & 38.113 & -122.570 & 50 km west of fault, at midpoint \\
4 & RC & 38.000 & -122.000 & South end of fault \\
5 & Steel & 37.910 & -122.000 & 10 km south of fault, along strike \\
6 & Wood & 38.225 & -122.000 & North end of fault \\
7 & Steel & 38.113 & -121.886 & 10 km east of fault, at midpoint \\
\hline
\end{tabular}

\caption{Asset sites and taxonomies for the multiple-asset test cases}
\label{tab:assets}
\end{table}

The list of assets in the exposure model used for the multiple-asset test cases for the scenario risk calculator is given in Table \ref{tab:assets}.

Ten thousand ground motion fields are generated for the given rupture, taking into consideration both the inter-event and intra-event variability in the ground motion. The ground motion prediction equation used is Boore and Atkinson (2008), and the Jayaram and Baker (2009) model for spatial correlation of ground motion values is applied.

\begin{table}[htbp]

\centering
\begin{tabular}{ l c c l }

\hline
\rowcolor{anti-flashwhite}
\bf{GMF #} & \bf{Site} & \bf{IMT} & \bf{GMV}\\
\hline
1 & 1 & PGA &  \\
1 & 2 & PGA &  \\
1 & 3 & PGA &  \\
1 & 4 & PGA &  \\
1 & 5 & PGA &  \\
1 & 6 & PGA &  \\
1 & 7 & PGA &  \\
\hline
2 & 1 & PGA &  \\
2 & 2 & PGA &  \\
2 & 3 & PGA &  \\
2 & 4 & PGA &  \\
2 & 5 & PGA &  \\
2 & 6 & PGA &  \\
2 & 7 & PGA &  \\
\hline
\vdots & \vdots & \vdots & \vdots \\
\hline
1000 & 1 & PGA &  \\
1000 & 2 & PGA &  \\
1000 & 3 & PGA &  \\
1000 & 4 & PGA &  \\
1000 & 5 & PGA &  \\
1000 & 6 & PGA &  \\
1000 & 7 & PGA &  \\
\hline
\end{tabular}

\caption{Ground motion fields for the multiple asset tests}
\label{tab:scenario-gmfs-single}
\end{table}

Table \ref{tab:scenario-gmfs-multiple} lists three of the ten thousand ground motion fields generated.

% ---------------------------------------------------------------------------
\subsubsection{Case 5a}
The purpose of this case is to test the computation of the mean and standard deviation of the loss for multiple assets, given variability in both the ground motion values and in the vulnerability function. The computation of the mean and standard deviation of the portfolio loss is also tested in this case. The asset vulnerability correlation factor is zero for this test case.


\begin{table}[htbp]

\centering
\begin{tabular}{ c l r r r }

\hline
\rowcolor{anti-flashwhite}
\bf{Asset} & \bf{Result} & \bf{Expected} & \bf{OpenQuake} & \bf{Difference}\\
\hline
a1 & Mean loss & 2,383.1 & 2,370.7 & 0.52\% \\
 & Std. loss & 2,419.3 & 2,401.7 & 0.72\% \\
\hline
a2 & Mean loss &  &  & \% \\
 & Std. loss &  &  & \% \\
\hline
a3 & Mean loss &  &  & \% \\
 & Std. loss &  &  & \% \\
\hline
a4 & Mean loss &  &  & \% \\
 & Std. loss &  &  & \% \\
\hline
a5 & Mean loss &  &  & \% \\
 & Std. loss &  &  & \% \\
\hline
a6 & Mean loss &  &  & \% \\
 & Std. loss &  &  & \% \\
\hline
a7 & Mean loss &  &  & \% \\
 & Std. loss &  &  & \% \\
\hline
Total & Mean loss &  &  & \% \\
 & Std. loss &  &  & \% \\
\hline
\end{tabular}

\caption{Results for scenario risk test case 5a}
\label{tab:result-scenario-risk-5a}
\end{table}

Table \ref{tab:result-scenario-risk-5a} shows the comparison of the OpenQuake result with the expected result.

