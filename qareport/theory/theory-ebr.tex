This calculator employs an event-based Monte Carlo simulation approach to probabilistic risk assessment in order to estimate the loss distribution for individual assets and aggregated loss distribution for a spatially distributed portfolio of assets within a specified time period. The calculator requires the definition of an exposure model, a vulnerability model for each loss type of interest with vulnerability functions for each taxonomy represented in the exposure model, and a set of ground motion fields representative of the seismicity of the region over the specified time period. Loss curves and loss maps can currently be calculated for five different loss types using this calculator: structural losses, nonstructural losses, contents losses, downtime losses, and occupant fatalities. The main results of this calculator are loss exceedance curves for each asset, which describe the probability of exceedance of different loss levels over the specified time period, and loss maps for the region, which describe the loss values that have a given probability of exceedance over the specified time period. Aggregate loss exceedance curves can be also be produced using this calculator; these describe the probability of exceedance of different loss levels for all assets of a single taxonomy, or for all assets in the portfolio, over the specified time period. Finally, event loss tables can be produced using this calculator; these tables describe the total loss across the portfolio for each seismic event in the stochastic event set.

This calculator relies on the probabilistic event-based hazard calculator, which simulates the seismicity of the chosen time period $T$ by producing a \textit{stochastic event set} (also known as a \textit{synthetic catalog}). For each rupture generated by a source, the number of occurrences in the given time span $T$ is simulated by sampling the corresponding probability distribution as given by $P_{rup}(k | T)$. A stochastic event set is therefore a \textit{sample} of the full population of ruptures as defined by a seismic source model. Each rupture is present zero, one or more times, depending on its probability. Symbolically, we can define a stochastic event set ($SES$) as:
\begin{align}
SES(T) = \left\{k \times rup,\;k\sim P_{rup}(k | T)\;\;\forall\;rup\;in\;Src\;\forall\;Src\;in\;SSM\right\}
\end{align}
where $k$, the number of occurrences, is a random sample of $P_{rup}(k | T)$, and $k \times rup$ means that rupture $rup$ is repeated $k$ times in the stochastic event set.

For each event in the stochastic event sets, a spatially correlated ground motion field (GMF) realisation is generated, taking into consideration both the inter-event variability of ground motions, and the intra-event residuals obtained from a spatial correlation model for ground motion residuals. The use of logic-trees allows for the consideration of uncertainty in the choice of a seismic source model, in the choice of GMPE models for the different tectonic regions, and in the choice of vulnerability functions for the different taxonomy types in the exposure model.

For each GMF realization, a loss ratio is sampled for every asset in the exposure model using the provided probabilistic vulnerability model, taking into consideration the correlation model for vulnerability of different assets of a given taxonomy. Finally loss exceedance curves are computed for both ground-up losses and insured losses.