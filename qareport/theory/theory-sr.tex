The scenario risk calculator computes loss statistics for all assets in a given exposure model for a single specified earthquake rupture. Loss statistics include the mean and standard deviation of ground-up losses and insured losses for each loss type considered in the analysis. Loss statistics can currently be computed for five different loss types using this calculator: structural losses, nonstructural losses, contents losses, downtime losses, and occupant fatalities. This calculator requires the definition of a finite rupture model, an exposure model and a vulnerability model for each loss type considered; the main results are the loss statistics per asset and mean loss maps.

The rupture characteristics—i.e. the magnitude, hypocenter and fault geometry—are modelled as deterministic in the scenario calculators. Multiple realizations of different possible ground motion fields (GMFs) due to the single rupture are generated, taking into consideration both the inter-event variability of ground motions, and the intra-event residuals obtained from a spatial correlation model for ground motion residuals. The use of logic-trees allows for the consideration of uncertainty in the choice of a ground motion model for the given tectonic region.

As an alternative to computing the GMFs with OpenQuake, users can also provide their own sets of GMFs as input to the scenario risk calculator.

For each GMF realization, a loss ratio is sampled for every asset in the exposure model using the provided probabilistic vulnerability model, taking into consideration the correlation model for vulnerability of different assets of a given taxonomy. Finally loss statistics, i.e., the mean loss and standard deviation of loss for both ground-up losses and insured losses across all realizations, are calculated for each asset. Mean loss maps are also generated by this calculator, describing the mean ground-up losses and mean insured losses caused by the scenario event for the different assets in the exposure model.

Given this set of functionalities, the acceptance testing framework for the scenario risk calculator is divided into the following four test sets:

\begin{enumerate}
	\item Single asset test cases
	\item Multiple asset test cases
	\item Insured losses test cases
	\item Logic-tree test cases
\end{enumerate}

The individual test cases are described in more detail in \ref{sec:acc-sr}.