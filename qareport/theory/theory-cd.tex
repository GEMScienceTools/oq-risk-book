The classical PSHA-based damage calculator integrates the fragility functions for an asset with the seismic hazard curve at the location of the asset, to give the expected damage distribution for the asset within a specified time period. The calculator requires the definition of an exposure model, a fragility model for each taxonomy represented in the exposure model, and hazard curves calculated in the region of interest. The main results of this calculator are the expected damage distribution for each asset, which describe the probability of the asset being in different damage states, and collapse maps for the region, which describe the probability of collapse for different assets in the portfolio over the specified time period.

The hazard curves required for this calculator can be calculated by the OpenQuake engine for all asset locations in the exposure model using the classical PSHA approach \citep{cornell1968, mcguire1976}. The use of logic-trees allows for the consideration of uncertainty in the choice of a GMPE model for the different tectonic region types in the region and in the choice of vulnerability functions for the different taxonomy types in the exposure model.

Given this set of functionalities, the acceptance testing framework for the classical PSHA-based damage calculator is divided into the following three test sets:

\begin{enumerate}
	\item Single asset test cases
	\item Multiple asset test cases
	\item Logic-tree test cases
\end{enumerate}

The individual test cases are described in more detail in \ref{sec:acc-cd}.
