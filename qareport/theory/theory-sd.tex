The scenario damage calculator computes damage distribution statistics for all assets in a given exposure model for a single specified earthquake rupture. Damage distribution statistics include the mean and standard deviation of damage fractions for different damage states. This calculator requires the definition of a finite rupture model, an exposure model and a fragility model; the main results are the damage distribution statistics per asset, aggregated damage distribution statistics per taxonomy, aggregated damage distribution statistics for the region, and collapse maps.

The rupture characteristics—i.e. the magnitude, hypocenter and fault geometry—are modelled as deterministic in the scenario calculators. Multiple realizations of different possible ground motion fields (GMFs) due to the single rupture are generated, taking into consideration both the inter-event variability of ground motions, and the intra-event residuals obtained from a spatial correlation model for ground motion residuals. The use of logic-trees allows for the consideration of uncertainty in the choice of a ground motion model for the given tectonic region.

As an alternative to computing the GMFs with OpenQuake, users can also provide their own sets of GMFs as input to the scenario damage calculator.

For each GMF realization, damage fractions (the fraction of buildings in each damage state) are estimated for every asset in the exposure model using the provided fragility model, and finally the damage distribution statistics (i.e., the mean damage fractions and standard deviation of damage fractions for all damage states) across all realizations are calculated. The calculator also provides aggregated damage distribution statistics for the portfolio, such as mean damage fractions and standard deviation of damage fractions for each taxonomy in the exposure model, and the mean damage fractions and standard deviation of damage fractions for the entire region of study.

Given this set of functionalities, the acceptance testing framework for the scenario damage calculator is divided into the following three test sets:

\begin{enumerate}
	\item Single asset test cases
	\item Multiple asset test cases
	\item Logic-tree test cases
\end{enumerate}

The individual test cases are described in more detail in \ref{sec:acc-sd}.