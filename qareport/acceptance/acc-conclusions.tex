The primary objective of this project was to develop a set of standard test-cases that can be used to validate the risk component of the OpenQuake-engine. Although it is practically impossible to guarantee that a complex software like the OpenQuake-engine is completely error-free, nevertheless having in place a rigorous testing framework incorporating a large set of input models and expected outputs affords a higher degree of confidence to both the developers and the users of the software. For the developers, it provides a means to ensure that the code they write performs according to the program specifications. For risk-modellers using OpenQuake for seismic risk assessment and other end-users of their analysis results, it provides reassurance that the calculations are done using a software that has undergone several levels of quality assurance checks.

Since this extensive level of testing comes attached with a significant increase in development effort, the amount of time spent on quality assurance and testing should be balanced with other tasks such as adding new features and performance improvements. Resorting to a time-boxed approach for testing has the drawback that several errors may remain undiscovered due to lack of sufficient time for testing. An open-ended approach is also not without disadvantages - as more tests will almost always lead to more discoveries of errors, particularly for large and complex software. In general, a diligent probing of the software for errors is likely to uncover an increasing number of trivial errors for common use-cases at first, followed by fewer, more subtle errors for obscure use-cases at later stages. At this stage, a decision can be taken to halt the development of any additional test cases and document the results of the testing process.

The suite of test cases described in this chapter were designed with the intent of exposing errors in the implementation of the OpenQuake risk calculators. During the testing phase, these tests did indeed lead to the uncovering of several previously unknown errors and unexpected behaviour, all of which have now been corrected. Debugging these errors was greatly aided by the availability of test cases that target specific features of the calculators that help narrow the range of possible locations of the errors. It should be noted that uncovering a new error or unexpected behaviour is not a failure, but indeed a success of a well-designed test case.

From the results for the test cases described in this chapter, it is seen that for cases not involving random sampling, the OpenQuake results match exactly with answers established through analytical calculations. For cases that do involve random sampling of the ground motion fields or loss ratios, the difference between the OpenQuake results and those obtained through alternate implementations of the calculator logic are well within 5\%. These results indicate that for most of the common use-cases, the OpenQuake risk calculators provide the expected results.

In conclusion, the testing framework for the OpenQuake risk calculators undoubtedly adds value to the overall quality of the program and increases its reliability.