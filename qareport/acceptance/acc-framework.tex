\section{Verification Framework}
The main purpose of the acceptance tests is to ensure that the risk calculators work according to the design specifications and to verify that the calculators produce correct results for a variety of input cases. Correctness of the test case results is verified by comparing with hand calculations for the simple test cases or with alternate implementaions in Julia for the complex cases.

\section{Theoretical Background}
% \subsection{Basic concepts}
%    \label{subsec:theory-basic-concepts}
%    % Rupture
% GMF
% SES
% VF
% FF
% HC
% Monte Carlo
% GMPE

An earthquake \textit{rupture model} describes the magnitude, geometry, and source typology of an earthquake occurrence.

Given an earthquake rupture, the simulation of ground shaking values on a set of locations $\bm{x}=(x_{1}, x_{2}, ..., x_{N})$
forms a \textit{ground motion field} ($GMF$). The ground motion field is simulated by sampling the probability distribution defined by the ground motion model.

Seismicity in a region is described by a seismic source model ($SSM$), which is a collection of \textit{independent seismic sources}. Independence of seismic sources implies that the occurrence of an earthquake rupture in a source does not affect the probability of earthquake occurrence in the other sources.

The main parameters describing a seismic source are the geometry constraining the earthquake rupture locations, and the \textit{magnitude-frequency distribution}, defining the average annual occurrence rate over a magnitude range. A seismic source model ($SSM$) can be therefore defined as a set of $I$ seismic sources ($Src$):
\begin{equation}
SSM = \left\{Src_{1}, Src_{2}, ..., Src_{I}\right\}
\end{equation}

Each source generates \textit{independent earthquake ruptures}. Independence of earthquake ruptures implies that the occurrence of an earthquake rupture in a source does not affect the probability of occurrence of the other potential earthquake ruptures in the same source. A generic $i$-th source defines therefore a set of $J$ earthquake ruptures:
\begin{equation}
Src_{i} = \left\{Rup_{i1}, Rup_{i2}, ..., Rup_{iJ}\right\}
\end{equation}





Probabilistic seismic hazard analysis allows calculating the probabilities of exceeding,
at least once in a given time span, and at a given site, a set of ground motion
parameter levels considering all possible earthquake ruptures defined in a
seismic source model. Such a list of probability values is usually referred to
as \textit{hazard curve}.


\subsection{Scenario risk}
   \label{subsec:theory-sr}
   The scenario risk calculator computes loss statistics for all assets in a given exposure model for a single specified earthquake rupture. Loss statistics include the mean and standard deviation of ground-up losses and insured losses for each loss type considered in the analysis. Loss statistics can currently be computed for five different loss types using this calculator: structural losses, nonstructural losses, contents losses, downtime losses, and occupant fatalities. This calculator requires the definition of a finite rupture model, an exposure model and a vulnerability model for each loss type considered; the main results are the loss statistics per asset and mean loss maps.

The rupture characteristics—i.e. the magnitude, hypocenter and fault geometry—are modelled as deterministic in the scenario calculators. Multiple realizations of different possible ground motion fields (GMFs) due to the single rupture are generated, taking into consideration both the inter-event variability of ground motions, and the intra-event residuals obtained from a spatial correlation model for ground motion residuals. The use of logic-trees allows for the consideration of uncertainty in the choice of a GMPE model for the given tectonic region and in the choice of vulnerability functions for the different taxonomy types in the exposure model.

As an alternative to computing the GMFs with OpenQuake, users can also provide their own sets of GMFs as input to the scenario risk calculator.

For each GMF realization, a loss ratio is sampled for every asset in the exposure model using the provided probabilistic vulnerability model, taking into consideration the correlation model for vulnerability of different assets of a given taxonomy. Finally loss statistics, i.e., the mean loss and standard deviation of loss for both ground-up losses and insured losses across all realizations, are calculated for each asset. Mean loss maps are also generated by this calculator, describing the mean ground-up losses and mean insured losses caused by the scenario event for the different assets in the exposure model.

\subsection{Scenario damage}
   \label{subsec:theory-sd}
   The scenario damage calculator computes damage distribution statistics for all assets in a given exposure model for a single specified earthquake rupture. Damage distribution statistics include the mean and standard deviation of damage fractions for different damage states. This calculator requires the definition of a finite rupture model, an exposure model and a fragility model; the main results are the damage distribution statistics per asset, aggregated damage distribution statistics per taxonomy, aggregated damage distribution statistics for the region, and collapse maps.

The rupture characteristics—i.e. the magnitude, hypocenter and fault geometry—are modelled as deterministic in the scenario calculators. Multiple realizations of different possible ground motion fields (GMFs) due to the single rupture are generated, taking into consideration both the inter-event variability of ground motions, and the intra-event residuals obtained from a spatial correlation model for ground motion residuals. The use of logic-trees allows for the consideration of uncertainty in the choice of a GMPE model for the given tectonic region and in the choice of fragility functions for the different taxonomy types in the exposure model.

As an alternative to computing the GMFs with OpenQuake, users can also provide their own sets of GMFs as input to the scenario damage calculator.

For each GMF realization, damage fractions (the fraction of buildings in each damage state) are estimated for every asset in the exposure model using the provided fragility model, and finally the damage distribution statistics (i.e., the mean damage fractions and standard deviation of damage fractions for all damage states) across all realizations are calculated. The calculator also provides aggregated damage distribution statistics for the portfolio, such as mean damage fractions and standard deviation of damage fractions for each taxonomy in the exposure model, and the mean damage fractions and standard deviation of damage fractions for the entire region of study.
\subsection{Classical PSHA-based risk}
   \label{subsec:theory-cr}
   The classical PSHA-based risk calculator convolves through numerical integration, the probabilistic vulnerability functions for an asset with the seismic hazard curve at the location of the asset, to give the loss distribution for the asset within a specified time period. The calculator requires the definition of an exposure model, a vulnerability model for each loss type of interest with vulnerability functions for each taxonomy represented in the exposure model, and hazard curves calculated in the region of interest. Loss curves and loss maps can currently be calculated for five different loss types using this calculator: structural losses, nonstructural losses, contents losses, downtime losses, and occupant fatalities. The main results of this calculator are loss exceedance curves for each asset, which describe the probability of exceedance of different loss levels over the specified time period, and loss maps for the region, which describe the loss values that have a given probability of exceedance over the specified time period.

The hazard curves required for this calculator can be calculated by the OpenQuake engine for all asset locations in the exposure model using the classical PSHA approach \citep{cornell1968, mcguire1976}. The use of logic-trees allows for the consideration of model uncertainty in the choice of a ground motion prediction equation for the different tectonic region types in the region.

Given this set of functionalities, the acceptance testing framework for the classical PSHA-based risk calculator is divided into the following three test sets:

\begin{enumerate}
	\item Single asset test cases
	\item Multiple asset test cases
	\item Logic-tree test cases
\end{enumerate}

The individual test cases are described in more detail in \ref{sec:acc-cr}.
\subsection{Classical PSHA-based damage}
   \label{subsec:theory-cd}
   The classical PSHA-based damage calculator integrates the fragility functions for an asset with the seismic hazard curve at the location of the asset, to give the expected damage distribution for the asset within a specified time period. The calculator requires the definition of an exposure model, a fragility model for each taxonomy represented in the exposure model, and hazard curves calculated in the region of interest. The main results of this calculator are the expected damage distribution for each asset, which describe the probability of the asset being in different damage states, and collapse maps for the region, which describe the probability of collapse for different assets in the portfolio over the specified time period.

The hazard curves required for this calculator can be calculated by the OpenQuake engine for all asset locations in the exposure model using the classical PSHA approach \citep{cornell1968, mcguire1976}. The use of logic-trees allows for the consideration of uncertainty in the choice of a GMPE model for the different tectonic region types in the region and in the choice of vulnerability functions for the different taxonomy types in the exposure model.

Given this set of functionalities, the acceptance testing framework for the classical PSHA-based damage calculator is divided into the following three test sets:

\begin{enumerate}
	\item Single asset test cases
	\item Multiple asset test cases
	\item Logic-tree test cases
\end{enumerate}

The individual test cases are described in more detail in \ref{sec:acc-cd}.

\subsection{Event-based risk}
   \label{subsec:theory-ebr}
   This calculator employs an event-based Monte Carlo simulation approach to probabilistic risk assessment in order to estimate the loss distribution for individual assets and aggregated loss distribution for a spatially distributed portfolio of assets within a specified time period. The calculator requires the definition of an exposure model, a vulnerability model for each loss type of interest with vulnerability functions for each taxonomy represented in the exposure model, and a set of ground motion fields representative of the seismicity of the region over the specified time period. Loss curves and loss maps can currently be calculated for five different loss types using this calculator: structural losses, nonstructural losses, contents losses, downtime losses, and occupant fatalities. The main results of this calculator are loss exceedance curves for each asset, which describe the probability of exceedance of different loss levels over the specified time period, and loss maps for the region, which describe the loss values that have a given probability of exceedance over the specified time period. Aggregate loss exceedance curves can be also be produced using this calculator; these describe the probability of exceedance of different loss levels for all assets of a single taxonomy, or for all assets in the portfolio, over the specified time period. Finally, event loss tables can be produced using this calculator; these tables describe the total loss across the portfolio for each seismic event in the stochastic event set.

This calculator relies on the probabilistic event-based hazard calculator, which simulates the seismicity of the chosen time period $T$ by producing a \textit{stochastic event set} (also known as a \textit{synthetic catalog}). For each rupture generated by a source, the number of occurrences in the given time span $T$ is simulated by sampling the corresponding probability distribution as given by $P_{rup}(k | T)$. A stochastic event set is therefore a \textit{sample} of the full population of ruptures as defined by a seismic source model. Each rupture is present zero, one or more times, depending on its probability. Symbolically, we can define a stochastic event set ($SES$) as:
\begin{align}
SES(T) = \left\{k \times rup,\;k\sim P_{rup}(k | T)\;\;\forall\;rup\;in\;Src\;\forall\;Src\;in\;SSM\right\}
\end{align}
where $k$, the number of occurrences, is a random sample of $P_{rup}(k | T)$, and $k \times rup$ means that rupture $rup$ is repeated $k$ times in the stochastic event set.

For each event in the stochastic event sets, a spatially correlated ground motion field (GMF) realisation is generated, taking into consideration both the inter-event variability of ground motions, and the intra-event residuals obtained from a spatial correlation model for ground motion residuals. The use of logic-trees allows for the consideration of uncertainty in the choice of a seismic source model, in the choice of GMPE models for the different tectonic regions, and in the choice of vulnerability functions for the different taxonomy types in the exposure model.

For each GMF realization, a loss ratio is sampled for every asset in the exposure model using the provided probabilistic vulnerability model, taking into consideration the correlation model for vulnerability of different assets of a given taxonomy. Finally loss exceedance curves are computed for both ground-up losses and insured losses.
   \cleardoublepage