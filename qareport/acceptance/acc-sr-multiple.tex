The multiple asset test cases are designed to test the loss aggregation functions of the scenario risk calculator, such as:

\begin{itemize}
\item portfolio loss computation for a given ground motion field
\item calculation of mean and standard deviation of portfolio scenario loss
\end{itemize}

% ---------------------------------------------------------------------------
\subsubsection{Case 4a}
The purpose of this case is to test the basic elements of a scenario risk calculation involving multiple assets, such as the computation of the mean and standard deviation of the total loss for a portfolio of assets.

\begin{table}

\centering
\begin{tabular}{ c c c c l }

\hline
\rowcolor{anti-flashwhite}
\bf{Site} & \bf{Taxonomy} & \bf{Latitude} & \bf{Longitude} & \bf{Comment} \\
\hline
1 & tax1 & 38.113 & -122.000 & On fault midpoint, along strike \\
2 & tax2 & 38.113 & -122.114 & 10 km west of fault, at midpoint \\
3 & tax1 & 38.113 & -122.570 & 50 km west of fault, at midpoint \\
4 & tax3 & 38.000 & -122.000 & South end of fault \\
5 & tax1 & 37.910 & -122.000 & 10 km south of fault, along strike \\
6 & tax2 & 38.225 & -122.000 & North end of fault \\
7 & tax1 & 38.113 & -121.886 & 10 km east of fault, at midpoint \\
\hline
\end{tabular}

\caption{Asset sites and taxonomies for the multiple-asset, multiple-taxonomy test cases}
\label{tab:assets-tax3}
\end{table}
The list of assets and their taxonomies are shown in Table~\ref{tab:assets-tax3}.

Five precomputed ground motion fields are used as the starting point for this case. These ground motion fields take into consideration both the inter-event and intra-event variability in the ground motion. The ground motion prediction equation used is Boore and Atkinson (2008), and the Jayaram and Baker (2009) model for spatial correlation of ground motion values is applied.

\begin{table}[htbp]

\centering
\begin{tabular}{ c c c c c c c c }

\hline
\rowcolor{anti-flashwhite}
\bf{GMF \#} & \bf{Site 3} & \bf{Site 2} & \bf{Site 5} & \bf{Site 4} & \bf{Site 1} & \bf{Site 6} & \bf{Site 7}\\
\hline
1 & 0.15g & 0.17g & 0.21g & 0.56g & 0.25g & 0.38g & 0.14g \\
2 & 0.05g & 0.21g & 0.18g & 0.69g & 0.94g & 0.72g & 0.43g \\
3 & 0.05g & 0.18g & 0.06g & 0.58g & 0.46g & 0.24g & 0.22g \\
4 & 0.15g & 0.46g & 0.72g & 0.79g & 0.81g & 0.29g & 0.51g \\
5 & 0.15g & 0.48g & 0.95g & 1.70g & 1.70g & 0.63g & 0.25g \\
\hline
\end{tabular}

\caption{Five precomputed spatially correlated ground motion fields (PGA). The sites are sorted first by longitude, then by latitude.}
\label{tab:gmfs-diff-l7-5}
\end{table}

Table~\ref{tab:gmfs-diff-l7-5} lists the ground motion fields used in this test case.


\begin{table}[htbp]

\centering
\begin{tabular}{ l l c c c c c c c c }

\hline
\rowcolor{anti-flashwhite}
\bf{Taxonomy} & \bf{PGA} & \bf{0.05g} & \bf{0.20g} & \bf{0.40g} & \bf{0.60g} & \bf{0.80g} & \bf{1.00g} & \bf{\dots} & \bf{2.00g} \\
\hline
\bf{tax1} & \bf{Mean LR} & 0.01 & 0.04 & 0.10 & 0.20 & 0.33 & 0.50 & \dots & 0.99 \\
    & \bf{CoV LR} & 0.0 & 0.0 & 0.0 & 0.0 & 0.0 & 0.0 & \dots & 0.0 \\
\hline
\bf{tax2} & \bf{Mean LR} & 0.01	& 0.02 & 0.05 & 0.11 & 0.18 & 0.26 & \dots & 0.51 \\
    & \bf{CoV LR} & 0.0 & 0.0 & 0.0 & 0.0 & 0.0 & 0.0 & \dots & 0.0 \\
\hline
\bf{tax3} & \bf{Mean LR} & 0.01 & 0.04 & 0.09 & 0.18 & 0.28 & 0.47 & \dots & 0.91 \\
    & \bf{CoV LR} & 0.0 & 0.0 & 0.0 & 0.0 & 0.0 & 0.0 & \dots & 0.0 \\
\hline
\end{tabular}

\caption{Lognormal vulnerability functions for three building typologies}
\label{tab:vf-ln-tax3-zcov}
\end{table}

Table~\ref{tab:vf-ln-tax3-zcov} shows the mean loss ratios and corresponding coefficients of variation in the vulnerability function used in this test case.

Since there is no variability in the loss ratio, calculation of the loss ratios is straightforward in this case. Consider asset $a3$, which has the taxonomy $tax1$.  The ground motion values at the location of the single asset are $[0.15, 0.05, 0.05, 0.15, 0.15] g$. Consider the first value of $PGA = 0.15 g$. The vulnerability function for this taxonomy provides mean loss ratio values at intensity measure levels $0.05 g$ and $0.20 g$, but none at $0.15 g$. The mean loss ratios at $0.05 g$ and $0.20 g$ are $0.01$ and $0.04$ respectively.

The mean loss ratio at $0.15 g$ is obtained by interpolating between these two values. Linear interpolation gives a mean loss ratio of $0.03$ for $PGA = 0.15 g$.

Similar interpolation for the other ground motion values gives mean loss ratios of $0.01$, $0.01$, $0.03$, and $0.03$ respectively. These numbers are multiplied by the asset value for $a1$ of $10,000$ to give loss values for asset $a1$ of $[300, 100, 100, 300, 300]$.

Repeating this exercise for the six other assets, using the appropriate vulnerability function for each taxonomy, we get the following loss values for the five ground motion fields:

\begin{itemize}
	\item $a1$: $[550, 4490, 1300, 3385, 9300]$
	\item $a2$: $[180, 215, 186.67, 680, 740]$
	\item $a3$: $[1800, 2585, 1900, 3235, 9300]$
	\item $a4$: $[1620, 2250, 1710, 2750, 8200]$
	\item $a5$: $[430, 360, 120, 2780, 4575]$
	\item $a6$: $[470, 1520, 260, 335, 1205]$
	\item $a7$: $[280, 1150, 460, 1550, 550]$
\end{itemize}

The portfolio losses are obtained simply as the sum of all the individual asset losses for each ground motion field. The portfolio losses are: $[5330, 12570, 5936.67, 14715, 33870]$.

Now, the mean and standard deviation of the scenario loss can be calculated for each of the individual assets, as well as for the portfolio.
\begin{table}[htbp]

\centering
\begin{tabular}{ c l r r r }

\hline
\rowcolor{anti-flashwhite}
\bf{Asset} & \bf{Result} & \bf{Expected} & \bf{OpenQuake} & \bf{Difference}\\
\hline
a1 & Mean loss &  &  & \% \\
 & Std. loss &  &  & \% \\
\hline
a2 & Mean loss &  &  & \% \\
 & Std. loss &  &  & \% \\
\hline
a3 & Mean loss &  &  & \% \\
 & Std. loss &  &  & \% \\
\hline
a4 & Mean loss &  &  & \% \\
 & Std. loss &  &  & \% \\
\hline
a5 & Mean loss &  &  & \% \\
 & Std. loss &  &  & \% \\
\hline
a6 & Mean loss &  &  & \% \\
 & Std. loss &  &  & \% \\
\hline
a7 & Mean loss &  &  & \% \\
 & Std. loss &  &  & \% \\
\hline
Total & Mean loss &  &  & \% \\
 & Std. loss &  &  & \% \\
\hline
\end{tabular}

\caption{Results for scenario risk test case 4a}
\label{tab:result-sr-4a}
\end{table}

 The expected values of these statistics are provided in Table~\ref{tab:result-sr-4a}, and the OpenQuake results for the same are also provided in the same table for comparison.

% ---------------------------------------------------------------------------
\subsubsection{Case 4b}
Ten thousand ground motion fields are generated for the given rupture, taking into consideration both the inter-event and intra-event variability in the ground motion. The ground motion prediction equation used is Boore and Atkinson (2008), and the Jayaram and Baker (2009) model for spatial correlation of ground motion values is applied.

\begin{table}

\centering
\begin{tabular}{ c l c c l }

\hline
\rowcolor{anti-flashwhite}
\bf{Site} & \bf{Taxonomy} & \bf{Latitude} & \bf{Longitude} & \bf{Comment} \\
\hline
1 & tax1 & 38.113 & -122.000 & On fault midpoint, along strike \\
2 & tax1 & 38.113 & -122.114 & 10 km west of fault, at midpoint \\
3 & tax1 & 38.113 & -122.570 & 50 km west of fault, at midpoint \\
4 & tax1 & 38.000 & -122.000 & South end of fault \\
5 & tax1 & 37.910 & -122.000 & 10 km south of fault, along strike \\
6 & tax1 & 38.225 & -122.000 & North end of fault \\
7 & tax1 & 38.113 & -121.886 & 10 km east of fault, at midpoint \\
\hline
\end{tabular}

\caption{Asset sites and taxonomies for the multiple-asset, single-taxonomy test cases}
\label{tab:assets-tax1}
\end{table}

The list of assets in the exposure model used for the test cases involving asset correlation (Cases 4b, 4c, and 4d) is given in Table~\ref{tab:assets-tax1}.

\begin{table}[htbp]

\centering
\begin{tabular}{ l c c l }

\hline

\hline
\end{tabular}
\rowcolor{anti-flashwhite}
\bf{GMF \#} & \bf{Site 1} & \bf{Site 2} & \bf{Site 3} & \bf{Site 4} & \bf{Site 5} & \bf{Site 6} & \bf{Site 7}\\
1 & 1.125 & 0.261 & 0.073 & 0.756 & 0.121 & 0.671 & 0.274 \\
2 & 0.386 & 0.198 & 0.058 & 0.660 & 0.396 & 0.529 & 0.297 \\
\vdots & \vdots & \vdots & \vdots & \vdots & \vdots & \vdots & \vdots \\
10,000 & 0.265 & 0.106 & 0.114 & 0.213 & 0.293 & 0.639 & 0.093 \\
\hline
\end{tabular}

\caption{10,000 simulated ground motion fields with spatial correlation}
\label{tab:gmfs-sim-l7-10000}
\end{table}

Table \ref{tab:gmfs-sim-l7-10000} lists five of the ten thousand ground motion fields generated by the OpenQuake scenario hazard calculator.

For the calculation using Julia, one million ground motion fields are similarly generated.
\begin{table}[htbp]

\centering
\begin{tabular}{ c l r r r }

\hline
\rowcolor{anti-flashwhite}
\bf{Asset} & \bf{Result} & \bf{Expected} & \bf{OpenQuake} & \bf{Difference}\\
\hline
a1 & Mean loss &  &  & \% \\
 & Std. loss &  &  & \% \\
\hline
a2 & Mean loss &  &  & \% \\
 & Std. loss &  &  & \% \\
\hline
a3 & Mean loss &  &  & \% \\
 & Std. loss &  &  & \% \\
\hline
a4 & Mean loss &  &  & \% \\
 & Std. loss &  &  & \% \\
\hline
a5 & Mean loss &  &  & \% \\
 & Std. loss &  &  & \% \\
\hline
a6 & Mean loss &  &  & \% \\
 & Std. loss &  &  & \% \\
\hline
a7 & Mean loss &  &  & \% \\
 & Std. loss &  &  & \% \\
\hline
Total & Mean loss &  &  & \% \\
 & Std. loss &  &  & \% \\
\hline
\end{tabular}

\caption{Results for scenario risk test case 4b}
\label{tab:result-scenario-risk-4b}
\end{table}

Table \ref{tab:result-sr-4b} shows the comparison of the OpenQuake result with the expected result.
% ---------------------------------------------------------------------------
\subsubsection{Case 4c}
This case is designed to test the computation of the mean and standard deviation of individual asset losses and also for the portfolio, when the vulnerability models of different assets of the same taxonomy are treated as fully correlated. In OpenQuake, this can be specified in the job configuration file, by setting the value of the parameter `asset\_correlation' to one.

Ten thousand ground motion fields are generated for the given rupture, taking into consideration both the inter-event and intra-event variability in the ground motion. The ground motion prediction equation used is \citet{boore2008}, and the \citet{jayaram2009} model for spatial correlation of ground motion values is applied. These ground motion fields are also used for the corresponding calculation in Julia.

Table~\ref{tab:gmfs-sim-l7-10000} lists five of the ten thousand ground motion fields generated by the OpenQuake scenario hazard calculator. The list of assets in the exposure model used for this case is given in Table~\ref{tab:assets-tax1}. Table~\ref{tab:vf-ln-tax1-nzcov} shows the mean loss ratios and corresponding coefficients of variation in the vulnerability function used in this test case.

Since the sampled loss ratios conditional on a given ground motion field for different assets of the same taxonomy are assumed to be fully correlated in this case, a single \emph{epsilon}, $\epsilon$,  is sampled from the standard normal distribution for each taxonomy. The parameters $m$ and $s$ from the vulnerability model are converted to the parameters $\mu$ and $\sigma$ of the corresponding normal distribution, and the sampled loss ratio is obtained simply as $\exp (\mu + \epsilon * \sigma)$.
\begin{table}[htbp]

\centering
\begin{tabular}{ c l r r r }

\hline
\rowcolor{anti-flashwhite}
\bf{Asset} & \bf{Result} & \bf{Expected} & \bf{OpenQuake} & \bf{Difference}\\
\hline
a1 & Mean loss &  &  & \% \\
 & Std. loss &  &  & \% \\
\hline
a2 & Mean loss &  &  & \% \\
 & Std. loss &  &  & \% \\
\hline
a3 & Mean loss &  &  & \% \\
 & Std. loss &  &  & \% \\
\hline
a4 & Mean loss &  &  & \% \\
 & Std. loss &  &  & \% \\
\hline
a5 & Mean loss &  &  & \% \\
 & Std. loss &  &  & \% \\
\hline
a6 & Mean loss &  &  & \% \\
 & Std. loss &  &  & \% \\
\hline
a7 & Mean loss &  &  & \% \\
 & Std. loss &  &  & \% \\
\hline
Total & Mean loss &  &  & \% \\
 & Std. loss &  &  & \% \\
\hline
\end{tabular}

\caption{Results for scenario risk test case 4c}
\label{tab:result-scenario-risk-4c}
\end{table}

Table \ref{tab:result-sr-4c} shows the comparison of the OpenQuake result with the expected result.

% ---------------------------------------------------------------------------
\subsubsection{Case 4d}
This case is designed to test the computation of the mean and standard deviation of individual asset losses and also for the portfolio, when the vulnerability models of different assets of the same taxonomy are treated as partially correlated, with a correlation coefficient of 0.5. In OpenQuake, this can be specified in the job configuration file, by setting the value of the parameter `asset\_correlation' to 0.5.

As in the previous cases for zero and full correlation, ten thousand ground motion fields are generated for the given rupture, and these ground motion fields are also used for the corresponding calculation in Julia. The list of assets in the exposure model used for this case is given in Table~\ref{tab:assets-tax1}. Table~\ref{tab:vf-ln-tax1-nzcov} shows the mean loss ratios and corresponding coefficients of variation in the vulnerability function used in this test case.

Since the sampled loss ratios conditional on a given ground motion field for different assets of the same taxonomy are assumed to be  correlated in this case, we proceed by first generating a vector of \emph{epsilons} for each taxonomy from the multivariate standard normal distribution which has the symmetric covariance matrix with $1.0$ as the diagonal elements and $\rho = 0.5$ as the off-diagonal elements.

Now, for each asset of that taxonomy, the parameters $m$ and $s$ are obtained for the ground motion value at the location of the asset through interpolation on the specified vulnerability model. Each asset of a particular taxonomy is also assigned a value of $\epsilon$ from the vector of \emph{epsilons} for that taxonomy sampled as described above. The parameters $m$ and $s$ are then converted to the parameters $\mu$ and $\sigma$ of the corresponding normal distribution, and the sampled loss ratio is obtained simply as $\exp (\mu + \epsilon * \sigma)$.

After the asset event loss tables are compiled by sampling correlated loss values as described above, the rest of the calculation concerning the derivation of loss curves and average loss proceeds as in previous cases.
\begin{table}[htbp]

\centering
\begin{tabular}{ c l r r r }

\hline
\rowcolor{anti-flashwhite}
\bf{Asset} & \bf{Result} & \bf{Julia} & \bf{OpenQuake} & \bf{Difference}\\
\hline
a1 & Mean loss &  &  & \% \\
 & Std. loss &  &  & \% \\
\hline
a2 & Mean loss &  &  & \% \\
 & Std. loss &  &  & \% \\
\hline
a3 & Mean loss &  &  & \% \\
 & Std. loss &  &  & \% \\
\hline
a4 & Mean loss &  &  & \% \\
 & Std. loss &  &  & \% \\
\hline
a5 & Mean loss &  &  & \% \\
 & Std. loss &  &  & \% \\
\hline
a6 & Mean loss &  &  & \% \\
 & Std. loss &  &  & \% \\
\hline
a7 & Mean loss &  &  & \% \\
 & Std. loss &  &  & \% \\
\hline
Total & Mean loss &  &  & \% \\
 & Std. loss &  &  & \% \\
\hline
\end{tabular}

\caption{Results for scenario risk test case 4d}
\label{tab:result-sr-4d}
\end{table}

Table \ref{tab:result-sr-4d} shows the comparison of the OpenQuake result with the expected result.

% ---------------------------------------------------------------------------