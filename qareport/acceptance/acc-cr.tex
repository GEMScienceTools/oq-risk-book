The tests for the classical PSHA-based risk calculator assume the correct computation of the hazard curves at the locations of the assets in the exposure model. Thus, the risk tests implicitly rely on the acceptance tests for the classical PSHA-based hazard calculator.

The source model used for the tests comprises a single vertical strike-slip fault with a Gutenberg-Richter b-value equal to 0.9 and a slip rate of 2 mm/yr. The MFD is a Gutenberg-Richter distribution truncated between magnitudes 5.0 and 6.5, while the Ground Motion Prediction Equation (GMPE) used is \citet{boore2008}.\\

\noindent Details of the fault geometry are given below:\\

\noindent
Fault type: Strike slip\\
Fault dip: $90^{\circ}$\\
Fault plane depths: 0--12 km\\
Fault coordinates:\\
South end: $38.0000^{\circ} N$, $122.0000^{\circ} W$\\
North end: $38.2248^{\circ} N$, $122.0000^{\circ} W$\\

Figure~\ref{fig:fault-sites}, shows the fault described above and the site geometry for the test cases described in the following sections. The single asset tests use only site $1$ shown in the figure, whereas the multiple asset tests use all seven of the sites. The geometry of the fault and locations of the sites match those selected by \citet{thomas2010} in their effort to verify PSHA computer programs.

\begin{figure}[htbp]
\centering
\includegraphics[width=12cm]{qareport/figures/fig-fault-sites}
\caption{Fault and site geometry for the classical risk tests. From \citet{thomas2010}}
\label{fig:fault-sites}
\end{figure}

The complete collection of input models and job configuration files used in these test cases can be accessed here:
\href{https://github.com/gem/oq-risklib/tree/master/openquake/qa_tests_data/classical_risk}
{https://github.com/gem/oq-risklib/tree/master/openquake/qa\_tests\_data/classical\_risk}

\subsection{Single asset tests}
\label{subsec:acc-cr-single}
The single asset test cases are designed to test the basic elements of the classical-PSHA based risk calculator, such as:

\begin{itemize}
\item asset loss ratio exceedance curve computation
\item asset loss exceedance curve computation
\end{itemize}

The location and taxonomy of the single asset in the exposure model used for the single-asset test cases for the classical risk calculator are given in Table~\ref{tab:asset}.

% ---------------------------------------------------------------------------
\subsubsection{Case 1a}
Table~\ref{tab:vf-ln-tax1-zcov} shows the mean loss ratios and corresponding coefficients of variation for the vulnerability function used in this test case.

When the exposure model and vulnerability model are provided to the OpenQuake classical PSHA-based hazard calculator, OpenQuake computes the hazard curves at the locations of the assets in the exposure model and at the specific intensity levels used in the vulnerability functions.

\input{qareport/tables/tab-hc-l1}

The intensity levels for the hazard curve are extracted from the vulnerability function: $[0.05, 0.20, 0.40, 0.60, 0.80, 1.00, 1.20, 1.40, 1.60, 1.80, 2.00]$. The hazard curve gives the probabilities of exceedance for a set of intensity levels within a specified time period. The time period in this case, $t_H$, is one year. The hazard curve at the location of the single asset used in this test case is shown in Table~\ref{tab:hc-l1}.

The probabilities of exceedance are: $[3.896\times10^{-2}, 2.222\times10^{-2}, 8.171\times10^{-3}, 3.070\times10^{-3}, 1.230\times10^{-3}, 5.195\times10^{-4}, 2.254\times10^{-4}, 9.918\times10^{-5}, 4.353\times10^{-5}, 1.830\times10^{-5}, 6.925\times10^{-6}]$. The probabilities of exceedance are first converted to annual rates (or frequencies) of exceedance by employing the Poissonion conversion:

\begin{equation}
	\lambda(iml) = \frac{-\ln [1 - prob(IML > iml, t_H)]}{t_H}
\end{equation}

The annual frequencies of exceedance are: $[3.974\times10^{2}, 2.247\times10^{2}, 8.205\times10^{3}, 3.075\times10^{3}, 1.231\times10^{3}, 5.197\times10^{4}, 2.254\times10^{4}, 9.918\times10^{5}, 4.353\times10^{5}, 1.829\times10^{5}, 6.925\times10^{6}]$.

The annual frequencies of occurrence are estimated by the differentiation of the annual frequencies of exceedance: $[1.727\times10^{2}, 1.426\times10^{2}, 5.130\times10^{3}, 1.845\times10^{3}, 7.109\times10^{4}, 2.942\times10^{4}, 1.262\times10^{4}, 5.565\times10^{5}, 2.524\times10^{5}, 1.137\times10^{5}]$.

The loss ratios at which the loss curve exceedance probabilities are calculated are obtained from the vulnerability function and the parameter `steps\_per\_interval'. The default value of `steps\_per\_interval' is one, which is the value used in this case. The loss ratios in the vulnerability function are $[0.01, 0.04, 0.10, 0.20, 0.33, 0.50, 0.67, 0.80, 0.90, 0.96, 0.99]$.

The vulnerability model is then transformed into a matrix describing probabilities of exceedance for the selected set of loss ratios conditional on the set of ground motion intensity levels. Since there is no variability in the loss ratio, calculation of the loss curves is straightforward in this case. Since the coefficients of variation in the vulnerability function are all zero, the lognormal distribution devolves into the degenerate distribution. The loss ratio exceedance matrix in this case is shown in Table~\ref{tab:lrem-ln-tax1-zcov}.

\begin{table}[htbp]

\centering
\begin{tabular}{ l c c c c c c c c c }

\hline
\rowcolor{anti-flashwhite}
\bf{LR | PGA} & \bf{0.05g} & \bf{0.20g} & \bf{0.40g} & \bf{0.60g} & \bf{0.80g} & \bf{1.00g} & \bf{1.20g} & \bf{\dots} & \bf{2.00g} \\
\hline
\bf{0.01} & 1 & 1 & 1 & 1 & 1 & 1 & 1 & \dots & 1 \\
\bf{0.04} & 0 & 1 & 1 & 1 & 1 & 1 & 1 & \dots & 1 \\
\bf{0.10} & 0 & 0 & 1 & 1 & 1 & 1 & 1 & \dots & 1 \\
\bf{0.20} & 0 & 0 & 0 & 1 & 1 & 1 & 1 & \dots & 1 \\
\bf{0.33} & 0 & 0 & 0 & 0 & 1 & 1 & 1 & \dots & 1 \\
\bf{0.50} & 0 & 0 & 0 & 0 & 0 & 1 & 1 & \dots & 1 \\
\bf{0.67} & 0 & 0 & 0 & 0 & 0 & 0 & 1 & \dots & 1 \\
\bf{0.80} & 0 & 0 & 0 & 0 & 0 & 0 & 0 & \dots & 1 \\
\bf{0.90} & 0 & 0 & 0 & 0 & 0 & 0 & 0 & \dots & 1 \\
\bf{0.96} & 0 & 0 & 0 & 0 & 0 & 0 & 0 & \dots & 1 \\
\bf{0.99} & 0 & 0 & 0 & 0 & 0 & 0 & 0 & \dots & 1 \\
\bf{1.00} & 0 & 0 & 0 & 0 & 0 & 0 & 0 & \dots & 0 \\
\hline
\end{tabular}

\caption{Conditional loss ratio exceedance matrix for classical risk test case 1a}
\label{tab:lrem-ln-tax1-zcov}
\end{table}

Now, the sum product of each row of the conditional loss ratio exceedance matrix with the annual frequencies of occurrence of the respective intensity levels gives the annual frequency of exceedance for the respective loss ratios. The loss ratio annual frequencies of exceedance thus calculated are: $[3.973\times10^{2}, 3.110\times10^{2}, 1.533\times10^{2}, 5.633\times10^{3}, 2.146\times10^{3}, 8.682\times10^{4}, 3.656\times10^{4}, 1.554\times10^{4}, 6.443\times10^{5}, 2.399\times10^{5}]$.

The probabilities of exceedance of the set of loss ratios are obtained by converting the frequencies of exceedance back into probabilities by using the Poissonion assumption. The loss curve probabilities of exceedance are: $[3.895\times10^{2}, 3.062\times10^{2}, 1.521\times10^{2}, 5.617\times10^{3}, 2.144\times10^{3}, 8.678\times10^{4}, 3.655\times10^{4}, 1.554\times10^{4}, 6.443\times10^{5}, 2.399\times10^{5}, 5.683\times10^{6}]$.

The loss curve thus calculated above is compared with the loss curve obtained using the OpenQuake classical PSHA based risk calculator in Figure~\ref{fig:lc-cr-1a}.

\begin{figure}[htbp]
\centering
\includegraphics[width=12cm]{qareport/figures/fig-lc-cr-1a}
\caption{Loss curve comparison for classical risk test case 1a}
\label{fig:lc-cr-1a}
\end{figure}

The area under the annual loss exceedance curve gives the average annual loss.

\begin{table}[htbp]

\centering
\begin{tabular}{ l r r r }

\hline
\rowcolor{anti-flashwhite}
\bf{Result} & \bf{Expected} & \bf{OpenQuake} & \bf{Difference}\\
\hline
Average loss & 47.63 & 47.63 & 0.00\% \\
\hline
\end{tabular}

\caption{Results for classical risk test case 1a}
\label{tab:result-cr-1a}
\end{table}
Table~\ref{tab:result-cr-1a} shows the comparison of the OpenQuake result for average annual loss with the expected result.

% % ---------------------------------------------------------------------------
\subsubsection{Case 1b}
\begin{table}[htbp]

\centering
\begin{tabular}{ l r r r }

\hline
\rowcolor{anti-flashwhite}
\bf{Result} & \bf{Expected} & \bf{OpenQuake} & \bf{Difference}\\
\hline
Average loss & 45.55 & 45.55 & 0\% \\
\hline
\end{tabular}

\caption{Results for classical risk test case 1b}
\label{tab:result-cr-1b}
\end{table}
\begin{table}[htbp]

\centering
\begin{tabular}{ l r r r }

\hline
\rowcolor{anti-flashwhite}
\bf{Result} & \bf{Expected} & \bf{OpenQuake} & \bf{Difference}\\
\hline
Average loss & 45.55 & 45.55 & 0\% \\
\hline
\end{tabular}

\caption{Results for classical risk test case 1b}
\label{tab:result-cr-1b}
\end{table}
Table~\ref{tab:result-cr-1b} shows the comparison of the OpenQuake result for average annual loss with the expected result.

% % % ---------------------------------------------------------------------------
\subsubsection{Case 1c}
This test case repeats the exercise from Case~1a using a vulnerability model with nonzero coefficients of variation. Table~\ref{tab:vf-ln-tax1-nzcov} shows the mean loss ratios and corresponding coefficients of variation for the vulnerability function used in this test case.

Apart from the computation of the conditional loss ratio exceedance matrix, the steps for computing the loss exceedance curve in this case are the same as those employed in Case~1a. The conditional loss ratio exceedance matrix in this case is populated by evaluating the complementary cumulative distribution function (CCDF) of the lognormal distribution at each of the prescribed intensity levels, for the set of loss ratios.

\begin{table}[htbp]

\centering
\begin{tabular}{ l c c c c c c c c c }

\hline
\rowcolor{anti-flashwhite}
\bf{LR | PGA} & \bf{0.05g} & \bf{0.20g} & \bf{0.40g} & \bf{0.60g} & \bf{0.80g} & \bf{1.00g} & \bf{1.20g} & \bf{\dots} & \bf{2.00g} \\
\hline
\bf{0.01} & 0.494 & 1.000 & 1.000 & 1.000 & 1.000 & 1.000 & 1.000 & \dots & 1.000 \\
\bf{0.04} & 0.000 & 0.476 & 1.000 & 1.000 & 1.000 & 1.000 & 1.000 & \dots & 1.000 \\
\bf{0.10} & 0.000 & 0.000 & 0.453 & 0.980 & 0.999 & 1.000 & 1.000 & \dots & 1.000 \\
\bf{0.20} & 0.000 & 0.000 & 0.001 & 0.438 & 0.881 & 0.986 & 0.999 & \dots & 1.000 \\
\bf{0.33} & 0.000 & 0.000 & 0.000 & 0.039 & 0.427 & 0.812 & 0.959 & \dots & 1.000 \\
\bf{0.50} & 0.000 & 0.000 & 0.000 & 0.001 & 0.094 & 0.424 & 0.730 & \dots & 1.000 \\
\bf{0.67} & 0.000 & 0.000 & 0.000 & 0.000 & 0.017 & 0.170 & 0.427 & \dots & 1.000 \\
\bf{0.80} & 0.000 & 0.000 & 0.000 & 0.000 & 0.005 & 0.079 & 0.253 & \dots & 1.000 \\
\bf{0.90} & 0.000 & 0.000 & 0.000 & 0.000 & 0.002 & 0.043 & 0.162 & \dots & 0.999 \\
\bf{0.96} & 0.000 & 0.000 & 0.000 & 0.000 & 0.001 & 0.030 & 0.122 & \dots & 0.844 \\
\bf{0.99} & 0.000 & 0.000 & 0.000 & 0.000 & 0.001 & 0.025 & 0.106 & \dots & 0.494 \\
\bf{1.00} & 0.000 & 0.000 & 0.000 & 0.000 & 0.001 & 0.023 & 0.101 & \dots & 0.363 \\
\hline
\end{tabular}

\caption{Conditional loss ratio exceedance matrix for classical risk test case 1c}
\label{tab:lrem-ln-tax1-nzcov}
\end{table}

The loss ratio exceedance matrix in this case is shown in Table~\ref{tab:lrem-ln-tax1-nzcov}.

The loss curve thus calculated above is compared with the loss curve obtained using the OpenQuake classical PSHA based risk calculator in Figure~\ref{fig:lc-cr-1c}.

\begin{figure}[htbp]
\centering
\includegraphics[width=12cm]{qareport/figures/fig-lc-cr-1c}
\caption{Loss curve comparison for classical risk test case 1c}
\label{fig:lc-cr-1c}
\end{figure}

The area under the annual loss exceedance curve gives the average annual loss.
\begin{table}[htbp]

\centering
\begin{tabular}{ l r r r }

\hline
\rowcolor{anti-flashwhite}
\bf{Result} & \bf{Expected} & \bf{OpenQuake} & \bf{Difference}\\
\hline
Average loss & 43.73 & 43.73 & 0\% \\
\hline
\end{tabular}

\caption{Results for classical risk test case 1c}
\label{tab:result-cr-1c}
\end{table}
Table~\ref{tab:result-cr-1c} shows the comparison of the OpenQuake result for average annual loss with the expected result.

% % % ---------------------------------------------------------------------------
\subsubsection{Case 1d}
This test case is identical to Case~1c described above, except for the use of the Beta distribution for the vulnerability functions instead of the lognormal distribution. The conditional loss ratio exceedance matrix in this case is populated by evaluating the complementary cumulative distribution function (CCDF) of the Beta distribution at each of the prescribed intensity levels, for the set of loss ratios.

\begin{table}[htbp]

\centering
\begin{tabular}{ l c c c c c c c c c }

\hline
\rowcolor{anti-flashwhite}
\bf{LR | PGA} & \bf{0.05g} & \bf{0.20g} & \bf{0.40g} & \bf{0.60g} & \bf{0.80g} & \bf{1.00g} & \bf{1.20g} & \bf{\dots} & \bf{2.00g} \\
\hline
\bf{0.01} & 0.496 & 1.000 & 1.000 & 1.000 & 1.000 & 1.000 & 1.000 & \dots & 1.000 \\
\bf{0.04} & 0.000 & 0.485 & 0.999 & 1.000 & 1.000 & 0.999 & 0.996 & \dots & 1.000 \\
\bf{0.10} & 0.000 & 0.000 & 0.472 & 0.959 & 0.984 & 0.987 & 0.982 & \dots & 1.000 \\
\bf{0.20} & 0.000 & 0.000 & 0.000 & 0.468 & 0.844 & 0.928 & 0.944 & \dots & 1.000 \\
\bf{0.33} & 0.000 & 0.000 & 0.000 & 0.032 & 0.473 & 0.778 & 0.871 & \dots & 1.000 \\
\bf{0.50} & 0.000 & 0.000 & 0.000 & 0.000 & 0.100 & 0.500 & 0.738 & \dots & 1.000 \\
\bf{0.67} & 0.000 & 0.000 & 0.000 & 0.000 & 0.006 & 0.222 & 0.563 & \dots & 0.999 \\
\bf{0.80} & 0.000 & 0.000 & 0.000 & 0.000 & 0.000 & 0.072 & 0.394 & \dots & 0.995 \\
\bf{0.90} & 0.000 & 0.000 & 0.000 & 0.000 & 0.000 & 0.013 & 0.234 & \dots & 0.976 \\
\bf{0.96} & 0.000 & 0.000 & 0.000 & 0.000 & 0.000 & 0.001 & 0.115 & \dots & 0.925 \\
\bf{0.99} & 0.000 & 0.000 & 0.000 & 0.000 & 0.000 & 0.000 & 0.038 & \dots & 0.822 \\
\bf{1.00} & 0.000 & 0.000 & 0.000 & 0.000 & 0.000 & 0.000 & 0.000 & \dots & 0.000 \\
\hline
\end{tabular}

\caption{Conditional loss ratio exceedance matrix for classical risk test case 1d}
\label{tab:lrem-bt-tax1-nzcov}
\end{table}

The loss ratio exceedance matrix in this case is shown in Table~\ref{tab:lrem-bt-tax1-nzcov}.
This test case is identical to Case~1c described above, except for the use of the Beta distribution for the vulnerability functions instead of the lognormal distribution. The conditional loss ratio exceedance matrix in this case is populated by evaluating the complementary cumulative distribution function (CCDF) of the Beta distribution at each of the prescribed intensity levels, for the set of loss ratios.

\begin{table}[htbp]

\centering
\begin{tabular}{ l c c c c c c c c c }

\hline
\rowcolor{anti-flashwhite}
\bf{LR | PGA} & \bf{0.05g} & \bf{0.20g} & \bf{0.40g} & \bf{0.60g} & \bf{0.80g} & \bf{1.00g} & \bf{1.20g} & \bf{\dots} & \bf{2.00g} \\
\hline
\bf{0.01} & 0.496 & 1.000 & 1.000 & 1.000 & 1.000 & 1.000 & 1.000 & \dots & 1.000 \\
\bf{0.04} & 0.000 & 0.485 & 0.999 & 1.000 & 1.000 & 0.999 & 0.996 & \dots & 1.000 \\
\bf{0.10} & 0.000 & 0.000 & 0.472 & 0.959 & 0.984 & 0.987 & 0.982 & \dots & 1.000 \\
\bf{0.20} & 0.000 & 0.000 & 0.000 & 0.468 & 0.844 & 0.928 & 0.944 & \dots & 1.000 \\
\bf{0.33} & 0.000 & 0.000 & 0.000 & 0.032 & 0.473 & 0.778 & 0.871 & \dots & 1.000 \\
\bf{0.50} & 0.000 & 0.000 & 0.000 & 0.000 & 0.100 & 0.500 & 0.738 & \dots & 1.000 \\
\bf{0.67} & 0.000 & 0.000 & 0.000 & 0.000 & 0.006 & 0.222 & 0.563 & \dots & 0.999 \\
\bf{0.80} & 0.000 & 0.000 & 0.000 & 0.000 & 0.000 & 0.072 & 0.394 & \dots & 0.995 \\
\bf{0.90} & 0.000 & 0.000 & 0.000 & 0.000 & 0.000 & 0.013 & 0.234 & \dots & 0.976 \\
\bf{0.96} & 0.000 & 0.000 & 0.000 & 0.000 & 0.000 & 0.001 & 0.115 & \dots & 0.925 \\
\bf{0.99} & 0.000 & 0.000 & 0.000 & 0.000 & 0.000 & 0.000 & 0.038 & \dots & 0.822 \\
\bf{1.00} & 0.000 & 0.000 & 0.000 & 0.000 & 0.000 & 0.000 & 0.000 & \dots & 0.000 \\
\hline
\end{tabular}

\caption{Conditional loss ratio exceedance matrix for classical risk test case 1d}
\label{tab:lrem-bt-tax1-nzcov}
\end{table}

The loss ratio exceedance matrix in this case is shown in Table~\ref{tab:lrem-bt-tax1-nzcov}.
Table~\ref{tab:result-cr-1d} shows the comparison of the OpenQuake result for average annual loss with the expected result.

% % % ---------------------------------------------------------------------------
\subsubsection{Case 1e}
This test case repeats the exercise from Case~1c using a vulnerability model with nonzero coefficients of variation, and using four `steps\_per\_interval'. Each interval between the loss ratios specified in the vulnerability model is further divided into four equal subdivisions, thus ensuring a greater number of loss values at which the exceedance curve will be computed. For instance, the interval between the loss ratios $[0.10, 0.20]$ is now subdivided into the following loss ratios: $[0.100, 0.125, 0.150, 0.175, 0.200]$.

The loss curve thus calculated above is compared with the loss curve obtained using the OpenQuake classical PSHA based risk calculator in Figure~\ref{fig:lc-cr-1e}.

\begin{figure}[htbp]
\centering
\includegraphics[width=12cm]{qareport/figures/fig-lc-cr-1e}
\caption{Loss curve comparison for classical risk test case 1e}
\label{fig:lc-cr-1e}
\end{figure}

The area under the annual loss exceedance curve gives the average annual loss.
\begin{table}[htbp]

\centering
\begin{tabular}{ l r r r }

\hline
\rowcolor{anti-flashwhite}
\bf{Result} & \bf{Expected} & \bf{OpenQuake} & \bf{Difference}\\
\hline
Average loss & 33.25 & 33.25 & 0.00\% \\
\hline
\end{tabular}

\caption{Results for classical risk test case 1e}
\label{tab:result-cr-1e}
\end{table}
Table~\ref{tab:result-cr-1e} shows the comparison of the OpenQuake result for average annual loss with the expected result.

% % % ---------------------------------------------------------------------------
\subsubsection{Case 2a}
In addition to computing direct structural losses, OpenQuake also provides support for computing losses incurred for the following other loss types:

\begin{itemize}
\item{Non-structural losses}
\item{Contents losses}
\item{Downtime, or business interruption losses}
\item{Occupant fatalities}
\end{itemize}

The purpose of this case is to test the calculation of the loss exceedance curve and average annual loss for the non-structural components of an asset. The replacement value of the non-structural components for the asset used in this case is $15,000$. Table~\ref{tab:vf-ln-tax1-nst} shows the mean loss ratios and corresponding coefficients of variation for the non-structural components vulnerability model used in this test case.

\begin{table}[htbp]

\centering
\begin{tabular}{ l r r r }

\hline
\rowcolor{anti-flashwhite}
\bf{Result} & \bf{Expected} & \bf{OpenQuake} & \bf{Difference}\\
\hline
Average loss & 63.48 & 63.48 & 0\% \\
\hline
\end{tabular}

\caption{Results for classical risk test case 2a}
\label{tab:result-cr-2a}
\end{table}
Table \ref{tab:result-cr-1b} shows the comparison of the OpenQuake result for average annual nonstructural loss with the expected result.

% % ---------------------------------------------------------------------------
\subsubsection{Case 2b}
The purpose of this case is to test the calculation of loss exceedance curve and average annual loss for the contents of an asset. The replacement value of the contents for the asset used in this case is $5,000$. Table~\ref{tab:vf-ln-tax1-con} shows the mean loss ratios and corresponding coefficients of variation in the contents vulnerability function used in this test case.
\begin{table}[htbp]

\centering
\begin{tabular}{ l r r r }

\hline
\rowcolor{anti-flashwhite}
\bf{Result} & \bf{Expected} & \bf{OpenQuake} & \bf{Difference}\\
\hline
Average loss & 59.60 & 59.60 & 0\% \\
\hline
\end{tabular}

\caption{Results for classical risk test case 2b}
\label{tab:result-classical-risk-2b}
\end{table}
Table \ref{tab:result-cr-2b} shows the comparison of the OpenQuake result for average annual contents loss with the expected result.

% % ---------------------------------------------------------------------------
\subsubsection{Case 2c}
\begin{table}[htbp]

\centering
\begin{tabular}{ l r r r }

\hline
\rowcolor{anti-flashwhite}
\bf{Result} & \bf{Expected} & \bf{OpenQuake} & \bf{Difference}\\
\hline
Average loss & 7.02 & 7.02 & 0\% \\
\hline
\end{tabular}

\caption{Results for classical risk test case 2c}
\label{tab:result-cr-2c}
\end{table}
\begin{table}[htbp]

\centering
\begin{tabular}{ l r r r }

\hline
\rowcolor{anti-flashwhite}
\bf{Result} & \bf{Expected} & \bf{OpenQuake} & \bf{Difference}\\
\hline
Average loss & 7.02 & 7.02 & 0\% \\
\hline
\end{tabular}

\caption{Results for classical risk test case 2c}
\label{tab:result-cr-2c}
\end{table}
Table \ref{tab:result-cr-2c} shows the comparison of the OpenQuake result for average annual downtime loss with the expected result.

% % ---------------------------------------------------------------------------
\subsubsection{Case 2d}
The purpose of this case is to test the calculation of the exceedance curve for fatalities and the average annual occupant fatalities for an asset. The number of occupants for the asset used in this case are 2 (day), 4 (transit), and 6 (night). An average value of 4 occupants is used for the calculation of the exceedance curve and average annual fatalities. Table~\ref{tab:vf-ln-tax1-dnt} shows the mean loss ratios and corresponding coefficients of variation for the occupants fatality vulnerability function used in this test case.
\begin{table}[htbp]

\centering
\begin{tabular}{ l r r r }

\hline
\rowcolor{anti-flashwhite}
\bf{Result} & \bf{Expected} & \bf{OpenQuake} & \bf{Difference}\\
\hline
Average loss & $8.78 \times 10^{-4}$ & $8.78 \times 10^{-4}$ & 0\% \\
\hline
\end{tabular}

\caption{Results for classical risk test case 2d}
\label{tab:result-cr-2d}
\end{table}
Table \ref{tab:result-cr-2d} shows the comparison of the OpenQuake result for average annual fatalities with the expected result.

% % ---------------------------------------------------------------------------
\subsubsection{Case 3a}
OpenQuake allows the time period for which the loss exceedance probabilities are calculated to be different from the time period associated with the hazard curve calculation. The hazard curve is calculated for a time period of 50 years, and the loss curve is calculated for a time period of 75 years.

Table~\ref{tab:vf-ln-tax1-nzcov} shows the mean loss ratios and corresponding coefficients of variation for the vulnerability function used in this test case.

\begin{table}[htbp]

\centering
\tabcolsep=0.11cm
\scalebox{0.6}{

\begin{tabular}{ l c c c c c c c c c c c }

\hline
\rowcolor{anti-flashwhite}
\bf{PGA} & \bf{0.05g} & \bf{0.20g} & \bf{0.40g} & \bf{0.60g} & \bf{0.80g} & \bf{1.00g} & \bf{1.20g} & \bf{1.40g} & \bf{1.60g} & \bf{1.80g} & \bf{2.00g} \\
\hline
\bf{P.O.E.} & 8.643\times10^{-1} & 7.171\times10^{-1} & 4.371\times10^{-1} & 2.364\times10^{-1} & 1.234\times10^{-1} & 6.427\times10^{-2} & 3.382\times10^{-2} & 1.802\times10^{-2} & 9.676\times10^{-3} & 5.192\times10^{-3} & 2.748\times10^{-3}\\
\hline
\end{tabular}

}

\caption{50-year hazard curve for PGA at a single site}
\label{tab:hc-l1-50}
\end{table}

The intensity levels for the hazard curve are extracted from the vulnerability function: $[0.05, 0.20, 0.40, 0.60, 0.80, 1.00, 1.20, 1.40, 1.60, 1.80, 2.00]$. The hazard curve gives the probabilities of exceedance for a set of intensity levels within a specified time period. The time period in this case, $t_H$, is fifty years. The hazard curve at the location of the single asset used in this test case is shown in Table~\ref{tab:hc-l1-50}.

The probabilities of exceedance are: $[8.643\times10^{-1}, 7.171\times10^{-1}, 4.371\times10^{-1}, 2.364\times10^{-1}, 1.234\times10^{-1}, 6.427\times10^{-2}, 3.382\times10^{-2}, 1.802\times10^{-2}, 9.676\times10^{-3}, 5.192\times10^{-3}, 2.748\times10^{-3}]$. The probabilities of exceedance are first converted to annual rates (or frequencies) of exceedance by employing the Poissonion conversion:

\begin{equation}
	\lambda(iml) = \frac{-\ln [1 - prob(IML > iml, t_H)]}{t_H}
\end{equation}

The annual frequencies of exceedance are: $[3.994\times10^{-2}, 2.525\times10^{-2}, 1.149\times10^{-2}, 5.394\times10^{-3}, 2.633\times10^{-3}, 1.329\times10^{-3}, 6.881\times10^{-4}, 3.636\times10^{-4}, 1.945\times10^{-4}, 1.041\times10^{-4}, 5.504\times10^{-5}]$.

The annual frequencies of occurrence are estimated by the differentiation of the annual frequencies of exceedance: $[1.469\times10^{-2}, 1.376\times10^{-2}, 6.101\times10^{-3}, 2.760\times10^{-3}, 1.305\times10^{-3}, 6.404\times10^{-4}, 3.245\times10^{-4}, 1.692\times10^{-4}, 9.035\times10^{-5}, 4.907\times10^{-5}]$.

The loss ratios at which the loss curve exceedance probabilities are calculated are obtained from the vulnerability function and the parameter `steps\_per\_interval'. The default value of `steps\_per\_interval' is one, which is the value used in this case. The loss ratios in the vulnerability function are $[0.01, 0.04, 0.10, 0.20, 0.33, 0.50, 0.67, 0.80, 0.90, 0.96, 0.99]$.

The vulnerability model is then transformed into a matrix describing probabilities of exceedance for the selected set of loss ratios conditional on the set of ground motion intensity levels. Since there is no variability in the loss ratio, calculation of the loss curves is straightforward in this case. Since the coefficients of variation in the vulnerability function are all zero, the lognormal distribution devolves into the degenerate distribution. The loss ratio exceedance matrix in this case is shown in Table~\ref{tab:lrem-ln-tax1-nzcov-75}.

\input{qareport/tables/tab-lrem-ln-tax1-nzcov-75}

Now, the sum product of each row of the conditional loss ratio exceedance matrix with the annual frequencies of occurrence of the respective intensity levels gives the annual frequency of exceedance for the respective loss ratios. The loss ratio annual frequencies of exceedance thus calculated are: $[3.988\times10^{-2}, 3.617\times10^{-2}, 2.509\times10^{-2}, 1.280\times10^{-2}, 5.654\times10^{-3}, 2.765\times10^{-3}, 1.408\times10^{-3}, 7.772\times10^{-4}, 4.896\times10^{-4}, 3.276\times10^{-4}, 2.457\times10^{-4}, 2.037\times10^{-4}, 1.903\times10^{-4}]$.

The probabilities of exceedance of the set of loss ratios are obtained by converting the annual frequencies of exceedance back into probabilities of exceedance over 75 years by using the Poissonion equation. The loss curve probabilities of exceedance for a time period of 75 years are: $[9.498\times10^{-1}, 9.336\times10^{-1}, 8.477\times10^{-1}, 6.170\times10^{-1}, 3.456\times10^{-1}, 1.873\times10^{-1}, 1.002\times10^{-1}, 5.663\times10^{-2}, 3.606\times10^{-2}, 2.427\times10^{-2}, 1.826\times10^{-2}, 1.516\times10^{-2}, 1.417\times10^{-2}]$.

The loss curve thus calculated above is compared with the loss curve obtained using the OpenQuake classical PSHA based risk calculator in Figure~\ref{fig:lc-cr-3a}.

\begin{figure}[htbp]
\centering
\includegraphics[width=12cm]{qareport/figures/fig-lc-cr-3a}
\caption{Loss curve comparison for classical risk test case 3a}
\label{fig:lc-cr-3a}
\end{figure}

The area under the loss exceedance curve gives the expected loss over 75 years.

\begin{table}[htbp]

\centering
\begin{tabular}{ l r r r }

\hline
\rowcolor{anti-flashwhite}
\bf{Result} & \bf{Expected} & \bf{OpenQuake} & \bf{Difference}\\
\hline
Average loss & 2,115.81 & 2,115.81 & 0.00\% \\
\hline
\end{tabular}

\caption{Results for classical risk test case 3a}
\label{tab:result-cr-3a}
\end{table}
Table \ref{tab:result-cr-3a} shows the comparison of the OpenQuake result for expected loss over 75 years with the expected result.

% % ---------------------------------------------------------------------------
\subsubsection{Case 4a}
\begin{table}[htbp]

\centering
\begin{tabular}{ l r r r }

\hline
\rowcolor{anti-flashwhite}
\bf{Result} & \bf{Expected} & \bf{OpenQuake} & \bf{Difference}\\
\hline
Average loss & 70.25 & 70.25 & 0.00\% \\
\hline
\end{tabular}

\caption{Results for classical risk test case 4a}
\label{tab:result-cr-4a}
\end{table}
\begin{table}[htbp]

\centering
\begin{tabular}{ l r r r }

\hline
\rowcolor{anti-flashwhite}
\bf{Result} & \bf{Expected} & \bf{OpenQuake} & \bf{Difference}\\
\hline
Average loss & 70.25 & 70.25 & 0.00\% \\
\hline
\end{tabular}

\caption{Results for classical risk test case 4a}
\label{tab:result-cr-4a}
\end{table}
Table \ref{tab:result-cr-4a} shows the comparison of the OpenQuake result for average annual loss with the expected result.

% % ---------------------------------------------------------------------------
\subsubsection{Case 4b}
This case tests the computation of the loss exceedance curve and average annual loss when the value of the assets is specified per unit, and the number of units in each asset are provided in the exposure model. The vulnerability function used is the same as in Case~1c and shown in Table~\ref{tab:vf-ln-tax1-nzcov}. The asset has two units, and the value per unit is $7,500$. The aggregate asset value in this case is $15,000$.
\begin{table}[htbp]

\centering
\begin{tabular}{ l r r r }

\hline
\rowcolor{anti-flashwhite}
\bf{Result} & \bf{Expected} & \bf{OpenQuake} & \bf{Difference}\\
\hline
Average loss & 52.69 & 52.69 & 0.00\% \\
\hline
\end{tabular}

\caption{Results for classical risk test case 4b}
\label{tab:result-cr-4b}
\end{table}
Table \ref{tab:result-cr-4b} shows the comparison of the OpenQuake result for average annual loss with the expected result.

% % ---------------------------------------------------------------------------
\subsubsection{Case 4c}
This case tests the computation of the loss exceedance curve and average annual loss when the value of the assets is specified per unit area, and the aggregate area of each asset is provided in the exposure model. The vulnerability function used is the same as in Case~1c and shown in Table~\ref{tab:vf-ln-tax1-nzcov}. The asset has an aggregate area of $1,000$ sq. units, and the value per unit area is $5$. The aggregate asset value in this case is $5,000$. The average annual loss in this case should be exactly half the value calculated in Case~1c.
This case tests the computation of the loss exceedance curve and average annual loss when the value of the assets is specified per unit area, and the aggregate area of each asset is provided in the exposure model. The vulnerability function used is the same as in Case~1c and shown in Table~\ref{tab:vf-ln-tax1-nzcov}. The asset has an aggregate area of $1,000$ sq. units, and the value per unit area is $5$. The aggregate asset value in this case is $5,000$. The average annual loss in this case should be exactly half the value calculated in Case~1c.
Table \ref{tab:result-cr-4c} shows the comparison of the OpenQuake result for average annual loss with the expected result.

% % ---------------------------------------------------------------------------
\subsubsection{Case 4d}
This case tests the computation of the loss exceedance curve and average annual loss when the value of the assets is specified per unit area, the area is specified per unit, and the number of units in each asset are provided in the exposure model. The vulnerability function used is the same as in Case~1c and shown in Table~\ref{tab:vf-ln-tax1-nzcov}. The asset has three units, the area per unit is $400$ sq. units, and the value per unit area is $10$. The aggregate asset value in this case is $12,000$.
\begin{table}[htbp]

\centering
\begin{tabular}{ l r r r }

\hline
\rowcolor{anti-flashwhite}
\bf{Result} & \bf{Expected} & \bf{OpenQuake} & \bf{Difference}\\
\hline
Average loss & 42.15 & 42.15 & 0.00\% \\
\hline
\end{tabular}

\caption{Results for classical risk test case 4c}
\label{tab:result-cr-4c}
\end{table}
Table \ref{tab:result-cr-4d} shows the comparison of the OpenQuake result for average annual loss with the expected result.


% -----------------------------------------------------------------------------
\subsection{Multiple asset tests}
\label{subsec:acc-cr-multiple}
The multiple asset test cases are designed to test the correct working of the classical risk calculator for a portfolio of assets of different taxonomies. The vulnerability functions for the different taxonomies are based on different intensity measure types.

The list of assets in the exposure model used for the multiple-asset test cases for the classical risk calculator is given in Table \ref{tab:assets-tax3}.

\subsubsection{Case 5a}
The purpose of this case is to test the basic elements of a classical risk calculation involving multiple assets, such as the computation of the loss exceedance curve and average annual loss for each asset in a portfolio of assets.


\begin{table}[htbp]

\centering
\begin{tabular}{ l r r r }

\hline
\rowcolor{anti-flashwhite}
\bf{Result} & \bf{Expected} & \bf{OpenQuake} & \bf{Difference}\\
\hline
Average loss & 91.10 & 91.10 & 0\% \\
\hline
\end{tabular}

\caption{Results for classical risk test case 5a}
\label{tab:result-cr-5a}
\end{table}
Table~\ref{tab:result-cr-5a} shows the comparison of the OpenQuake results for average annual losses for the seven assets with the expected results.

% % -----------------------------------------------------------------------------
% \subsection{Insurance tests}
% \label{subsec:acc-cr-insurance}
% \subsubsection{Case 8a}


\begin{table}[htbp]

\centering
\begin{tabular}{ l r r r }

\hline
\rowcolor{anti-flashwhite}
\bf{Result} & \bf{Expected} & \bf{OpenQuake} & \bf{Difference}\\
\hline
Average insured loss &  &  & 0\% \\
\hline
\end{tabular}

\caption{Results for classical risk test case 8a}
\label{tab:result-classical-risk-8a}
\end{table}

Table \ref{tab:result-classical-risk-8a} shows the comparison of the OpenQuake result with the expected result.

% ---------------------------------------------------------------------------

% % -----------------------------------------------------------------------------
\subsection{Calculation with logic-trees}
\label{subsec:acc-cr-lt}
\subsubsection{Case 7a}
The OpenQuake scenario risk calculator allows the user to employ more than one ground motion prediction equation (GMPE) for computing the hazard curves used for the loss curve calculation. The loss exceedance curves, expected losses over the specified time period, and insured loss curves (if any), are calculated and output for each GMPE branch independently. No sampling is involved, and any branch weights assigned to the different GMPE branches are ignored. This case is designed to test the computation of the loss curve and average annual loss for an asset with two different GMPEs. The two ground motion prediction equations used are Boore and Atkinson (2008), and Chiou and Youngs (2008).

A single asset is used in this test case. Table~\ref{tab:vf-ln-tax1-nzcov} shows the mean loss ratios and corresponding coefficients of variation in the vulnerability function used in this test case. The loss curve calculation procedure follows the same steps as described earlier in Case~1c for each of the two individual hazard branches.

The expected loss curves for this case, computed using the procedure described in Case~1c, are compared with the loss curves obtained using the OpenQuake classical PSHA based risk calculator in Figure~\ref{fig:lc-cr-7a}.

\begin{figure}[htbp]
\centering
\includegraphics[width=12cm]{qareport/figures/fig-lc-cr-7a}
\caption{Loss curve comparison for classical risk test case 7a}
\label{fig:lc-cr-7a}
\end{figure}

The area under the loss exceedance curves gives the average annual loss values for the two hazard branches.
\begin{table}[htbp]

\centering
\begin{tabular}{ l l l r r r }

\hline
\rowcolor{anti-flashwhite}
\bf{Branch} & \bf{Asset} & \bf{Result} & \bf{Expected} & \bf{OpenQuake} & \bf{Difference}\\
\hline
BA2008 & a1 & Average loss & 35.13 & 35.13 & 0.00\% \\
CY2008 & a1 & Average loss & 49.55 & 49.55 & 0.00\% \\
\hline
\end{tabular}

\caption{Results for classical risk test case 7a}
\label{tab:result-cr-7a}
\end{table}

Table \ref{tab:result-cr-7a} shows the comparison of the OpenQuake result for average annual losses for the two branches with the expected results.

% ---------------------------------------------------------------------------

