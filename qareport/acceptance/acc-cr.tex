The tests for the classical PSHA-based risk calculator assume the correct computation of the hazard curves at the locations of the assets in the exposure model. Thus, the risk tests implicitly rely on the acceptance tests for the classical PSHA-based hazard calculator.

The source model used for the tests comprises a single vertical strike-slip fault with a Gutenberg-Richter b-value equal to 0.9 and a slip rate of 2 mm/yr. The MFD is a Gutenberg-Richter distribution truncated between magnitudes 5.0 and 6.5, while the Ground Motion Prediction Equation (GMPE) used is Sadigh et al. (1997), with sigma set to zero.\\

\noindent Details of the fault geometry are given below:\\

\noindent
Fault type: Strike slip\\
Fault dip: $90^{\circ}$\\
Fault plane depths: 0--12 km\\
Fault coordinates:\\
South end: $38.0000^{\circ} N$, $122.0000^{\circ} W$\\
North end: $38.2248^{\circ} N$, $122.0000^{\circ} W$\\

The complete collection of input models and job configuration files used in these test cases can be accessed here:
\href{https://github.com/gem/oq-risklib/tree/master/openquake/qa_tests_data/classical_risk}
{https://github.com/gem/oq-risklib/tree/master/openquake/qa\_tests\_data/classical\_risk}

\subsection{Single asset tests}
\label{subsec:classical-risk-single}
The single asset test cases are designed to test the basic elements of the classical-PSHA based risk calculator, such as:

\begin{itemize}
\item asset loss ratio exceedance curve computation
\item asset loss exceedance curve computation
\end{itemize}

The location and taxonomy of the single asset in the exposure model used for the single-asset test cases for the classical risk calculator are given in Table \ref{tab:assets}.

% ---------------------------------------------------------------------------
\subsubsection{Case 1a}
Test case 1a 

\begin{table}[htbp]

\centering
\begin{tabular}{ l r r r }

\hline
\rowcolor{anti-flashwhite}
\bf{Result} & \bf{Expected} & \bf{OpenQuake} & \bf{Difference}\\
\hline
Average loss &  &  & 0\% \\
\hline
\end{tabular}

\caption{Results for classical risk test case 1a}
\label{tab:result-classical-risk-1a}
\end{table}

Table \ref{tab:result-classical-risk-1a} shows the comparison of the OpenQuake result with the expected result.

% ---------------------------------------------------------------------------
\subsubsection{Case 1b}
Test case 1b 

\begin{table}[htbp]

\centering
\begin{tabular}{ l r r r }

\hline
\rowcolor{anti-flashwhite}
\bf{Result} & \bf{Expected} & \bf{OpenQuake} & \bf{Difference}\\
\hline
Average loss & 45.55 & 45.55 & 0\% \\
\hline
\end{tabular}

\caption{Results for classical risk test case 1b}
\label{tab:result-classical-risk-1b}
\end{table}

Table \ref{tab:result-classical-risk-1b} shows the comparison of the OpenQuake result with the expected result.

% ---------------------------------------------------------------------------
\subsubsection{Case 1c}


\begin{table}[htbp]

\centering
\begin{tabular}{ l r r r }

\hline
\rowcolor{anti-flashwhite}
\bf{Result} & \bf{Expected} & \bf{OpenQuake} & \bf{Difference}\\
\hline
Average loss &  &  & 0\% \\
\hline
\end{tabular}

\caption{Results for classical risk test case 1c}
\label{tab:result-classical-risk-1c}
\end{table}

Table \ref{tab:result-classical-risk-1c} shows the comparison of the OpenQuake result with the expected result.

% ---------------------------------------------------------------------------
\subsubsection{Case 1d}


\begin{table}[htbp]

\centering
\begin{tabular}{ l r r r }

\hline
\rowcolor{anti-flashwhite}
\bf{Result} & \bf{Expected} & \bf{OpenQuake} & \bf{Difference}\\
\hline
Average loss &  &  & 0\% \\
\hline
\end{tabular}

\caption{Results for classical risk test case 1d}
\label{tab:result-classical-risk-1d}
\end{table}

Table \ref{tab:result-classical-risk-1d} shows the comparison of the OpenQuake result with the expected result.

% ---------------------------------------------------------------------------
\subsubsection{Case 1e}


\begin{table}[htbp]

\centering
\begin{tabular}{ l r r r }

\hline
\rowcolor{anti-flashwhite}
\bf{Result} & \bf{Expected} & \bf{OpenQuake} & \bf{Difference}\\
\hline
Average loss & 46.07 & 46.07 & 0\% \\
\hline
\end{tabular}

\caption{Results for classical risk test case 1e}
\label{tab:result-classical-risk-1e}
\end{table}

Table \ref{tab:result-classical-risk-1e} shows the comparison of the OpenQuake result with the expected result.

% ---------------------------------------------------------------------------
\subsubsection{Case 2a}


\begin{table}[htbp]

\centering
\begin{tabular}{ l r r r }

\hline
\rowcolor{anti-flashwhite}
\bf{Result} & \bf{Expected} & \bf{OpenQuake} & \bf{Difference}\\
\hline
Average loss &  &  & 0\% \\
\hline
\end{tabular}

\caption{Results for classical risk test case 2a}
\label{tab:result-classical-risk-2a}
\end{table}

Table \ref{tab:result-classical-risk-2a} shows the comparison of the OpenQuake result with the expected result.

% ---------------------------------------------------------------------------
\subsubsection{Case 2b}


\begin{table}[htbp]

\centering
\begin{tabular}{ l r r r }

\hline
\rowcolor{anti-flashwhite}
\bf{Result} & \bf{Expected} & \bf{OpenQuake} & \bf{Difference}\\
\hline
Average loss &  &  & 0\% \\
\hline
\end{tabular}

\caption{Results for classical risk test case 2b}
\label{tab:result-classical-risk-2b}
\end{table}

Table \ref{tab:result-classical-risk-2b} shows the comparison of the OpenQuake result with the expected result.

% ---------------------------------------------------------------------------
\subsubsection{Case 2c}


\begin{table}[htbp]

\centering
\begin{tabular}{ l r r r }

\hline
\rowcolor{anti-flashwhite}
\bf{Result} & \bf{Expected} & \bf{OpenQuake} & \bf{Difference}\\
\hline
Average loss & 9.11 & 9.11 & 0\% \\
\hline
\end{tabular}

\caption{Results for classical risk test case 2c}
\label{tab:result-classical-risk-2c}
\end{table}

Table \ref{tab:result-classical-risk-2c} shows the comparison of the OpenQuake result with the expected result.

% ---------------------------------------------------------------------------
\subsubsection{Case 2d}


\begin{table}[htbp]

\centering
\begin{tabular}{ l r r r }

\hline
\rowcolor{anti-flashwhite}
\bf{Result} & \bf{Expected} & \bf{OpenQuake} & \bf{Difference}\\
\hline
Average loss & $8.78 \times 10^{-4}$ & $8.78 \times 10^{-4}$ & 0\% \\
\hline
\end{tabular}

\caption{Results for classical risk test case 2d}
\label{tab:result-classical-risk-2d}
\end{table}

Table \ref{tab:result-classical-risk-2d} shows the comparison of the OpenQuake result with the expected result.

% ---------------------------------------------------------------------------
\subsubsection{Case 3a}


\begin{table}[htbp]

\centering
\begin{tabular}{ l r r r }

\hline
\rowcolor{anti-flashwhite}
\bf{Result} & \bf{Expected} & \bf{OpenQuake} & \bf{Difference}\\
\hline
Average loss &  &  & 0\% \\
\hline
\end{tabular}

\caption{Results for classical risk test case 3a}
\label{tab:result-classical-risk-3a}
\end{table}

Table \ref{tab:result-classical-risk-3a} shows the comparison of the OpenQuake result with the expected result.

% ---------------------------------------------------------------------------
\subsubsection{Case 4a}


\begin{table}[htbp]

\centering
\begin{tabular}{ l r r r }

\hline
\rowcolor{anti-flashwhite}
\bf{Result} & \bf{Expected} & \bf{OpenQuake} & \bf{Difference}\\
\hline
Average loss & 45.55 & 45.55 & 0\% \\
\hline
\end{tabular}

\caption{Results for classical risk test case 4a}
\label{tab:result-classical-risk-4a}
\end{table}

Table \ref{tab:result-classical-risk-4a} shows the comparison of the OpenQuake result with the expected result.

% ---------------------------------------------------------------------------
\subsubsection{Case 4b}


\begin{table}[htbp]

\centering
\begin{tabular}{ l r r r }

\hline
\rowcolor{anti-flashwhite}
\bf{Result} & \bf{Expected} & \bf{OpenQuake} & \bf{Difference}\\
\hline
Average loss &  &  & 0\% \\
\hline
\end{tabular}

\caption{Results for classical risk test case 4b}
\label{tab:result-classical-risk-4b}
\end{table}

Table \ref{tab:result-classical-risk-4b} shows the comparison of the OpenQuake result with the expected result.

% ---------------------------------------------------------------------------
\subsubsection{Case 4c}


\begin{table}[htbp]

\centering
\begin{tabular}{ l r r r }

\hline
\rowcolor{anti-flashwhite}
\bf{Result} & \bf{Expected} & \bf{OpenQuake} & \bf{Difference}\\
\hline
Average loss &  &  & 0\% \\
\hline
\end{tabular}

\caption{Results for classical risk test case 4c}
\label{tab:result-classical-risk-4c}
\end{table}

Table \ref{tab:result-classical-risk-4c} shows the comparison of the OpenQuake result with the expected result.

% ---------------------------------------------------------------------------
\subsubsection{Case 5a}


\begin{table}[htbp]

\centering
\begin{tabular}{ l r r r }

\hline
\rowcolor{anti-flashwhite}
\bf{Result} & \bf{Expected} & \bf{OpenQuake} & \bf{Difference}\\
\hline
Average loss &  &  & 0\% \\
\hline
\end{tabular}

\caption{Results for classical risk test case 5a}
\label{tab:result-classical-risk-5a}
\end{table}

Table \ref{tab:result-classical-risk-5a} shows the comparison of the OpenQuake result with the expected result.

% ---------------------------------------------------------------------------
\subsubsection{Case 5b}


\begin{table}[htbp]

\centering
\begin{tabular}{ l r r r }

\hline
\rowcolor{anti-flashwhite}
\bf{Result} & \bf{Expected} & \bf{OpenQuake} & \bf{Difference}\\
\hline
Average loss & 68.32 & 68.32 & 0\% \\
\hline
\end{tabular}

\caption{Results for classical risk test case 5b}
\label{tab:result-classical-risk-5b}
\end{table}

Table \ref{tab:result-classical-risk-5b} shows the comparison of the OpenQuake result with the expected result.

% ---------------------------------------------------------------------------
\subsubsection{Case 5c}


\begin{table}[htbp]

\centering
\begin{tabular}{ l r r r }

\hline
\rowcolor{anti-flashwhite}
\bf{Result} & \bf{Expected} & \bf{OpenQuake} & \bf{Difference}\\
\hline
Average loss &  &  & 0\% \\
\hline
\end{tabular}

\caption{Results for classical risk test case 5c}
\label{tab:result-classical-risk-5c}
\end{table}

Table \ref{tab:result-classical-risk-5c} shows the comparison of the OpenQuake result with the expected result.

% ---------------------------------------------------------------------------
\subsubsection{Case 5d}


\begin{table}[htbp]

\centering
\begin{tabular}{ l r r r }

\hline
\rowcolor{anti-flashwhite}
\bf{Result} & \bf{Expected} & \bf{OpenQuake} & \bf{Difference}\\
\hline
Average loss &  &  & 0\% \\
\hline
\end{tabular}

\caption{Results for classical risk test case 5d}
\label{tab:result-classical-risk-5d}
\end{table}

Table \ref{tab:result-classical-risk-5d} shows the comparison of the OpenQuake result with the expected result.

% -----------------------------------------------------------------------------
\subsection{Multiple asset tests}
\label{subsec:classical-risk-multiple}
The multiple asset test cases are designed to test the correct working of the classical risk calculator for a portfolio of assets of different taxonomies. The vulnerability functions for the different taxonomies are based on different intensity measure types.

\begin{table}

\centering
\begin{tabular}{ c l c c l }

\hline
\rowcolor{anti-flashwhite}
\bf{Site} & \bf{Taxonomy} & \bf{Latitude} & \bf{Longitude} & \bf{Comment} \\
\hline
1 & Wood & 38.113 & -122.000 & On fault midpoint, along strike \\
2 & Wood & 38.113 & -122.114 & 10 km west of fault, at midpoint \\
3 & RC & 38.113 & -122.570 & 50 km west of fault, at midpoint \\
4 & RC & 38.000 & -122.000 & South end of fault \\
5 & Steel & 37.910 & -122.000 & 10 km south of fault, along strike \\
6 & Wood & 38.225 & -122.000 & North end of fault \\
7 & Steel & 38.113 & -121.886 & 10 km east of fault, at midpoint \\
\hline
\end{tabular}

\caption{Asset sites and taxonomies for the multiple-asset test cases}
\label{tab:assets}
\end{table}

The list of assets in the exposure model used for the multiple-asset test cases for the classical risk calculator is given in Table \ref{tab:assets}.

% -----------------------------------------------------------------------------
\subsection{Calculation with logic-trees}
\label{subsec:classical-risk-logictrees}
\subsubsection{Case 7a}


\begin{table}[htbp]

\centering
\begin{tabular}{ l l l r r r }

\hline
\rowcolor{anti-flashwhite}
\bf{Branch} & \bf{Asset} & \bf{Result} & \bf{Expected} & \bf{OpenQuake} & \bf{Difference}\\
\hline
b1 & a1 & Average loss &  &  & 0\% \\
b2 & a1 & Average loss &  &  & 0\% \\
\hline
\end{tabular}

\caption{Results for classical risk test case 7a}
\label{tab:result-classical-risk-7a}
\end{table}

Table \ref{tab:result-classical-risk-7a} shows the comparison of the OpenQuake result with the expected result.

% ---------------------------------------------------------------------------

% -----------------------------------------------------------------------------
\subsection{Insurance tests}
\label{subsec:classical-risk-insurance}
\subsubsection{Case 8a}


\begin{table}[htbp]

\centering
\begin{tabular}{ l r r r }

\hline
\rowcolor{anti-flashwhite}
\bf{Result} & \bf{Expected} & \bf{OpenQuake} & \bf{Difference}\\
\hline
Average insured loss &  &  & 0\% \\
\hline
\end{tabular}

\caption{Results for classical risk test case 8a}
\label{tab:result-classical-risk-8a}
\end{table}

Table \ref{tab:result-classical-risk-8a} shows the comparison of the OpenQuake result with the expected result.

% ---------------------------------------------------------------------------
