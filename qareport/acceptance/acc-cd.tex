The tests for the classical PSHA-based damage calculator assume the correct computation of the hazard curves at the locations of the assets in the exposure model. Thus, the damage tests implicitly rely on the acceptance tests for the classical PSHA-based hazard calculator.

The source model used for the tests comprises a single vertical strike-slip fault with a Gutenberg-Richter b-value equal to 0.9 and a slip rate of 2 mm/yr. The MFD is a Gutenberg-Richter distribution truncated between magnitudes 5.0 and 6.5, while the Ground Motion Prediction Equation (GMPE) used is Sadigh et al. (1997), with sigma set to zero.\\

\noindent Details of the fault geometry are given below:\\

\noindent
Fault type: Strike slip\\
Fault dip: $90^{\circ}$\\
Fault plane depths: 0--12 km\\
Fault coordinates:\\
South end: $38.0000^{\circ} N$, $122.0000^{\circ} W$\\
North end: $38.2248^{\circ} N$, $122.0000^{\circ} W$\\

The complete collection of input models and job configuration files used in these test cases can be accessed here:
\href{https://github.com/gem/oq-risklib/tree/master/openquake/qa_tests_data/classical_damage}
{https://github.com/gem/oq-risklib/tree/master/openquake/qa\_tests\_data/classical\_damage}

\subsection{Single asset tests}
\label{subsec:cd-single}
The single asset test cases are designed to test the basic elements of the classical-PSHA based damage calculator, such as:

\begin{itemize}
\item damage state occurrence probabilities for assets
\end{itemize}

The location and taxonomy of the single asset in the exposure model used for the single-asset test cases for the classical damage calculator are given in Table~\ref{tab:asset}.


% ---------------------------------------------------------------------------
\subsubsection{Case 1a}
The purpose of this case is to test the integration of the derivative of the site hazard curve with the discrete fragility function for a single asset. Table~\ref{tab:ff-disc-tax1-zndl} shows the set of ground motion intensity levels and corresponding probabilities of exceedance for the four damage states for the discrete fragility function used in this test case.

When the exposure model and discrete fragility model are provided to the OpenQuake classical PSHA-based hazard calculator, OpenQuake computes the hazard curves at the locations of the assets in the exposure model and at the specific intensity levels used in the fragility functions.

\begin{table}[htbp]

\centering
\tabcolsep=0.11cm
\scalebox{0.7}{

\begin{tabular}{ l c c c c c c c c c c }

\hline
\rowcolor{anti-flashwhite}
\bf{PGA} & \bf{0.05g} & \bf{0.20g} & \bf{0.40g} & \bf{0.60g} & \bf{0.80g} & \bf{1.00g} & \bf{1.20g} & \bf{1.40g} & \bf{\dots} & \bf{5.00g} \\
\hline
\bf{P.O.E.} & 3.896\times10^{-2} & 2.222\times10^{-2} & 8.171\times10^{-3} & 3.070\times10^{-3} & 1.230\times10^{-3} & 5.195\times10^{-4} & 2.254\times10^{-4} & 9.918\times10^{-5} & \dots & 0.000 \\
\hline
\end{tabular}

}

\caption{Hazard curve for PGA at a single site}
\label{tab:hc-l1-cd}
\end{table}

The intensity levels for the hazard curve are extracted from the fragility function: \\
$[0.05, 0.20, 0.40, 0.60, 0.80, 1.00, 1.20, 1.40, \dots, 5.00]$\\

The hazard curve gives the probabilities of exceedance for a set of intensity levels within a specified time period. The time period in this case, $t_H$, is one year. The hazard curve at the location of the single asset used in this test case is shown in Table~\ref{tab:hc-l1-cd}.

The probabilities of exceedance are: $[3.986\times10^{-2}, 2.222\times10^{-2}, 8.171\times10^{-3}, 3.071\times10^{-3}, 1.230\times10^{-3}, 5.195\times10^{-4}, 2.254\times10^{-4}, 9.918\times10^{-5}, \dots, 0]$. The probabilities of exceedance are first converted to annual rates (or frequencies) of exceedance by employing the Poissonion conversion:

\begin{equation}
	\lambda(iml) = \frac{-\ln [1 - prob(IML > iml, t_H)]}{t_H}
\end{equation}

The annual frequencies of exceedance are: $[3.974\times10^{2}, 2.247\times10^{2}, 8.205\times10^{3}, 3.075\times10^{3}, 1.231\times10^{3}, 5.197\times10^{4}, 2.254\times10^{4}, 9.918\times10^{5}, \dots, 0]$.

The annual frequencies of occurrence are estimated by the differentiation of the annual frequencies of exceedance: $[1.727\times10^{2}, 1.426\times10^{2}, 5.130\times10^{3}, 1.845\times10^{3}, 7.109\times10^{4}, 2.942\times10^{4}, 1.262\times10^{4}, 5.565\times10^{5}, \dots, 0]$.

Now, the annual frequencies of occurrence of the set of intensity measure levels, as computed above, are multiplied by the corresponding probabilities of exceedance of the damage states given these intensity measure levels. The probabilities of exceedance of the damage states are obtained directly from the discrete fragility function. For each damage state, the products thus obtained are summed across all the intensity measure levels, giving the annual frequency of exceedance for that damage state.

Assuming that exceedances of the damage states follow Poisson processes, the probabilities of exceedance for the set of damage states are calculated from the annual frequencies of exceedance $\lambda_{D \geq d_i}$ and the exposure time period $t_R$ through:

\begin{equation}
	prob(D \geq d_i, t_R) = 1 - \exp (-\lambda_{D \geq d_i} \times t_R)
\end{equation}

Finaly, the probabilities of occurrence for the set of damage states within the specified time period are computed from the probabilities of exceedance calculated above.
\begin{table}[htbp]

\centering
\begin{tabular}{ l l r r r }

\hline
\rowcolor{anti-flashwhite}
\bf{Asset} & \bf{Damage State} & \bf{Expected} & \bf{OpenQuake} & \bf{Difference}\\
\hline
a1 & none & $9.93 \times 10^{-1}$ & $9.93 \times 10^{-1}$ & $0\%$ \\
   & ds1 & $4.92 \times 10^{-3}$ & $4.92 \times 10^{-3}$ & $0\%$ \\
   & ds2 & $6.40 \times 10^{-4}$ & $6.40 \times 10^{-4}$ & $0\%$ \\
   & ds3 & $1.87 \times 10^{-4}$ & $1.87 \times 10^{-4}$ & $0\%$ \\
   & ds4 & $7.96 \times 10^{-4}$ & $7.96 \times 10^{-4}$ & $0\%$ \\
\hline
\end{tabular}

\caption{Results for classical damage test case 1a}
\label{tab:result-cd-1a}
\end{table}

Table \ref{tab:result-cd-1a} shows the comparison of the OpenQuake result with the expected result.




% -----------------------------------------------------------------------------
\subsection{Multiple asset tests}
\label{subsec:cd-multiple}
The multiple asset test cases are designed to test the correct working of the classical damage calculator for a portfolio of assets of different taxonomies using different fragility functions.

\subsubsection{Case 3a}
The list of assets in the exposure model used in this case is given in Table~\ref{tab:assets-tax3}. As shown in the table, the assets fall into three different taxonomies. Table~\ref{tab:ff-cont-tax3-zmin} shows the parameters for the continuous lognormal fragility functions for the three taxonomies. Seven hazard curves are computed, one each for the seven sites. The procedure used for calculating the damage state probabilities for each of the seven individual assets remains the same as described in Case~1b, with the appropriate fragility function used for each asset depending on its assigned taxonomy.
\begin{table}[htbp]

\centering
\begin{tabular}{ l l l r r r }

\hline
\rowcolor{anti-flashwhite}
\bf{Asset} & \bf{Damage State} & \bf{Result} & \bf{Expected} & \bf{OpenQuake} & \bf{Difference}\\
\hline
a1 & No damage & Probability of occurrence & $9.72 \times 10^{-1}$ & $9.72 \times 10^{-1}$ & 0\% \\
   & Slight    & Probability of occurrence & $8.02 \times 10^{-3}$ & $8.02 \times 10^{-3}$ & 0\% \\
   & Moderate  & Probability of occurrence & $1.06 \times 10^{-2}$ & $1.06 \times 10^{-2}$ & 0\% \\
   & Extreme   & Probability of occurrence & $5.73 \times 10^{-3}$ & $5.73 \times 10^{-3}$ & 0\% \\
   & Complete  & Probability of occurrence & $3.15 \times 10^{-3}$ & $3.15 \times 10^{-3}$ & 0\% \\
\hline
\end{tabular}

\caption{Results for classical damage test case 3a}
\label{tab:result-classical-damage-3a}
\end{table}
Table \ref{tab:result-cd-3a} shows the comparison of the OpenQuake results with the expected results for assets $a1$, $a2$, and $a3$.

% -----------------------------------------------------------------------------
\subsection{Calculation with logic-trees}
\label{subsec:cd-lt}
\subsubsection{Case 7a}


\input{qareport/results/cd-7a}

Table \ref{tab:result-cd-7a} shows the comparison of the OpenQuake result with the expected result.