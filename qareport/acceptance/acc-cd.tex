The tests for the classical PSHA-based damage calculator assume the correct computation of the hazard curves at the locations of the assets in the exposure model. Thus, the damage tests implicitly rely on the acceptance tests for the classical PSHA-based hazard calculator.

The source model used for the tests comprises a single vertical strike-slip fault with a Gutenberg-Richter b-value equal to 0.9 and a slip rate of 2 mm/yr. The MFD is a Gutenberg-Richter distribution truncated between magnitudes 5.0 and 6.5, while the Ground Motion Prediction Equation (GMPE) used is Boore and Atkinson (2008).\\

\noindent Details of the fault geometry are given below:\\

\noindent
Fault type: Strike slip\\
Fault dip: $90^{\circ}$\\
Fault plane depths: 0--12 km\\
Fault coordinates:\\
South end: $38.0000^{\circ} N$, $122.0000^{\circ} W$\\
North end: $38.2248^{\circ} N$, $122.0000^{\circ} W$\\

The complete collection of input models and job configuration files used in these test cases can be accessed here:
\href{https://github.com/gem/oq-risklib/tree/master/openquake/qa_tests_data/classical_damage}
{https://github.com/gem/oq-risklib/tree/master/openquake/qa\_tests\_data/classical\_damage}

\subsection{Single asset tests}
\label{subsec:cd-single}
The single asset test cases are designed to test the basic elements of the classical-PSHA based damage calculator, such as:

\begin{itemize}
\item damage state occurrence probabilities for assets
\end{itemize}

The location and taxonomy of the single asset in the exposure model used for the single-asset test cases for the classical damage calculator are given in Table~\ref{tab:asset}.


% ---------------------------------------------------------------------------
\subsubsection{Case 1a}
The purpose of this case is to test the integration of the derivative of the site hazard curve with the discrete fragility function for a single asset. Table~\ref{tab:ff-disc-tax1-zndl} shows the set of ground motion intensity levels and corresponding probabilities of exceedance for the four damage states for the discrete fragility function used in this test case.

When the exposure model and discrete fragility model are provided to the OpenQuake classical PSHA-based hazard calculator, OpenQuake computes the hazard curves at the locations of the assets in the exposure model and at the specific intensity levels used in the fragility functions.

\begin{table}[htbp]

\centering
\tabcolsep=0.11cm
\scalebox{0.7}{

\begin{tabular}{ l c c c c c c c c c c }

\hline
\rowcolor{anti-flashwhite}
\bf{PGA} & \bf{0.05g} & \bf{0.20g} & \bf{0.40g} & \bf{0.60g} & \bf{0.80g} & \bf{1.00g} & \bf{1.20g} & \bf{1.40g} & \bf{\dots} & \bf{5.00g} \\
\hline
\bf{P.O.E.} & 3.896\times10^{-2} & 2.222\times10^{-2} & 8.171\times10^{-3} & 3.070\times10^{-3} & 1.230\times10^{-3} & 5.195\times10^{-4} & 2.254\times10^{-4} & 9.918\times10^{-5} & \dots & 0.000 \\
\hline
\end{tabular}

}

\caption{Hazard curve for PGA at a single site}
\label{tab:hc-l1-cd}
\end{table}

The intensity levels for the hazard curve are extracted from the fragility function: \\
$[0.05, 0.20, 0.40, 0.60, 0.80, 1.00, 1.20, 1.40, \dots, 5.00]$\\

The hazard curve gives the probabilities of exceedance for a set of intensity levels within a specified time period. The time period in this case, $t_H$, is one year. The hazard curve at the location of the single asset used in this test case is shown in Table~\ref{tab:hc-l1-cd}.

The probabilities of exceedance are: $[3.986\times10^{-2}, 2.222\times10^{-2}, 8.171\times10^{-3}, 3.071\times10^{-3}, 1.230\times10^{-3}, 5.195\times10^{-4}, 2.254\times10^{-4}, 9.918\times10^{-5}, \dots, 0]$. The probabilities of exceedance are first converted to annual rates (or frequencies) of exceedance by employing the Poissonion conversion:

\begin{equation}
	\lambda(iml) = \frac{-\ln [1 - prob(IML > iml, t_H)]}{t_H}
\end{equation}

The annual frequencies of exceedance are: $[3.974\times10^{2}, 2.247\times10^{2}, 8.205\times10^{3}, 3.075\times10^{3}, 1.231\times10^{3}, 5.197\times10^{4}, 2.254\times10^{4}, 9.918\times10^{5}, \dots, 0]$.

The annual frequencies of occurrence are estimated by the differentiation of the annual frequencies of exceedance: $[1.727\times10^{2}, 1.426\times10^{2}, 5.130\times10^{3}, 1.845\times10^{3}, 7.109\times10^{4}, 2.942\times10^{4}, 1.262\times10^{4}, 5.565\times10^{5}, \dots, 0]$.

Now, the annual frequencies of occurrence of the set of intensity measure levels, as computed above, are multiplied by the corresponding probabilities of exceedance of the damage states given these intensity measure levels. The probabilities of exceedance of the damage states are obtained directly from the discrete fragility function. For each damage state, the products thus obtained are summed across all the intensity measure levels, giving the annual frequency of exceedance for that damage state.

Assuming that exceedances of the damage states follow Poisson processes, the probabilities of exceedance for the set of damage states are calculated from the annual frequencies of exceedance $\lambda_{D \geq d_i}$ and the exposure time period $t_R$ through:

\begin{equation}
	prob(D \geq d_i, t_R) = 1 - \exp (-\lambda_{D \geq d_i} \times t_R)
\end{equation}

Finaly, the probabilities of occurrence for the set of damage states within the specified time period are computed from the probabilities of exceedance calculated above.
\begin{table}[htbp]

\centering
\begin{tabular}{ l l r r r }

\hline
\rowcolor{anti-flashwhite}
\bf{Asset} & \bf{Damage State} & \bf{Expected} & \bf{OpenQuake} & \bf{Difference}\\
\hline
a1 & none & $9.93 \times 10^{-1}$ & $9.93 \times 10^{-1}$ & $0\%$ \\
   & ds1 & $4.92 \times 10^{-3}$ & $4.92 \times 10^{-3}$ & $0\%$ \\
   & ds2 & $6.40 \times 10^{-4}$ & $6.40 \times 10^{-4}$ & $0\%$ \\
   & ds3 & $1.87 \times 10^{-4}$ & $1.87 \times 10^{-4}$ & $0\%$ \\
   & ds4 & $7.96 \times 10^{-4}$ & $7.96 \times 10^{-4}$ & $0\%$ \\
\hline
\end{tabular}

\caption{Results for classical damage test case 1a}
\label{tab:result-cd-1a}
\end{table}

Table \ref{tab:result-cd-1a} shows the comparison of the OpenQuake damage state probabilities with the expected results.

% ---------------------------------------------------------------------------
\subsubsection{Case 1b}
Whereas the previous case was concerned with checking the correct implementation and usage of \emph{discrete}discrete fragility functions, the purpose of this case is to verify the correct calculation of damage distribution statistics for the classical damage calculator using \emph{continuous} (lognormal CDF) fragility functions.

Table~\ref{tab:ff-cont-tax1-zmin} shows the mean and standard deviation of the ground motion intensity level for the four damage states, which are the parameters for the lognormal fragility function used in this test case.

The only difference in the calculation procedure compared with Case~1a is the criterion used for selecting the set of intensity levels at which to compute the hazard curve. OpenQuake discretizes the continuous lognormal fragility functions into a set of intensity levels and corresponding probabilities of exceedance for each damage state. The minimum and maximum intensity levels used for this discretization are those specified in the fragility model file. The number of intervals into which this range of intensity levels is discretized is specified using the configuration parameter `continuous\_fragility\_discretization'. In this case, the minimum and maximum intensity levels specifed in the continuous fragility model are $0.0 g$ and $5.0 g$ respectively. The parameter `continuous\_fragility\_discretization' is set to 29, which means that OpenQuake will discretize the interval $[0.0, 5.0]g$ into 29 equal subintervals, yielding thirty intensity levels including the minimum and maximum limits.

The damage state exceedance probabilities are obtained by evaluating the complementary cumulative distribution function (CCDF) of the lognormal distribution at each of these thirty intensity levels, for the set of four damage states. Now, the calculation proceeds in the same manner as described in Case~1a above.
\begin{table}[htbp]

\centering
\begin{tabular}{ l l r r r }

\hline
\rowcolor{anti-flashwhite}
\bf{Asset} & \bf{Damage State} & \bf{Expected} & \bf{OpenQuake} & \bf{Difference}\\
\hline
a1 & none & $9.88 \times 10^{-1}$ & $9.88 \times 10^{-1}$ & 0\% \\
   & ds1 & $7.49 \times 10^{-3}$ & $7.49 \times 10^{-3}$ & 0\% \\
   & ds2 & $2.43 \times 10^{-3}$ & $2.43 \times 10^{-3}$ & 0\% \\
   & ds3 & $9.88 \times 10^{-4}$ & $9.88 \times 10^{-4}$ & 0\% \\
   & ds4 & $1.06 \times 10^{-3}$ & $1.06 \times 10^{-3}$ & 0\% \\
\hline
\end{tabular}

\caption{Results for classical damage test case 1b}
\label{tab:result-cd-1b}
\end{table}

Table \ref{tab:result-cd-1b} shows the comparison of the OpenQuake damage state probabilities with the expected results.


% -----------------------------------------------------------------------------
\subsection{Multiple asset tests}
\label{subsec:cd-multiple}
The multiple asset test cases are designed to test the correct working of the classical risk calculator for a portfolio of assets of different taxonomies. The vulnerability functions for the different taxonomies are based on different intensity measure types.

The list of assets in the exposure model used for the multiple-asset test cases for the classical risk calculator is given in Table \ref{tab:assets}.

\subsubsection{Case 6a}


\begin{table}[htbp]

\centering
\begin{tabular}{ l l l r r r }

\hline
\rowcolor{anti-flashwhite}
\bf{Asset} & \bf{Damage State} & \bf{Result} & \bf{Expected} & \bf{OpenQuake} & \bf{Difference}\\
\hline
a1 & No damage & Probability of occurrence &  &  & \% \\
   & Slight    & Probability of occurrence &  &  & \% \\
   & Moderate  & Probability of occurrence &  &  & \% \\
   & Extreme   & Probability of occurrence &  &  & \% \\
   & Complete  & Probability of occurrence &  &  & \% \\
\hline
a2 & No damage & Probability of occurrence &  &  & \% \\
   & Slight    & Probability of occurrence &  &  & \% \\
   & Moderate  & Probability of occurrence &  &  & \% \\
   & Extreme   & Probability of occurrence &  &  & \% \\
   & Complete  & Probability of occurrence &  &  & \% \\
\hline
a3 & No damage & Probability of occurrence &  &  & \% \\
   & Slight    & Probability of occurrence &  &  & \% \\
   & Moderate  & Probability of occurrence &  &  & \% \\
   & Extreme   & Probability of occurrence &  &  & \% \\
   & Complete  & Probability of occurrence &  &  & \% \\
\hline
a4 & No damage & Probability of occurrence &  &  & \% \\
   & Slight    & Probability of occurrence &  &  & \% \\
   & Moderate  & Probability of occurrence &  &  & \% \\
   & Extreme   & Probability of occurrence &  &  & \% \\
   & Complete  & Probability of occurrence &  &  & \% \\
\hline
a5 & No damage & Probability of occurrence &  &  & \% \\
   & Slight    & Probability of occurrence &  &  & \% \\
   & Moderate  & Probability of occurrence &  &  & \% \\
   & Extreme   & Probability of occurrence &  &  & \% \\
   & Complete  & Probability of occurrence &  &  & \% \\
\hline
a6 & No damage & Probability of occurrence &  &  & \% \\
   & Slight    & Probability of occurrence &  &  & \% \\
   & Moderate  & Probability of occurrence &  &  & \% \\
   & Extreme   & Probability of occurrence &  &  & \% \\
   & Complete  & Probability of occurrence &  &  & \% \\
\hline
a7 & No damage & Probability of occurrence &  &  & \% \\
   & Slight    & Probability of occurrence &  &  & \% \\
   & Moderate  & Probability of occurrence &  &  & \% \\
   & Extreme   & Probability of occurrence &  &  & \% \\
   & Complete  & Probability of occurrence &  &  & \% \\
\hline
\end{tabular}

\caption{Results for classical damage test case 7a}
\label{tab:result-classical-damage-7a}
\end{table}

Table \ref{tab:result-cd-6a} shows the comparison of the OpenQuake result with the expected result.

% -----------------------------------------------------------------------------
\subsection{Calculation with logic-trees}
\label{subsec:cd-lt}
\subsubsection{Case 4a}
The OpenQuake scenario damage calculator allows the user to employ more than one ground motion prediction equation (GMPE) for computing the ground motion fields used for the loss calculation. The mean and standard deviation of the damage state probabilities (or fractions), are calculated and output for each GMPE branch independently. No sampling is involved, and any branch weights assigned to the different GMPE branches are ignored.

A single asset is used in this test case. The two ground motion prediction equations used are Boore and Atkinson (2008), and Chiou and Youngs (2008). Table~\ref{tab:ff-disc-tax1-zndl} shows the parameters of the discrete fragility function used in this test case. Two hazard curves are computed, one for each of the GMPE branches specified in the logic tree. The procedure used for calculating the damage state probabilities for the asset remains the same as described in Case~1a. The damage state probabilities are computed individually for each of the two hazard branches.
\begin{table}[htbp]

\centering
\begin{tabular}{ l l l r r r }

\hline
\rowcolor{anti-flashwhite}
\bf{Asset} & \bf{Damage State} & \bf{Result} & \bf{Expected} & \bf{OpenQuake} & \bf{Difference}\\
\hline
a1 & No damage & Probability of occurrence & $9.71 \times 10^{-1}$ & $9.71 \times 10^{-1}$ & 0\% \\
   & Slight    & Probability of occurrence & $4.43 \times 10^{-3}$ & $4.43 \times 10^{-3}$ & 0\% \\
   & Moderate  & Probability of occurrence & $8.41 \times 10^{-3}$ & $8.41 \times 10^{-3}$ & 0\% \\
   & Extreme   & Probability of occurrence & $5.29 \times 10^{-3}$ & $5.29 \times 10^{-3}$ & 0\% \\
   & Complete  & Probability of occurrence & $1.10 \times 10^{-2}$ & $1.10 \times 10^{-2}$ & 0\% \\
\hline
\end{tabular}

\caption{Results for classical damage test case 4a}
\label{tab:result-classical-damage-4a}
\end{table}

Table~\ref{tab:result-cd-4a} shows the comparison of the OpenQuake results for the damage distribution with the expected results for the two GMPE branches used in this case.