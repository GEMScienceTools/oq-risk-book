The tests for the scenario damage calculator assume the correct computation of the ground motion fields at the locations of the assets in the exposure model. Thus, the risk quality assurance tests implicitly rely on the acceptance tests for the scenario hazard calculator.

The rupture model used for the tests comprises a magnitude $M6.7$ rupture on a vertical strike-slip fault, the same as is used in the tests for the scenario risk calculator.\\

\noindent Details of the rupture are repeated below for convenience:\\

\noindent
Fault type: Strike slip\\
Fault dip: $90^{\circ}$\\
Fault plane depths: 0--20 km\\
Fault coordinates:\\
South end: $38.0000^{\circ} N$, $122.0000^{\circ} W$\\
North end: $38.2248^{\circ} N$, $122.0000^{\circ} W$\\
Rupture magnitude: 6.7\\
Rupture hypocenter: $38.1124^{\circ} N$, $122.0000^{\circ} W$\\
Hypocenter depth: 10 km\\

The complete collection of input models and job configuration files used in these test cases can be accessed here:
\href{https://github.com/gem/oq-risklib/tree/master/openquake/qa_tests_data/scenario_damage}
{https://github.com/gem/oq-risklib/tree/master/openquake/qa\_tests\_data/scenario\_damage}

\subsection{Single asset tests}
\label{subsec:acc-sd-single}
The single asset test cases are designed to test the basic elements of the scenario damage calculator, such as:

\begin{itemize}
\item interpolation of the discrete fragility functions
\item damage distribution computation for a given set of ground motion fields
\item extraction of the probability of collapse
\end{itemize}

The location and taxonomy of the single asset in the exposure model used for the single-asset test cases for the scenario risk calculator are given in Table~\ref{tab:asset}.

% ---------------------------------------------------------------------------
\subsubsection{Case 1a}
Test case 1a uses a set of five precomputed ground motion values to test the correct interpolation of the damage state exceedance probabilities of the discrete fragility function at intermediate intensity measure levels.

Table~\ref{tab:gmfs-diff-l1-5} lists the five ground motion values used in this test case.

\begin{table}[htbp]

\centering
\begin{tabular}{ c c c c c c c c c c }

\hline
\rowcolor{anti-flashwhite}
\bf{LS | PGA} & \bf{0.2g} & \bf{0.4g} & \bf{0.6g} & \bf{0.8g} & \bf{1.0g} & \bf{1.2g} & \bf{1.4g} & \bf{\dots} & \bf{5.0g} \\
\hline
\bf{ds1} & 0.000 & 0.152 & 0.846 & 0.993 & 1.000 & 1.000 & 1.000 & \dots & 1.000 \\
\bf{ds2} & 0.000 & 0.014 & 0.129 & 0.350 & 0.576 & 0.747 & 0.857 & \dots & 1.000 \\
\bf{ds3} & 0.000 & 0.008 & 0.085 & 0.196 & 0.325 & 0.450 & 0.561 & \dots & 0.993 \\
\bf{ds4} & 0.000 & 0.006 & 0.067 & 0.171 & 0.263 & 0.354 & 0.438 & \dots & 0.951 \\
\hline
\end{tabular}

\caption{Discrete fragility function with zero no damage limit}
\label{tab:ff-disc-tax1-zndl}
\end{table}

Table~\ref{tab:ff-disc-tax1-zndl} shows the set of ground motion intensity levels and corresponding probabilities of exceedance for the four damage states for the discrete fragility function used in this test case.

The ground motion values at the location of the single asset are $[1.3, 0.044, 0.52, 1.0, 1.2] g$. Consider the first value of $PGA = 1.3 g$. The discrete fragility function for this case provides damage state probabilities of exceedance at intensity measure levels $1.2 g$ and $1.4 g$, but none at $1.3 g$. The exceedance probabilities at $1.2 g$ and $1.4 g$ corresponding to the discrete damage states $[ds_1, ds_2, ds_3, ds_4]$ are $[1.000, 0.747, 0.450, 0.354]$ and $[1.000, 0.857, 0.561, 0.438]$ respectively.

The exceedance probabilities at $1.3 g$ are obtained by interpolating between these two sets of values. Linear interpolation gives exceedance probabilities of $[1.000, 0.802, 0.5055, 0.396]$ for $PGA = 1.3 g$. The probabilities of damage state occurrence are given by the pairwise differences of the exceedance probabilities as $[1.000 - 0.802, 0.802 - 0.5055, 0.5055 - 0.396, 0.396] = [0.198, 0.2965, 0.1095, 0.396]$. These four damage state probabilities sum up to one, indicating that the probability of observing no damage is zero.

Similar interpolation at the other four ground motion intensity levels gives the following sets of damage state probabilities:

\begin{itemize}
	\item GMF1: $[0.000, 0.198, 0.2965, 0.1095, 0.396]$
	\item GMF2: $[1.000, 0.000, 0.000, 0.000, 0.000]$
	\item GMF3: $[0.4316, 0.4854, 0.0288, 0.0116, 0.0426]$
	\item GMF4: $[0.000, 0.424, 0.251, 0.062, 0.263]$
	\item GMF5: $[0.000, 0.253, 0.297, 0.096, 0.354]$
\end{itemize}

The mean and standard deviation of the four damage state probabilities and also the probability of observing no damage is now calculated using the above set of probabilities collected from each of the five ground motion simulations.
\begin{table}[htbp]

\centering
\begin{tabular}{ l l l r r r }

\hline
\rowcolor{anti-flashwhite}
\bf{Asset} & \bf{Damage State} & \bf{Result} & \bf{Expected} & \bf{OpenQuake} & \bf{Difference}\\
\hline
a1 & none & Mean & 0.2863 & 0.2863 & 0.00\% \\
   &      & Std. & 0.4406 & 0.4406 & 0.00\% \\
   & ds1  & Mean & 0.2721 & 0.2721 & 0.00\% \\
   &      & Std. & 0.1927 & 0.1927 & 0.00\% \\
   & ds2  & Mean & 0.1747 & 0.1747 & 0.00\% \\
   &      & Std. & 0.1478 & 0.1478 & 0.00\% \\
   & ds3  & Mean & 0.0558 & 0.0558 & 0.00\% \\
   &      & Std. & 0.0490 & 0.0490 & 0.00\% \\
   & ds4  & Mean & 0.2111 & 0.2111 & 0.00\% \\
   &      & Std. & 0.1805 & 0.1805 & 0.00\% \\
\hline
\end{tabular}

\caption{Results for scenario damage test case 1a}
\label{tab:result-sd-1a}
\end{table}
Table \ref{tab:result-sd-1a} shows the comparison of the OpenQuake result with the expected result.

% ---------------------------------------------------------------------------
\subsubsection{Case 1b}
Whereas the previous case was concerned with checking the correct implementation and usage of \emph{discrete}discrete fragility functions, the purpose of this case is to verify the correct calculation of damage distribution statistics for a scenario using \emph{continuous} (lognormal CDF) fragility functions.

\begin{table}[htbp]

\centering
\begin{tabular}{ c c c }

\hline
\rowcolor{anti-flashwhite}
\bf{LS} & \bf{Mean IML} & \bf{Std. IML} \\
\hline
\bf{ds1} & 0.50 & 0.40 \\
\bf{ds2} & 1.00 & 0.80 \\
\bf{ds3} & 1.50 & 1.20 \\
\bf{ds4} & 2.00 & 1.60 \\
\hline
\end{tabular}

\caption{Fragility function with zero no damage limit}
\label{tab:ff-cont-tax1-zmin}
\end{table}

Table~\ref{tab:ff-cont-tax1-zmin} shows the mean and standard deviation of the ground motion intensity level for the four damage states, which are the parameters for the lognormal fragility function used in this test case.

The set of five precomputed ground motion values described in Table~\ref{tab:gmfs-diff-l1-5}  are used in this case. The ground motion values at the location of the single asset are $[1.3, 0.044, 0.52, 1.0, 1.2] g$. Consider the first value of $PGA = 1.3 g$. The exceedance probability for damage state $ds_1$ is obtained by employing the equation for the cumulative distribution function for the lognormal distribution, using the mean, $m_1 = 0.5 g$, and standard deviation, $s_1 = 0.4 g$, specified by the fragility function for that damage state. The equations are given below:

\begin{equation}
	\mu_1 = \ln \left( \frac{m_1}{\sqrt{1 + \frac{s^2}{m^2}}} \right) = -0.940
\end{equation}

\begin{equation}
	\sigma_1 = \sqrt{\ln \left( 1 + \frac{s^2}{m^2} \right)} = 0.703
\end{equation}

\begin{equation}
	p.o.e.(ds_1) = \frac{1}{2} + \frac{1}{2} erf \left[ \frac{\ln 1.3 - \mu_1}{\sqrt{2} \sigma_1} \right] = 0.956
\end{equation}

Next, the exceedance probabilities for the other damage states for the first ground motion are obtained in a similar manner. We have, for the first ground motion:

\begin{itemize}
	\item $p.o.e.(ds_1) = 0.956$
	\item $p.o.e.(ds_2) = 0.766$
	\item $p.o.e.(ds_3) = 0.559$
	\item $p.o.e.(ds_4) = 0.397$
\end{itemize}

Thus we have exceedance probabilities of $[0.956, 0.766, 0.559, 0.397]$ for $PGA = 1.3 g$. The probabilities of damage state occurrence are given by the pairwise differences of the exceedance probabilities as $[0.956 - 0.766, 0.766 - 0.559, 0.559 - 0.397, 0.397] = [0.190, 0.207, 0.162, 0.397]$. The probability of observing no damage is the remainder of the probability after summing up the probabilities for the four damage states, i.e., $1.0 - (0.190 + 0.207 + 0.162 + 0.397) = 0.044$.

This procedure is repeated for the other four ground motion fields to give the following sets of damage state probabilities:

\begin{itemize}
	\item GMF1: $[0.0436, 0.191, 0.207, 0.162, 0.397]$
	\item GMF2: $[0.9991, 0.0009, 0.000, 0.000, 0.000]$
	\item GMF3: $[0.342, 0.376, 0.157, 0.0652, 0.059]$
	\item GMF4: $[0.091, 0.272, 0.226, 0.148, 0.263]$
	\item GMF5: $[0.055, 0.215, 0.215, 0.160, 0.354]$
\end{itemize}

The mean and standard deviation of the four damage state probabilities and also the probability of observing no damage is now calculated using the above set of probabilities collected from each of the five ground motion simulations.
\begin{table}[htbp]

\centering
\begin{tabular}{ l l l r r r }

\hline
\rowcolor{anti-flashwhite}
\bf{Asset} & \bf{Damage State} & \bf{Result} & \bf{Expected} & \bf{OpenQuake} & \bf{Difference}\\
\hline

\hline
\end{tabular}

\caption{Results for scenario damage test case 1b}
\label{tab:result-sd-1b}
\end{table}
Table \ref{tab:result-sd-1b} shows the comparison of the OpenQuake result with the expected result.

% ---------------------------------------------------------------------------
\subsubsection{Case 1c}
Test Case~1c repeats the exercise from Case~1a, with the difference that the discrete fragility function specifies a minimum ground motion intensity, below which the probability of exceedance for all damage states is assumed to be zero.

\begin{table}[htbp]

\centering
\begin{tabular}{ l c c l }

\hline
\rowcolor{anti-flashwhite}
\bf{GMF \#} & \bf{Site} & \bf{PGA (g)}\\
\hline
1 & 1 & 1.300 \\
2 & 1 & 0.044 \\
3 & 1 & 0.220 \\
4 & 1 & 1.000 \\
5 & 1 & 1.200 \\
\hline
\end{tabular}

\caption{Five precomputed ground motion fields at a single site}
\label{tab:gmfs-diff-l1-5b}
\end{table}

Table~\ref{tab:gmfs-diff-l1-5b} lists the five ground motion values used in this test case, and Table~\ref{tab:ff-disc-tax1-zndl} the discrete fragility function used in this test case. The "no damage limit" is specified to be $0.3 g$.

The ground motion values at the location of the single asset are $[1.3, 0.044, 0.22, 1.0, 1.2] g$. The calculation of the damage state exceedance probabilities proceeds in exactly the same manner as demonstrated in Case~1a, except that for the ground motion values of $0.044 g$ and $0.22 g$, the probabilities for all four damage states are assumed to be zero.

We have the following sets of damage state probabilities:

\begin{itemize}
	\item GMF1: $[0.000, 0.198, 0.2965, 0.1095, 0.396]$
	\item GMF2: $[1.000, 0.000, 0.000, 0.000, 0.000]$
	\item GMF3: $[1.000, 0.000, 0.000, 0.000, 0.000]$
	\item GMF4: $[0.000, 0.424, 0.251, 0.062, 0.263]$
	\item GMF5: $[0.000, 0.253, 0.297, 0.096, 0.354]$
\end{itemize}

The mean and standard deviation of the four damage state probabilities and also the probability of observing no damage is now calculated using the above set of probabilities collected from each of the five ground motion simulations.
\begin{table}[htbp]

\centering
\begin{tabular}{ l l l r r r }

\hline
\rowcolor{anti-flashwhite}
\bf{Asset} & \bf{Damage State} & \bf{Result} & \bf{Expected} & \bf{OpenQuake} & \bf{Difference}\\
\hline
a1 & none & Mean & 0.4000 & 0.4000 & 0.00\% \\
   &      & Std. & 0.5477 & 0.5477 & 0.00\% \\
   & ds1  & Mean & 0.1750 & 0.1750 & 0.00\% \\
   &      & Std. & 0.1802 & 0.1802 & 0.00\% \\
   & ds2  & Mean & 0.1689 & 0.1689 & 0.00\% \\
   &      & Std. & 0.1553 & 0.1553 & 0.00\% \\
   & ds3  & Mean & 0.0535 & 0.0535 & 0.00\% \\
   &      & Std. & 0.0518 & 0.0518 & 0.00\% \\
   & ds4  & Mean & 0.2026 & 0.2026 & 0.00\% \\
   &      & Std. & 0.1911 & 0.1911 & 0.00\% \\
\hline
\end{tabular}

\caption{Results for scenario damage test case 1c}
\label{tab:result-sd-1c}
\end{table}
Table \ref{tab:result-sd-1c} shows the comparison of the OpenQuake result with the expected result.

% ---------------------------------------------------------------------------
\subsubsection{Case 1d}
Test Case~1d repeats the exercise from Case~1b, with the difference that the continuous fragility function specifies a minimum ground motion intensity, below which the probability of exceedance for all damage states is assumed to be zero.

Table~\ref{tab:gmfs-diff-l1-5b} lists the five ground motion values used in this test case, and Table~\ref{tab:ff-cont-tax1-zmin} the continuous fragility function used in this test case. The "minimum intensity level" is specified to be $0.3 g$.

The ground motion values at the location of the single asset are $[1.3, 0.044, 0.22, 1.0, 1.2] g$. The calculation of the damage state exceedance probabilities proceeds in exactly the same manner as demonstrated in Case~1a, except that for the ground motion values of $0.044 g$ and $0.22 g$, the probabilities for all four damage states are assumed to be zero.

We have the following sets of damage state probabilities:

\begin{itemize}
	\item GMF1: $[0.0436, 0.191, 0.207, 0.162, 0.397]$
	\item GMF2: $[1.000, 0.000, 0.000, 0.000, 0.000]$
	\item GMF3: $[1.000, 0.000, 0.000, 0.000, 0.000]$
	\item GMF4: $[0.091, 0.272, 0.226, 0.148, 0.263]$
	\item GMF5: $[0.055, 0.215, 0.215, 0.160, 0.354]$
\end{itemize}

The mean and standard deviation of the four damage state probabilities and also the probability of observing no damage is now calculated using the above set of probabilities collected from each of the five ground motion simulations.

\begin{table}[htbp]

\centering
\begin{tabular}{ l l l r r r }

\hline
\rowcolor{anti-flashwhite}
\bf{Asset} & \bf{Damage State} & \bf{Result} & \bf{Expected} & \bf{OpenQuake} & \bf{Difference}\\
\hline
a1 & none & Mean & 0.4379 & 0.4379 & 0.00\% \\
   &      & Std. & 0.5134 & 0.5134 & 0.00\% \\
   & ds1  & Mean & 0.1356 & 0.1356 & 0.00\% \\
   &      & Std. & 0.1272 & 0.1272 & 0.00\% \\
   & ds2  & Mean & 0.1296 & 0.1296 & 0.00\% \\
   &      & Std. & 0.1185 & 0.1185 & 0.00\% \\
   & ds3  & Mean & 0.0940 & 0.0940 & 0.00\% \\
   &      & Std. & 0.0860 & 0.0860 & 0.00\% \\
   & ds4  & Mean & 0.2028 & 0.2028 & 0.00\% \\
   &      & Std. & 0.1913 & 0.1913 & 0.00\% \\
\hline
\end{tabular}

\caption{Results for scenario damage test case 1d}
\label{tab:result-sd-1d}
\end{table}
Table \ref{tab:result-sd-1d} shows the comparison of the OpenQuake result with the expected result.
% ---------------------------------------------------------------------------
\subsubsection{Case 1e}
The purpose of this case is to test the computation of the mean and standard deviation of the damage state probabilities, given variability in both the ground motion values. Ten thousand ground motion fields are generated for the given rupture, taking into consideration both the inter-event and intra-event variability in the ground motion. The ground motion prediction equation used is Boore and Atkinson (2008).

Table \ref{tab:gmfs-sim-l1-10000} lists five of the ten thousand ground motion values generated by OpenQuake. Table~\ref{tab:ff-disc-tax1-zndl} shows the set of ground motion intensity levels and corresponding probabilities of exceedance for the four damage states for the discrete fragility function used in this test case.

In order to check the OpenQuake results, an alternate implementation of the calculator algorithm in the programming language Julia is used for comparison. In order to provide a representative baseline for the comparison, one million ground motion fields are used in the Julia calculation.

The mean and standard deviation of the logarithm of the ground motion calculated at the location of the asset as obtained by using the Boore and Atkinson (2008) equation are $-0.648$ and $0.564$ respectively. Assuming a lognormal distribution for the variability in the ground motion, one million ground motion values are generated using Julia with these logarithmic mean and standard deviation values.

For each simulated ground motion value, the damage state exceedance probabilities are obtained through interpolation on the discrete fragility function specified for this case. The probabilities of damage state occurrence are calculated by the pairwise differences of the exceedance probabilities of adjacent damage states as described in Case~1a.

The mean and standard deviation of the damage state probabilities and also the probability of observing no damage are finally calculated using the above sets of probabilities collected from each of the one million ground motion simulations.
\begin{table}[htbp]

\centering
\begin{tabular}{ l l l r r r }

\hline
\rowcolor{anti-flashwhite}
\bf{Asset} & \bf{Damage State} & \bf{Result} & \bf{Julia} & \bf{OpenQuake} & \bf{Difference}\\
\hline
a1 & none & Mean & 0.4550 & 0.4575 & -0.53\% \\
   &      & Std. & 0.3934 & 0.3931 & 0.07\% \\
   & ds1  & Mean & 0.3400 & 0.3401 & -0.02\% \\
   &      & Std. & 0.2512 & 0.2520 & -0.31\% \\
   & ds2  & Mean & 0.0775 & 0.0767 & 0.97\% \\
   &      & Std. & 0.0949 & 0.0945 & 0.41\% \\
   & ds3  & Mean & 0.0240 & 0.0238 & 0.94\% \\
   &      & Std. & 0.0343 & 0.0342 & 0.16\% \\
   & ds4  & Mean & 0.1036 & 0.1020 & 1.47\% \\
   &      & Std. & 0.1405 & 0.1384 & 1.50\% \\
\hline
\end{tabular}

\caption{Results for scenario damage test case 1e}
\label{tab:result-sd-1e}
\end{table}
Table \ref{tab:result-sd-1e} shows the comparison of the OpenQuake result with the expected result.
% ---------------------------------------------------------------------------
\subsubsection{Case 1f}
This test case is identical to Case~1e described above, except for the use of a continous fragility function (see Table~\ref{tab:ff-cont-tax1-zmin}) instead of the discrete fragility function used in the previous case.
\begin{table}[htbp]

\centering
\begin{tabular}{ l l l r r r }

\hline
\rowcolor{anti-flashwhite}
\bf{Asset} & \bf{Damage State} & \bf{Result} & \bf{Julia} & \bf{OpenQuake} & \bf{Difference}\\
\hline
a1 & none & Mean & 0.3729 & 0.3736 & -0.19\% \\
   &      & Std. & 0.2419 & 0.2403 & 0.66\% \\
   & ds1  & Mean & 0.2988 & 0.2998 & -0.31\% \\
   &      & Std. & 0.0837 & 0.0824 & 1.56\% \\
   & ds2  & Mean & 0.1427 & 0.1426 & 0.05\% \\
   &      & Std. & 0.0676 & 0.0673 & 0.43\% \\
   & ds3  & Mean & 0.0731 & 0.0728 & 0.32\% \\
   &      & Std. & 0.0504 & 0.0503 & 0.26\% \\
   & ds4  & Mean & 0.1125 & 0.1112 & 1.19\% \\
   &      & Std. & 0.1367 & 0.1345 & 1.61\% \\
\hline
\end{tabular}

\caption{Results for scenario damage test case 1f}
\label{tab:result-sd-1f}
\end{table}
Table \ref{tab:result-sd-1f} shows the comparison of the OpenQuake result with the expected result.
% ---------------------------------------------------------------------------
\subsubsection{Case 1g}
Case~1g is designed to test a simple multiplication of the mean and standard deviation of damage state probabilities by the number of units comprising an asset. When an asset comprises more than one unit, the Scenario Damage calculator returns the expected fraction of buildings in each damage state, and the corresponding standard deviation. This case is thus designed identical to Case~1f, except that the single asset used in this case comprises three units, instead of one as in the previous case. The expected results should be three times those obtained in Case~1f.
\begin{table}[htbp]

\centering
\begin{tabular}{ l l l r r r }

\hline
\rowcolor{anti-flashwhite}
\bf{Asset} & \bf{Damage State} & \bf{Result} & \bf{Expected} & \bf{OpenQuake} & \bf{Difference}\\
\hline

\hline
\end{tabular}

\caption{Results for scenario damage test case 1g}
\label{tab:result-sd-1g}
\end{table}
Table \ref{tab:result-sd-1g} shows the comparison of the OpenQuake result with the expected result.

% -----------------------------------------------------------------------------
\subsection{Multiple asset tests}
\label{subsec:acc-sd-multiple}
The multiple asset test cases are designed to test the loss aggregation functions of the scenario damage calculator, such as:

\begin{itemize}
\item damage distribution per taxonomy
\item damage distribution for the portfolio
\end{itemize}

The list of assets in the exposure model used for the multiple-asset test cases for the scenario damage calculator is given in Table \ref{tab:assets}.






% -----------------------------------------------------------------------------
\subsection{Calculation with logic-trees}
\label{subsec:acc-sd-lt}
% ---------------------------------------------------------------------------
\subsubsection{Case 3a}
The OpenQuake scenario damage calculator allows the user to employ more than one ground motion prediction equation (GMPE) for computing the ground motion fields used for the loss calculation. The mean and standard deviation of the damage state probabilities (or fractions), are calculated and output for each GMPE branch independently. No sampling is involved, and any branch weights assigned to the different GMPE branches are ignored.

A single asset is used in this test case. Table~\ref{tab:ff-cont-tax1-zmin} shows the parameters of the continuous lognormal fragility function used in this test case.

The two ground motion prediction equations used are Boore and Atkinson (2008), and Chiou and Youngs (2008). Ten thousand ground motion fields are generated using OpenQuake for the given rupture, taking into consideration both the inter-event and intra-event variability in the ground motion, for both GMPE branches. The rest of the loss calculation procedure follows the same steps as described earlier in Case~1e.

For comparison, one million ground motion fields are generated using both GMPEs using Julia.
\begin{table}[htbp]

\centering
\begin{tabular}{ l l l l r r r }

\hline
\rowcolor{anti-flashwhite}
\bf{Branch} & \bf{Asset} & \bf{Damage State} & \bf{Result} & \bf{Julia} & \bf{OpenQuake} & \bf{Difference}\\
\hline
BA2008 & a1 & none & Mean & 0.3725 & 0.3736 & -0.30\% \\
       &    &      & Std. & 0.2420 & 0.2403 & 0.69\% \\
       &    & ds1  & Mean & 0.2988 & 0.2998 & -0.31\% \\
       &    &      & Std. & 0.0838 & 0.0824 & 1.64\% \\
       &    & ds2  & Mean & 0.1428 & 0.1426 & 0.15\% \\
       &    &      & Std. & 0.0676 & 0.0673 & 0.44\% \\
       &    & ds3  & Mean & 0.0732 & 0.0728 & 0.47\% \\
       &    &      & Std. & 0.0504 & 0.0503 & 0.27\% \\
       &    & ds4  & Mean & 0.1127 & 0.1112 & 1.33\% \\
       &    &      & Std. & 0.1367 & 0.1345 & 1.56\% \\
\hline
CY2008 & a1 & none & Mean & 0.3086 & 0.3093 & -0.25\% \\
       &    &      & Std. & 0.2218 & 0.2197 & 0.92\% \\
       &    & ds1  & Mean & 0.3019 & 0.3032 & -0.45\% \\
       &    &      & Std. & 0.0814 & 0.0799 & 1.92\% \\
       &    & ds2  & Mean & 0.1590 & 0.1590 & -0.02\% \\
       &    &      & Std. & 0.0617 & 0.0613 & 0.61\% \\
       &    & ds3  & Mean & 0.0862 & 0.0859 & 0.34\% \\
       &    &      & Std. & 0.0500 & 0.0498 & 0.40\% \\
       &    & ds4  & Mean & 0.1444 & 0.1426 & 1.30\% \\
       &    &      & Std. & 0.1535 & 0.1512 & 1.53\% \\
\hline
\end{tabular}

\caption{Results for scenario damage test case 3a}
\label{tab:result-sd-3a}
\end{table}

Table \ref{tab:result-sd-3a} shows the comparison of the OpenQuake results with the expected results.
