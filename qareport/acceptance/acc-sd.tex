The tests for the scenario damage calculator assume the correct computation of the ground motion fields at the locations of the assets in the exposure model. Thus, the risk quality assurance tests implicitly rely on the acceptance tests for the scenario hazard calculator.

The rupture model used for the tests comprises a magnitude $M6.7$ rupture on a vertical strike-slip fault.\\

\noindent Details of the rupture are given below:\\

\noindent
Fault type: Strike slip\\
Fault dip: $90^{\circ}$\\
Fault plane depths: 0--20 km\\
Fault coordinates:\\
South end: $38.0000^{\circ} N$, $122.0000^{\circ} W$\\
North end: $38.2248^{\circ} N$, $122.0000^{\circ} W$\\
Rupture magnitude: 6.7\\
Rupture hypocenter: $38.1124^{\circ} N$, $122.0000^{\circ} W$\\
Hypocenter depth: 10 km\\

The complete collection of input models and job configuration files used in these test cases can be accessed here:
\href{https://github.com/gem/oq-risklib/tree/master/openquake/qa_tests_data/scenario_damage}
{https://github.com/gem/oq-risklib/tree/master/openquake/qa\_tests\_data/scenario\_damage}

\subsection{Single asset tests}
\label{subsec:acc-sd-single}
The single asset test cases are designed to test the basic elements of the scenario damage calculator, such as:

\begin{itemize}
\item loss field computation for a given set of ground motion fields
\item calculation of mean and standard deviation of scenario loss
\end{itemize}

The location and taxonomy of the single asset in the exposure model used for the single-asset test cases for the scenario risk calculator are given in Table \ref{tab:asset}.

% ---------------------------------------------------------------------------
\subsubsection{Case 1a}
Test case 1a uses a set of 10 identical ground motion values.

The purpose of this case is to test the accurate computation of the probability of occurence of each damage state, given zero variability in the ground motion values. This case exercises the interpolation of discrete fragility functions. Hand calculations are used to verify the correctness of the results obtained.

% \begin{table}[htbp]

\centering
\begin{tabular}{ l c c l }

\hline
\rowcolor{anti-flashwhite}
\bf{GMF \#} & \bf{Site} & \bf{IMT} & \bf{GMV}\\
\hline
1 & 1 & PGA & 0.5000 \\
2 & 1 & PGA & 0.5000 \\
3 & 1 & PGA & 0.5000 \\
4 & 1 & PGA & 0.5000 \\
\vdots & \vdots & \vdots & \vdots \\
10 & 1 & PGA & 0.5000 \\
\hline
\end{tabular}

\caption{Ground motion fields for the test cases 1a and 1b}
\label{tab:scenario-gmfs-identical-10}
\end{table}

% Table \ref{tab:scenario-gmfs-identical-10} lists five of the ten ground motion values used in this test case.

\begin{table}[htbp]

\centering
\begin{tabular}{ l c c c c c c c c c c c}

\hline
\rowcolor{anti-flashwhite}
\bf{LS | PGA} & \bf{0.2} & \bf{0.4} & \bf{0.6} & \bf{0.8} & \bf{1.0} & \bf{1.2} & \bf{1.4} & \bf{\dots} & \bf{5.0} \\
\hline
\bf{ds1} & 0.000 & 0.152 & 0.846 & 0.993 & 1.000 & 1.000 & 1.000 & \dots & 1 \\
\bf{ds2} & 0.000 & 0.014 & 0.129 & 0.35 & 0.576 & 0.747 & 0.857 & \dots & 1.000 \\
\bf{ds3} & 0.000 & 0.008 & 0.085 & 0.196 & 0.325 & 0.45 & 0.561 & \dots & 0.993 \\
\bf{ds4} & 0.000 & 0.006 & 0.067 & 0.171 & 0.263 & 0.354 & 0.438 & \dots & 0.951 \\
\hline
\end{tabular}

\caption{Discrete fragility function with zero no damage limit}
\label{tab:ff-disc-zndl}
\end{table}

Table \ref{tab:ff-disc-zndl} shows the mean loss ratios and corresponding coefficients of variation in the vulnerability function used in this test case.

Since all ground motion values are identical, calculation of the damage distribution is straightforward in this case. The ground motion value at the location of the single asset is $PGA = 0.5 g$. The discrete fragility function for this case provides damage state probabilities of exceedance at intensity measure levels $0.4 g$ and $0.6 g$, but none at $0.5 g$. The exceedance probabilities at $0.4 g$ and $0.6 g$ corresponding to the discrete damage states $[ds_1, ds_2, ds_3, ds_4]$ are $[0.152, 0.014, 0.008, 0.006]$ and $[0.846, 0.129, 0.085, 0.067]$ respectively.

The exceedance probabilities at $0.5 g$ are obtained by interpolating between these two sets of values. Linear interpolation gives exceedance probabilities of $[0.499, 0.0715, 0.08225, 0.0365]$ for $PGA = 0.5 g$. The probabilities of damage state occurrence are given by the pairwise differences of the exceedance probabilities as $[0.4990 - 0.0715, 0.0715 - 0.08225, 0.08225 - 0.0365, 0.0365]$

Since there is no variability in the ground motion, the mean loss ratio is also $0.15$, and the standard deviation of the loss ratio is $0.0$.

These numbers are multiplied by the asset value of $10,000$ to give the mean and standard deviation of loss for the scenario as $1,500$ and $0$ respectively.

\input{qareport/results/tab-result-scenario-damage-1a}

Table \ref{tab:result-scenario-damage-1a} shows the comparison of the OpenQuake result with the expected result.

% ---------------------------------------------------------------------------
\subsubsection{Case 1b}
Test case 1b uses the same set of identical ground motion values as case 1a, and described in Table \ref{tab:scenario-gmfs-identical-10}. However, in contrast to case 1a, variability in the loss ratio \emph{is} considered in the vulnerability function for this case.

\input{qareport/tables/tab-ff-cont-zmin}

Table \ref{tab:ff-cont-zmin} shows the mean loss ratios and corresponding coefficients of variation in the vulnerability function used in this test case.

Similar to case 1a described above, linear interpolation gives a mean loss ratio of $0.15$ for $PGA = 0.5 g$. The vulnerability function for this case provides coefficients of variation for the loss ratio at intensity measure levels $0.4 g$ and $0.6 g$, but none at $0.5 g$. The CoVs of the loss ratio at $0.4 g$ and $0.6 g$ are $0.24$ and $0.32$ respectively. The coefficient of variation of the loss ratio for $PGA = 0.5 g$ is thus obtained by linear interpolation as $0.28$.

The loss ratio at $PGA = 0.5 g$ follows a lognormal distribution with a mean of $0.15$ and a standard deviation of $0.28 \times 0.15 = 0.042$.

Since there is no variability in the ground motion, the mean loss ratio for the scenario is also $0.15$, and the standard deviation of the loss ratio is $0.042$.

These numbers are multiplied by the asset value of $10,000$ to give the mean and standard deviation of loss for the scenario as $1,500$ and $420$ respectively.

\input{qareport/results/tab-result-scenario-damage-1b}

Table \ref{tab:result-scenario-damage-1b} shows the comparison of the OpenQuake result with the expected result.

% ---------------------------------------------------------------------------
\subsubsection{Case 1c}
Variability in the ground motion is considered in all cases starting from case 1c. Ten thousand ground motion fields are generated for the given rupture, taking into consideration both the inter-event and intra-event variability in the ground motion. The ground motion prediction equation used is Boore and Atkinson (2008).

The purpose of this case is to test the computation of the mean and standard deviation of the loss, given variability in both the ground motion values and no variability in the vulnerability function.

\begin{table}[htbp]

\centering
\begin{tabular}{ l c c l }

\hline
\rowcolor{anti-flashwhite}
\bf{GMF #} & \bf{Site} & \bf{IMT} & \bf{GMV}\\
\hline
1 & 1 & PGA & 1.3495 \\
2 & 1 & PGA & 0.5393 \\
3 & 1 & PGA & 0.5240 \\
4 & 1 & PGA & 1.0385 \\
\vdots & \vdots & \vdots & \vdots \\
10000 & 1 & PGA & 0.1327 \\
\hline
\end{tabular}

\caption{Ground motion fields for the single asset tests}
\label{tab:scenario-gmfs-single}
\end{table}

Table \ref{tab:scenario-gmfs-single} lists five of the ten thousand ground motion values generated by OpenQuake.

Since the mean loss ratios in the vulnerability function are not a linear function of the intensity measure levels, an analytical solution for the mean and standard deviation of loss for the scenario cannot be found. Thus, in order to check the OpenQuake results, an alternate implementation of the calculator algorithm in the programming language Julia is used.

The mean and standard deviation of the logarithm of the ground motion calculated at the location of the asset as obtained by using the Boore and Atkinson (2008) equation are $-0.648$ and $0.564$ respectively. Assuming a lognormal distribution for the variability in the ground motion, one million ground motion values are generated using Julia with these logarithmic mean and standard deviation values.

The variability in the loss ratios is zero for the vulnerability function used in this test case. Thus, the loss ratio for each simulated ground motion value is obtained through interpolation on the mean loss ratios provided by the vulnerability function. The mean and standard deviation of loss ratio for the scenario are estimated simply as the mean and standard deviation of the million simulated loss ratios.

\begin{table}[htbp]

\centering
\begin{tabular}{ l l l r r r }

\hline
\rowcolor{anti-flashwhite}
\bf{Asset} & \bf{Damage State} & \bf{Result} & \bf{Expected} & \bf{OpenQuake} & \bf{Difference}\\
\hline
a1 & No damage & Mean &  &  & \% \\
   &           & Std. &  &  & \% \\
   & Slight    & Mean &  &  & \% \\
   &           & Std. &  &  & \% \\
   & Moderate  & Mean &  &  & \% \\
   &           & Std. &  &  & \% \\
   & Extreme   & Mean &  &  & \% \\
   &           & Std. &  &  & \% \\
   & Complete  & Mean &  &  & \% \\
   &           & Std. &  &  & \% \\
\hline
\end{tabular}

\caption{Results for scenario damage test case 1c}
\label{tab:result-scenario-damage-1c}
\end{table}

Table \ref{tab:result-scenario-damage-1c} shows the comparison of the OpenQuake result with the expected result.

% ---------------------------------------------------------------------------
\subsubsection{Case 1d}
The purpose of this case is to test the computation of the mean and standard deviation of the loss, given variability in both the ground motion values and in the vulnerability function.

Similar to case 1c, one million ground motion values are generated using Julia with the logarithmic mean and standard deviation of the ground motion value at the location of the asset as obtained by using the Boore and Atkinson (2008).

The mean loss ratio and standard deviation of loss ratio for each simulated ground motion value are obtained through interpolation on the mean loss ratios and corresponding coefficients of variation provided by the vulnerability function. Using the interpolated mean and standard deviation of loss ratios, one loss ratio is sampled for each ground motion value, assuming a lognormal distibution.

The mean and standard deviation of loss ratio for the scenario are estimated simply as the mean and standard deviation of the million simulated loss ratios.

\input{qareport/results/tab-result-scenario-damage-1d}

Table \ref{tab:result-scenario-damage-1d} shows the comparison of the OpenQuake result with the expected result.

% ---------------------------------------------------------------------------
\subsubsection{Case 2a}


\input{qareport/results/tab-result-scenario-damage-2a}

Table \ref{tab:result-scenario-damage-2a} shows the comparison of the OpenQuake result with the expected result.

% ---------------------------------------------------------------------------
\subsubsection{Case 3a}


\input{qareport/results/tab-result-scenario-damage-3a}

Table \ref{tab:result-scenario-damage-3a} shows the comparison of the OpenQuake result with the expected result.

% -----------------------------------------------------------------------------
\subsection{Multiple asset tests}
\label{subsec:acc-sd-multiple}
The multiple asset test cases are designed to test the loss aggregation functions of the scenario damage calculator, such as:

\begin{itemize}
\item damage distribution per taxonomy
\item damage distribution for the portfolio
\end{itemize}

% ---------------------------------------------------------------------------
\subsubsection{Case 2a}
The purpose of this case is to test the basic elements of a scenario damage calculation involving multiple assets, such as the computation of the mean and standard deviation of the number of buildings in each damage state for assets aggregated by taxonomy, and also for the overall portfolio comprising all assets.

The list of assets in the exposure model used in this case is given in Table~\ref{tab:assets-tax3}.

Five precomputed ground motion fields are used as the starting point for this case. These ground motion fields take into consideration both the inter-event and intra-event variability in the ground motion. The ground motion prediction equation used is \citet{boore2008}, and the \citet{jayaram2009} model for spatial correlation of ground motion values is applied. Table~\ref{tab:gmfs-diff-l7-5} lists the ground motion fields used in this test case.

\begin{table}[htbp]

\centering
\begin{tabular}{ l c c c }

\hline
\rowcolor{anti-flashwhite}
\bf{Taxonomy} & \bf{LS} & \bf{Mean IML} & \bf{Std. IML} \\
\hline
\bf{tax1} & \bf{ds1} & 0.50 & 0.40 \\
          & \bf{ds2} & 1.00 & 0.80 \\
          & \bf{ds3} & 1.50 & 1.20 \\
          & \bf{ds4} & 2.00 & 1.60 \\
\hline
\bf{tax2} & \bf{ds1} & 1.00 & 0.80 \\
          & \bf{ds2} & 1.50 & 1.20 \\
          & \bf{ds3} & 2.50 & 2.00 \\
          & \bf{ds4} & 4.00 & 3.20 \\
\hline
\bf{tax3} & \bf{ds1} & 1.20 & 0.90 \\
          & \bf{ds2} & 1.80 & 1.50 \\
          & \bf{ds3} & 3.00 & 2.00 \\
          & \bf{ds4} & 5.00 & 3.50 \\
\hline
\end{tabular}

\caption{Fragility functions for three taxonomies}
\label{tab:ff-cont-tax3-zmin}
\end{table}

Table~\ref{tab:ff-cont-tax3-zmin} shows the parameters for the continuous lognormal fragility functions for the three taxonomies.

Consider asset $a3$, which has the taxonomy $tax1$.  The ground motion values at the location of the single asset are $[0.15, 0.05, 0.05, 0.15, 0.15] g$. Consider the first value of $PGA = 0.15 g$. The exceedance probabilities for the four damage states $ds_1, ds_2, ds_3, ds_4$ are obtained by employing the equation for the cumulative distribution function for the lognormal distribution, using the parameters specified by the fragility function for $tax1$. This process is described in Case~1b, and is not repeated here. The damage state probabilities are multiplied by the number of units comprising asset $a3$, which in this case is one. This gives us the expected number of buildings for asset $a3$ in each damage state for the five ground motions:

\begin{itemize}
	\item GMF1: $[0.9131, 0.0774, 0.0077, 0.0013, 0.0004]$
	\item GMF2: $[0.9982, 0.0018, 0.0000, 0.0000, 0.0000]$
	\item GMF3: $[0.9982, 0.0018, 0.0000, 0.0000, 0.0000]$
	\item GMF4: $[0.9131, 0.0774, 0.0077, 0.0013, 0.0004]$
	\item GMF5: $[0.9131, 0.0774, 0.0077, 0.0013, 0.0004]$
\end{itemize}

The mean number of buildings with no damage for asset $a3$ is thus calculated as $(0.9131 + 0.9982 + 0.9982 + 0.9131 + 0.9131)/5 = 0.94716$. The standard deviation of the number of buildings with no damage for asset $a3$ is $0.04661$. Similarly, the mean and standard deviation of the number of buildings with no damage for the other assets are calculated and listed below:

\begin{itemize}
	\item $a1$ (tax1): mean = $0.28370$; std.dev = $0.29192$
	\item $a2$ (tax2): mean = $0.89301$; std.dev = $0.11739$
	\item $a3$ (tax1): mean = $0.94716$; std.dev = $0.04661$
	\item $a4$ (tax3): mean = $0.61303$; std.dev = $0.24347$
	\item $a5$ (tax1): mean = $0.59337$; std.dev = $0.41368$
	\item $a6$ (tax2): mean = $0.77733$; std.dev = $0.18346$
	\item $a7$ (tax1): mean = $0.65092$; std.dev = $0.24269$
\end{itemize}

Repeating this for each of the other damage states, we can also compute the mean and standard deviation of the number of buildings in each damage state for the seven assets. Table \ref{tab:result-sd-2a} shows the comparison of the OpenQuake result with the expected result for assets $a1$, $a2$, and $a3$.

\begin{table}[htbp]

\centering
\begin{tabular}{ l l l r r r }

\hline
\rowcolor{anti-flashwhite}
\bf{Asset} & \bf{Damage State} & \bf{Result} & \bf{Expected} & \bf{OpenQuake} & \bf{Difference}\\
\hline
a1 & none & Mean & 0.2837 & 0.2837 & 0.00\% \\
   &      & Std. & 0.2919 & 0.2919 & 0.00\% \\
   & ds1  & Mean & 0.2625 & 0.2625 & 0.00\% \\
   &      & Std. & 0.1002 & 0.1002 & 0.00\% \\
   & ds2  & Mean & 0.1568 & 0.1568 & 0.00\% \\
   &      & Std. & 0.0767 & 0.0767 & 0.00\% \\
   & ds3  & Mean & 0.0962 & 0.0962 & 0.00\% \\
   &      & Std. & 0.0629 & 0.0629 & 0.00\% \\
   & ds4  & Mean & 0.2008 & 0.2008 & 0.00\% \\
   &      & Std. & 0.2159 & 0.2159 & 0.00\% \\
\hline
a2 & none & Mean & 0.8930 & 0.8930 & 0.00\% \\
   &      & Std. & 0.1174 & 0.1174 & 0.00\% \\
   & ds1  & Mean & 0.0653 & 0.0653 & 0.00\% \\
   &      & Std. & 0.0666 & 0.0666 & 0.00\% \\
   & ds2  & Mean & 0.0328 & 0.0328 & 0.00\% \\
   &      & Std. & 0.0392 & 0.0392 & 0.00\% \\
   & ds3  & Mean & 0.0074 & 0.0074 & 0.00\% \\
   &      & Std. & 0.0096 & 0.0096 & 0.00\% \\
   & ds4  & Mean & 0.0014 & 0.0014 & 0.00\% \\
   &      & Std. & 0.0019 & 0.0019 & 0.00\% \\
\hline
a3 & none & Mean & 0.9472 & 0.9472 & 0.00\% \\
   &      & Std. & 0.0466 & 0.0466 & 0.00\% \\
   & ds1  & Mean & 0.0471 & 0.0471 & 0.00\% \\
   &      & Std. & 0.0415 & 0.0415 & 0.00\% \\
   & ds2  & Mean & 0.0047 & 0.0047 & 0.00\% \\
   &      & Std. & 0.0042 & 0.0042 & 0.00\% \\
   & ds3  & Mean & 0.0008 & 0.0008 & 0.00\% \\
   &      & Std. & 0.0007 & 0.0007 & 0.00\% \\
   & ds4  & Mean & 0.0003 & 0.0003 & 0.00\% \\
   &      & Std. & 0.0002 & 0.0002 & 0.00\% \\
\hline
\end{tabular}

\caption{Results for scenario damage test case 2a — individual assets}
\label{tab:result-sd-2a}
\end{table}

Now, aggregating the expected number of buildings of taxonomy $tax1$ with no damage for each of the five ground motions, we have the following for $tax1$: $0.28370 + 0.94716 + 0.59337 + 0.65092 = 2.47515$. Repeating this for each of the other damage states, we can also compute the expected total number of buildings in each damage state for assets of taxonomy $tax1$. Table \ref{tab:result-sd-2a-tax} shows the comparison of the OpenQuake result with the expected result for taxonomies $tax1$, $tax2$, and $tax3$.

\input{qareport/results/sd-2a-tax}

Finally, aggregating the expected number of buildings with no damage for across all taxonomies for each of the five ground motions, we have the following for the overall portfolio: $0.28370 + 0.89301 + 0.94716 + 0.61303 + 0.59337 + 0.77733 + 0.65092 = 4.75852$. Repeating this for each of the other damage states, we can also compute the expected total number of buildings in each damage state for the overall portfolio. Table \ref{tab:result-sd-2a-total} shows the comparison of the OpenQuake result with the expected result for the overall portfolio.

\input{qareport/results/sd-2a-total}




% -----------------------------------------------------------------------------
\subsection{Calculation with logic-trees}
\label{subsec:acc-sd-lt}
% ---------------------------------------------------------------------------
\subsubsection{Case 3a}
\begin{table}[htbp]

\centering
\begin{tabular}{ l l l l r r r }

\hline
\rowcolor{anti-flashwhite}
\bf{Branch} & \bf{Asset} & \bf{Damage State} & \bf{Result} & \bf{Julia} & \bf{OpenQuake} & \bf{Difference}\\
\hline
BA2008 & a1 & none & Mean & 0.3725 & 0.3736 & -0.30\% \\
       &    &      & Std. & 0.2420 & 0.2403 & 0.69\% \\
       &    & ds1  & Mean & 0.2988 & 0.2998 & -0.31\% \\
       &    &      & Std. & 0.0838 & 0.0824 & 1.64\% \\
       &    & ds2  & Mean & 0.1428 & 0.1426 & 0.15\% \\
       &    &      & Std. & 0.0676 & 0.0673 & 0.44\% \\
       &    & ds3  & Mean & 0.0732 & 0.0728 & 0.47\% \\
       &    &      & Std. & 0.0504 & 0.0503 & 0.27\% \\
       &    & ds4  & Mean & 0.1127 & 0.1112 & 1.33\% \\
       &    &      & Std. & 0.1367 & 0.1345 & 1.56\% \\
\hline
CY2008 & a1 & none & Mean & 0.3086 & 0.3093 & -0.25\% \\
       &    &      & Std. & 0.2218 & 0.2197 & 0.92\% \\
       &    & ds1  & Mean & 0.3019 & 0.3032 & -0.45\% \\
       &    &      & Std. & 0.0814 & 0.0799 & 1.92\% \\
       &    & ds2  & Mean & 0.1590 & 0.1590 & -0.02\% \\
       &    &      & Std. & 0.0617 & 0.0613 & 0.61\% \\
       &    & ds3  & Mean & 0.0862 & 0.0859 & 0.34\% \\
       &    &      & Std. & 0.0500 & 0.0498 & 0.40\% \\
       &    & ds4  & Mean & 0.1444 & 0.1426 & 1.30\% \\
       &    &      & Std. & 0.1535 & 0.1512 & 1.53\% \\
\hline
\end{tabular}

\caption{Results for scenario damage test case 3a}
\label{tab:result-sd-3a}
\end{table}

Table \ref{tab:result-sd-3a} shows the comparison of the OpenQuake results with the expected results.
