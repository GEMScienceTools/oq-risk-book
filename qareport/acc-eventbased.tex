The tests for the event-based risk calculator assume the correct computation of the ground motion fields at the locations of the assets in the exposure model. Thus, the risk tests implicitly rely on the acceptance tests for the event-based hazard calculator.

The source model used for the tests comprises a single vertical strike-slip fault with a Gutenberg-Richter b-value equal to 0.9 and a slip rate of 2 mm/yr. The MFD is a Gutenberg-Richter distribution truncated between magnitudes 5.0 and 6.5, while the Ground Motion Prediction Equation (GMPE) used is Sadigh et al. (1997), with sigma set to zero.\\

\noindent Details of the fault geometry are given below:\\

\noindent
Fault type: Strike slip\\
Fault dip: $90^{\circ}$\\
Fault plane depths: 0--12 km\\
Fault coordinates:\\
South end: $38.0000^{\circ} N$, $122.0000^{\circ} W$\\
North end: $38.2248^{\circ} N$, $122.0000^{\circ} W$\\

\begin{table}

\centering
\begin{tabular}{ c l c c l }

\hline
\rowcolor{anti-flashwhite}
\bf{Site} & \bf{Taxonomy} & \bf{Latitude} & \bf{Longitude} & \bf{Comment} \\
\hline
1 & Wood & 38.113 & -122.000 & On fault midpoint, along strike \\
2 & Wood & 38.113 & -122.114 & 10 km west of fault, at midpoint \\
3 & RC & 38.113 & -122.570 & 50 km west of fault, at midpoint \\
4 & RC & 38.000 & -122.000 & South end of fault \\
5 & Steel & 37.910 & -122.000 & 10 km south of fault, along strike \\
6 & Wood & 38.225 & -122.000 & North end of fault \\
7 & Steel & 38.113 & -121.886 & 10 km east of fault, at midpoint \\
\hline
\end{tabular}

\caption{Asset sites and taxonomies for the multiple-asset test cases}
\label{tab:assets}
\end{table}

The list of assets in the exposure model used for the multiple-asset test cases for the event-based risk calculator is given in Table \ref{tab:assets}. The complete collection of input models and job configuration files used in these test cases can be accessed here:
\href{https://github.com/gem/oq-risklib/tree/master/openquake/qa_tests_data/event_based_risk}
{https://github.com/gem/oq-risklib/tree/master/openquake/qa\_tests\_data/event\_based\_risk}

\subsection{Single asset tests}
\label{subsec:eventbased-single}
The single asset test cases are designed to test the basic elements of the event-based risk calculator, such as:

\begin{itemize}
\item asset event loss table computation
\item asset loss ratio exceedance curve computation
\item asset loss exceedance curve computation
\end{itemize}

\begin{table}

\centering
\begin{tabular}{ c l c c l }

\hline
\rowcolor{anti-flashwhite}
\bf{Site} & \bf{Taxonomy} & \bf{Latitude} & \bf{Longitude} & \bf{Comment} \\
\hline
1 & Wood & 38.113 & -122.000 & On fault midpoint, along strike \\
\hline
\end{tabular}

\caption{Asset location and taxonomy for the single-asset test cases}
\label{tab:asset}
\end{table}

The location and taxonomy of the single asset in the exposure model used for the single-asset test cases for the event-based risk calculator are given in Table \ref{tab:asset}.

% -----------------------------------------------------------------------------
\subsection{Multiple asset tests}
\label{subsec:eventbased-multiple}
The multiple asset test cases are designed to test the loss aggregation functions of the event-based risk calculator, such as:

\begin{itemize}
\item portfolio loss computation for a given ground motion field
\item calculation of portfolio loss exceedance curves
\end{itemize}

% -----------------------------------------------------------------------------
\subsection{Calculation with logic-trees}
\label{subsec:eventbased-logictrees}
\include{qareport/acc-eventbased-logictrees}

% -----------------------------------------------------------------------------
\subsection{Insurance tests}
\label{subsec:eventbased-insurance}
\include{qareport/acc-eventbased-insurance}
