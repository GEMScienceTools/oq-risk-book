The single asset test cases are designed to test the basic elements of the classical-PSHA based risk calculator, such as:

\begin{itemize}
\item asset loss ratio exceedance curve computation
\item asset loss exceedance curve computation
\end{itemize}

The location and taxonomy of the single asset in the exposure model used for the single-asset test cases for the classical risk calculator are given in Table \ref{tab:assets}.

% ---------------------------------------------------------------------------
\subsubsection{Case 1a}
Test case 1a 

\begin{table}[htbp]

\centering
\begin{tabular}{ l r r r }

\hline
\rowcolor{anti-flashwhite}
\bf{Result} & \bf{Expected} & \bf{OpenQuake} & \bf{Difference}\\
\hline
Average loss &  &  & 0\% \\
\hline
\end{tabular}

\caption{Results for classical risk test case 1a}
\label{tab:result-classical-risk-1a}
\end{table}

Table \ref{tab:result-classical-risk-1a} shows the comparison of the OpenQuake result with the expected result.

% ---------------------------------------------------------------------------
\subsubsection{Case 1b}
Test case 1b 

\begin{table}[htbp]

\centering
\begin{tabular}{ l r r r }

\hline
\rowcolor{anti-flashwhite}
\bf{Result} & \bf{Expected} & \bf{OpenQuake} & \bf{Difference}\\
\hline
Average loss & 45.55 & 45.55 & 0\% \\
\hline
\end{tabular}

\caption{Results for classical risk test case 1b}
\label{tab:result-classical-risk-1b}
\end{table}

Table \ref{tab:result-classical-risk-1b} shows the comparison of the OpenQuake result with the expected result.

% ---------------------------------------------------------------------------
\subsubsection{Case 1c}


\begin{table}[htbp]

\centering
\begin{tabular}{ l r r r }

\hline
\rowcolor{anti-flashwhite}
\bf{Result} & \bf{Expected} & \bf{OpenQuake} & \bf{Difference}\\
\hline
Average loss &  &  & 0\% \\
\hline
\end{tabular}

\caption{Results for classical risk test case 1c}
\label{tab:result-classical-risk-1c}
\end{table}

Table \ref{tab:result-classical-risk-1c} shows the comparison of the OpenQuake result with the expected result.

% ---------------------------------------------------------------------------
\subsubsection{Case 1d}


\begin{table}[htbp]

\centering
\begin{tabular}{ l r r r }

\hline
\rowcolor{anti-flashwhite}
\bf{Result} & \bf{Expected} & \bf{OpenQuake} & \bf{Difference}\\
\hline
Average loss &  &  & 0\% \\
\hline
\end{tabular}

\caption{Results for classical risk test case 1d}
\label{tab:result-classical-risk-1d}
\end{table}

Table \ref{tab:result-classical-risk-1d} shows the comparison of the OpenQuake result with the expected result.

% ---------------------------------------------------------------------------
\subsubsection{Case 1e}


\begin{table}[htbp]

\centering
\begin{tabular}{ l r r r }

\hline
\rowcolor{anti-flashwhite}
\bf{Result} & \bf{Expected} & \bf{OpenQuake} & \bf{Difference}\\
\hline
Average loss & 46.07 & 46.07 & 0\% \\
\hline
\end{tabular}

\caption{Results for classical risk test case 1e}
\label{tab:result-classical-risk-1e}
\end{table}

Table \ref{tab:result-classical-risk-1e} shows the comparison of the OpenQuake result with the expected result.

% ---------------------------------------------------------------------------
\subsubsection{Case 2a}


\begin{table}[htbp]

\centering
\begin{tabular}{ l r r r }

\hline
\rowcolor{anti-flashwhite}
\bf{Result} & \bf{Expected} & \bf{OpenQuake} & \bf{Difference}\\
\hline
Average loss &  &  & 0\% \\
\hline
\end{tabular}

\caption{Results for classical risk test case 2a}
\label{tab:result-classical-risk-2a}
\end{table}

Table \ref{tab:result-classical-risk-2a} shows the comparison of the OpenQuake result with the expected result.

% ---------------------------------------------------------------------------
\subsubsection{Case 2b}


\begin{table}[htbp]

\centering
\begin{tabular}{ l r r r }

\hline
\rowcolor{anti-flashwhite}
\bf{Result} & \bf{Expected} & \bf{OpenQuake} & \bf{Difference}\\
\hline
Average loss &  &  & 0\% \\
\hline
\end{tabular}

\caption{Results for classical risk test case 2b}
\label{tab:result-classical-risk-2b}
\end{table}

Table \ref{tab:result-classical-risk-2b} shows the comparison of the OpenQuake result with the expected result.

% ---------------------------------------------------------------------------
\subsubsection{Case 2c}


\begin{table}[htbp]

\centering
\begin{tabular}{ l r r r }

\hline
\rowcolor{anti-flashwhite}
\bf{Result} & \bf{Expected} & \bf{OpenQuake} & \bf{Difference}\\
\hline
Average loss & 9.11 & 9.11 & 0\% \\
\hline
\end{tabular}

\caption{Results for classical risk test case 2c}
\label{tab:result-classical-risk-2c}
\end{table}

Table \ref{tab:result-classical-risk-2c} shows the comparison of the OpenQuake result with the expected result.

% ---------------------------------------------------------------------------
\subsubsection{Case 2d}


\begin{table}[htbp]

\centering
\begin{tabular}{ l r r r }

\hline
\rowcolor{anti-flashwhite}
\bf{Result} & \bf{Expected} & \bf{OpenQuake} & \bf{Difference}\\
\hline
Average loss & $8.78 \times 10^{-4}$ & $8.78 \times 10^{-4}$ & 0\% \\
\hline
\end{tabular}

\caption{Results for classical risk test case 2d}
\label{tab:result-classical-risk-2d}
\end{table}

Table \ref{tab:result-classical-risk-2d} shows the comparison of the OpenQuake result with the expected result.

% ---------------------------------------------------------------------------
\subsubsection{Case 3a}


\begin{table}[htbp]

\centering
\begin{tabular}{ l r r r }

\hline
\rowcolor{anti-flashwhite}
\bf{Result} & \bf{Expected} & \bf{OpenQuake} & \bf{Difference}\\
\hline
Average loss &  &  & 0\% \\
\hline
\end{tabular}

\caption{Results for classical risk test case 3a}
\label{tab:result-classical-risk-3a}
\end{table}

Table \ref{tab:result-classical-risk-3a} shows the comparison of the OpenQuake result with the expected result.

% ---------------------------------------------------------------------------
\subsubsection{Case 4a}


\begin{table}[htbp]

\centering
\begin{tabular}{ l r r r }

\hline
\rowcolor{anti-flashwhite}
\bf{Result} & \bf{Expected} & \bf{OpenQuake} & \bf{Difference}\\
\hline
Average loss & 45.55 & 45.55 & 0\% \\
\hline
\end{tabular}

\caption{Results for classical risk test case 4a}
\label{tab:result-classical-risk-4a}
\end{table}

Table \ref{tab:result-classical-risk-4a} shows the comparison of the OpenQuake result with the expected result.

% ---------------------------------------------------------------------------
\subsubsection{Case 4b}


\begin{table}[htbp]

\centering
\begin{tabular}{ l r r r }

\hline
\rowcolor{anti-flashwhite}
\bf{Result} & \bf{Expected} & \bf{OpenQuake} & \bf{Difference}\\
\hline
Average loss &  &  & 0\% \\
\hline
\end{tabular}

\caption{Results for classical risk test case 4b}
\label{tab:result-classical-risk-4b}
\end{table}

Table \ref{tab:result-classical-risk-4b} shows the comparison of the OpenQuake result with the expected result.

% ---------------------------------------------------------------------------
\subsubsection{Case 4c}


\begin{table}[htbp]

\centering
\begin{tabular}{ l r r r }

\hline
\rowcolor{anti-flashwhite}
\bf{Result} & \bf{Expected} & \bf{OpenQuake} & \bf{Difference}\\
\hline
Average loss &  &  & 0\% \\
\hline
\end{tabular}

\caption{Results for classical risk test case 4c}
\label{tab:result-classical-risk-4c}
\end{table}

Table \ref{tab:result-classical-risk-4c} shows the comparison of the OpenQuake result with the expected result.

% ---------------------------------------------------------------------------
\subsubsection{Case 5a}


\begin{table}[htbp]

\centering
\begin{tabular}{ l r r r }

\hline
\rowcolor{anti-flashwhite}
\bf{Result} & \bf{Expected} & \bf{OpenQuake} & \bf{Difference}\\
\hline
Average loss &  &  & 0\% \\
\hline
\end{tabular}

\caption{Results for classical risk test case 5a}
\label{tab:result-classical-risk-5a}
\end{table}

Table \ref{tab:result-classical-risk-5a} shows the comparison of the OpenQuake result with the expected result.

% ---------------------------------------------------------------------------
\subsubsection{Case 5b}


\begin{table}[htbp]

\centering
\begin{tabular}{ l r r r }

\hline
\rowcolor{anti-flashwhite}
\bf{Result} & \bf{Expected} & \bf{OpenQuake} & \bf{Difference}\\
\hline
Average loss & 68.32 & 68.32 & 0\% \\
\hline
\end{tabular}

\caption{Results for classical risk test case 5b}
\label{tab:result-classical-risk-5b}
\end{table}

Table \ref{tab:result-classical-risk-5b} shows the comparison of the OpenQuake result with the expected result.

% ---------------------------------------------------------------------------
\subsubsection{Case 5c}


\begin{table}[htbp]

\centering
\begin{tabular}{ l r r r }

\hline
\rowcolor{anti-flashwhite}
\bf{Result} & \bf{Expected} & \bf{OpenQuake} & \bf{Difference}\\
\hline
Average loss &  &  & 0\% \\
\hline
\end{tabular}

\caption{Results for classical risk test case 5c}
\label{tab:result-classical-risk-5c}
\end{table}

Table \ref{tab:result-classical-risk-5c} shows the comparison of the OpenQuake result with the expected result.

% ---------------------------------------------------------------------------
\subsubsection{Case 5d}


\begin{table}[htbp]

\centering
\begin{tabular}{ l r r r }

\hline
\rowcolor{anti-flashwhite}
\bf{Result} & \bf{Expected} & \bf{OpenQuake} & \bf{Difference}\\
\hline
Average loss &  &  & 0\% \\
\hline
\end{tabular}

\caption{Results for classical risk test case 5d}
\label{tab:result-classical-risk-5d}
\end{table}

Table \ref{tab:result-classical-risk-5d} shows the comparison of the OpenQuake result with the expected result.